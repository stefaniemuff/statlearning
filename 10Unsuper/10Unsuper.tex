% Options for packages loaded elsewhere
\PassOptionsToPackage{unicode}{hyperref}
\PassOptionsToPackage{hyphens}{url}
\PassOptionsToPackage{dvipsnames,svgnames,x11names}{xcolor}
%
\documentclass[
  10pt,
  ignorenonframetext,
]{beamer}
\usepackage{pgfpages}
\setbeamertemplate{caption}[numbered]
\setbeamertemplate{caption label separator}{: }
\setbeamercolor{caption name}{fg=normal text.fg}
\beamertemplatenavigationsymbolsempty
% Prevent slide breaks in the middle of a paragraph
\widowpenalties 1 10000
\raggedbottom
\setbeamertemplate{part page}{
  \centering
  \begin{beamercolorbox}[sep=16pt,center]{part title}
    \usebeamerfont{part title}\insertpart\par
  \end{beamercolorbox}
}
\setbeamertemplate{section page}{
  \centering
  \begin{beamercolorbox}[sep=12pt,center]{part title}
    \usebeamerfont{section title}\insertsection\par
  \end{beamercolorbox}
}
\setbeamertemplate{subsection page}{
  \centering
  \begin{beamercolorbox}[sep=8pt,center]{part title}
    \usebeamerfont{subsection title}\insertsubsection\par
  \end{beamercolorbox}
}
\AtBeginPart{
  \frame{\partpage}
}
\AtBeginSection{
  \ifbibliography
  \else
    \frame{\sectionpage}
  \fi
}
\AtBeginSubsection{
  \frame{\subsectionpage}
}
\usepackage{amsmath,amssymb}
\usepackage{lmodern}
\usepackage{iftex}
\ifPDFTeX
  \usepackage[T1]{fontenc}
  \usepackage[utf8]{inputenc}
  \usepackage{textcomp} % provide euro and other symbols
\else % if luatex or xetex
  \usepackage{unicode-math}
  \defaultfontfeatures{Scale=MatchLowercase}
  \defaultfontfeatures[\rmfamily]{Ligatures=TeX,Scale=1}
\fi
\usetheme[]{Singapore}
\usefonttheme{serif}
% Use upquote if available, for straight quotes in verbatim environments
\IfFileExists{upquote.sty}{\usepackage{upquote}}{}
\IfFileExists{microtype.sty}{% use microtype if available
  \usepackage[]{microtype}
  \UseMicrotypeSet[protrusion]{basicmath} % disable protrusion for tt fonts
}{}
\makeatletter
\@ifundefined{KOMAClassName}{% if non-KOMA class
  \IfFileExists{parskip.sty}{%
    \usepackage{parskip}
  }{% else
    \setlength{\parindent}{0pt}
    \setlength{\parskip}{6pt plus 2pt minus 1pt}}
}{% if KOMA class
  \KOMAoptions{parskip=half}}
\makeatother
\usepackage{xcolor}
\newif\ifbibliography
\usepackage{color}
\usepackage{fancyvrb}
\newcommand{\VerbBar}{|}
\newcommand{\VERB}{\Verb[commandchars=\\\{\}]}
\DefineVerbatimEnvironment{Highlighting}{Verbatim}{commandchars=\\\{\}}
% Add ',fontsize=\small' for more characters per line
\usepackage{framed}
\definecolor{shadecolor}{RGB}{248,248,248}
\newenvironment{Shaded}{\begin{snugshade}}{\end{snugshade}}
\newcommand{\AlertTok}[1]{\textcolor[rgb]{0.94,0.16,0.16}{#1}}
\newcommand{\AnnotationTok}[1]{\textcolor[rgb]{0.56,0.35,0.01}{\textbf{\textit{#1}}}}
\newcommand{\AttributeTok}[1]{\textcolor[rgb]{0.77,0.63,0.00}{#1}}
\newcommand{\BaseNTok}[1]{\textcolor[rgb]{0.00,0.00,0.81}{#1}}
\newcommand{\BuiltInTok}[1]{#1}
\newcommand{\CharTok}[1]{\textcolor[rgb]{0.31,0.60,0.02}{#1}}
\newcommand{\CommentTok}[1]{\textcolor[rgb]{0.56,0.35,0.01}{\textit{#1}}}
\newcommand{\CommentVarTok}[1]{\textcolor[rgb]{0.56,0.35,0.01}{\textbf{\textit{#1}}}}
\newcommand{\ConstantTok}[1]{\textcolor[rgb]{0.00,0.00,0.00}{#1}}
\newcommand{\ControlFlowTok}[1]{\textcolor[rgb]{0.13,0.29,0.53}{\textbf{#1}}}
\newcommand{\DataTypeTok}[1]{\textcolor[rgb]{0.13,0.29,0.53}{#1}}
\newcommand{\DecValTok}[1]{\textcolor[rgb]{0.00,0.00,0.81}{#1}}
\newcommand{\DocumentationTok}[1]{\textcolor[rgb]{0.56,0.35,0.01}{\textbf{\textit{#1}}}}
\newcommand{\ErrorTok}[1]{\textcolor[rgb]{0.64,0.00,0.00}{\textbf{#1}}}
\newcommand{\ExtensionTok}[1]{#1}
\newcommand{\FloatTok}[1]{\textcolor[rgb]{0.00,0.00,0.81}{#1}}
\newcommand{\FunctionTok}[1]{\textcolor[rgb]{0.00,0.00,0.00}{#1}}
\newcommand{\ImportTok}[1]{#1}
\newcommand{\InformationTok}[1]{\textcolor[rgb]{0.56,0.35,0.01}{\textbf{\textit{#1}}}}
\newcommand{\KeywordTok}[1]{\textcolor[rgb]{0.13,0.29,0.53}{\textbf{#1}}}
\newcommand{\NormalTok}[1]{#1}
\newcommand{\OperatorTok}[1]{\textcolor[rgb]{0.81,0.36,0.00}{\textbf{#1}}}
\newcommand{\OtherTok}[1]{\textcolor[rgb]{0.56,0.35,0.01}{#1}}
\newcommand{\PreprocessorTok}[1]{\textcolor[rgb]{0.56,0.35,0.01}{\textit{#1}}}
\newcommand{\RegionMarkerTok}[1]{#1}
\newcommand{\SpecialCharTok}[1]{\textcolor[rgb]{0.00,0.00,0.00}{#1}}
\newcommand{\SpecialStringTok}[1]{\textcolor[rgb]{0.31,0.60,0.02}{#1}}
\newcommand{\StringTok}[1]{\textcolor[rgb]{0.31,0.60,0.02}{#1}}
\newcommand{\VariableTok}[1]{\textcolor[rgb]{0.00,0.00,0.00}{#1}}
\newcommand{\VerbatimStringTok}[1]{\textcolor[rgb]{0.31,0.60,0.02}{#1}}
\newcommand{\WarningTok}[1]{\textcolor[rgb]{0.56,0.35,0.01}{\textbf{\textit{#1}}}}
\usepackage{graphicx}
\makeatletter
\def\maxwidth{\ifdim\Gin@nat@width>\linewidth\linewidth\else\Gin@nat@width\fi}
\def\maxheight{\ifdim\Gin@nat@height>\textheight\textheight\else\Gin@nat@height\fi}
\makeatother
% Scale images if necessary, so that they will not overflow the page
% margins by default, and it is still possible to overwrite the defaults
% using explicit options in \includegraphics[width, height, ...]{}
\setkeys{Gin}{width=\maxwidth,height=\maxheight,keepaspectratio}
% Set default figure placement to htbp
\makeatletter
\def\fps@figure{htbp}
\makeatother
\setlength{\emergencystretch}{3em} % prevent overfull lines
\providecommand{\tightlist}{%
  \setlength{\itemsep}{0pt}\setlength{\parskip}{0pt}}
\setcounter{secnumdepth}{-\maxdimen} % remove section numbering
\newlength{\cslhangindent}
\setlength{\cslhangindent}{1.5em}
\newlength{\csllabelwidth}
\setlength{\csllabelwidth}{3em}
\newlength{\cslentryspacingunit} % times entry-spacing
\setlength{\cslentryspacingunit}{\parskip}
\newenvironment{CSLReferences}[2] % #1 hanging-ident, #2 entry spacing
 {% don't indent paragraphs
  \setlength{\parindent}{0pt}
  % turn on hanging indent if param 1 is 1
  \ifodd #1
  \let\oldpar\par
  \def\par{\hangindent=\cslhangindent\oldpar}
  \fi
  % set entry spacing
  \setlength{\parskip}{#2\cslentryspacingunit}
 }%
 {}
\usepackage{calc}
\newcommand{\CSLBlock}[1]{#1\hfill\break}
\newcommand{\CSLLeftMargin}[1]{\parbox[t]{\csllabelwidth}{#1}}
\newcommand{\CSLRightInline}[1]{\parbox[t]{\linewidth - \csllabelwidth}{#1}\break}
\newcommand{\CSLIndent}[1]{\hspace{\cslhangindent}#1}
\usepackage{multirow}
\ifLuaTeX
  \usepackage{selnolig}  % disable illegal ligatures
\fi
\IfFileExists{bookmark.sty}{\usepackage{bookmark}}{\usepackage{hyperref}}
\IfFileExists{xurl.sty}{\usepackage{xurl}}{} % add URL line breaks if available
\urlstyle{same} % disable monospaced font for URLs
\hypersetup{
  pdftitle={Module 10: Unsupervised learning (Overview/quizz lecture)},
  pdfauthor={Stefanie Muff, Department of Mathematical Sciences, NTNU},
  colorlinks=true,
  linkcolor={Maroon},
  filecolor={Maroon},
  citecolor={Blue},
  urlcolor={blue},
  pdfcreator={LaTeX via pandoc}}

\title{Module 10: Unsupervised learning (Overview/quizz lecture)}
\subtitle{TMA4268 Statistical Learning V2023}
\author{Stefanie Muff, Department of Mathematical Sciences, NTNU}
\date{March 23, 2023}

\begin{document}
\frame{\titlepage}

\begin{frame}
\begin{block}{The biplot}
\protect\hypertarget{the-biplot}{}
\(~\)

\centering

\includegraphics[width=0.75\textwidth,height=\textheight]{figure121.png}
\end{block}
\end{frame}

\begin{frame}
\begin{block}{PC loadings vectors \(\Phi\)}
\protect\hypertarget{pc-loadings-vectors-phi}{}
\(~\)

\centering

\includegraphics[width=0.85\textwidth,height=\textheight]{table121.png}
\flushleft (Table 2.1)

\(~\)

Loadings vectors
\(\Phi_i=(\Phi_{1j} , \Phi_{2j},\ldots, \Phi_{pj})^\top\): How much does
the respective covariate contribute to PC\(_j\)?
\end{block}
\end{frame}

\begin{frame}[fragile]
\begin{block}{Example from Compulsory 3, 2020}
\protect\hypertarget{example-from-compulsory-3-2020}{}
\(~\)

\begin{itemize}
\item
  We study the \texttt{decathlon2} dataset from the \texttt{factoextra}
  package in R, where Athletes' performance during a sporting meeting
  was recorded.
\item
  We look at 23 athletes and the results from the 10 disciplines in two
  competitions.
\end{itemize}

\(~\)

\scriptsize

\begin{Shaded}
\begin{Highlighting}[]
\NormalTok{decathlon2.active[}\FunctionTok{c}\NormalTok{(}\DecValTok{1}\NormalTok{, }\DecValTok{3}\NormalTok{, }\DecValTok{4}\NormalTok{), ]}
\end{Highlighting}
\end{Shaded}

\begin{verbatim}
##          100m long_jump shot_put high_jump  400m 110.hurdle discus pole_vault
## SEBRLE  11.04      7.58    14.83      2.07 49.81      14.69  43.75       5.02
## BERNARD 11.02      7.23    14.25      1.92 48.93      14.99  40.87       5.32
## YURKOV  11.34      7.09    15.19      2.10 50.42      15.31  46.26       4.72
##         javeline 1500m
## SEBRLE     63.19 291.7
## BERNARD    62.77 280.1
## YURKOV     63.44 276.4
\end{verbatim}
\end{block}
\end{frame}

\begin{frame}
\begin{figure}
\includegraphics[width=0.55\linewidth]{10Unsuper_files/figure-beamer/biplot-1} \end{figure}
\end{frame}

\begin{frame}
\begin{block}{Scree plot}
\protect\hypertarget{scree-plot}{}
\(~\)

A graphical description of the \textbf{proportion of variance explained
(PVE)} by a certain number of PCs (see Fig 12.3 from James et al.
(2021)):

\centering

\includegraphics[width=0.9\textwidth,height=\textheight]{123.png}
\end{block}
\end{frame}

\begin{frame}
\begin{block}{Proportion of varianced explained (PVE)}
\protect\hypertarget{proportion-of-varianced-explained-pve}{}
\(~\)

\textbf{Recap:} The PVE by PC \(m\) is given by

\[
\frac{\sum_{i=1}^m z_{im}^2} {\sum_{j=1}^p\sum_{i=1}^n x_{ij}^2}
\]
\end{block}
\end{frame}

\begin{frame}{Clustering}
\protect\hypertarget{clustering}{}
\(~\)

\begin{itemize}
\item
  The aim is to find \emph{clusters} or \emph{subgroups}.
\item
  Clustering looks for homogeneous subgroups in the data.
\end{itemize}

\(~\)

Difference to PCA?

\pause

\(\rightarrow\) PCA looks for low-dimensional representation of the
data.
\end{frame}

\begin{frame}
\begin{block}{K-means vs.~hierarchical clustering}
\protect\hypertarget{k-means-vs.-hierarchical-clustering}{}
\(~\)

See menti.com
\end{block}
\end{frame}

\begin{frame}
\begin{block}{K-means clustering}
\protect\hypertarget{k-means-clustering}{}
\(~\)

\begin{itemize}
\tightlist
\item
  Fix the number of clusters \(K\).
\end{itemize}

\(~\)

\begin{itemize}
\tightlist
\item
  Find groups such that the sum of the within-cluster variation is
  minimized.
\end{itemize}

\(~\)

\begin{itemize}
\tightlist
\item
  Algorithm?
\end{itemize}
\end{block}
\end{frame}

\begin{frame}
\centering

\includegraphics[width=0.75\textwidth,height=\textheight]{fig12_8.png}
\flushleft \small (Fig 12.8 from course book)
\end{frame}

\begin{frame}
\begin{block}{Hierarchical clustering}
\protect\hypertarget{hierarchical-clustering}{}
\(~\)

Bottom-up agglomerative clustering that results in a
\emph{\textcolor{red}{dendogram}}.

\(~\)

\includegraphics[width=0.9\textwidth,height=\textheight]{hierclust.png}
\end{block}
\end{frame}

\begin{frame}
\begin{block}{Important in hierarchical clustering}
\protect\hypertarget{important-in-hierarchical-clustering}{}
\(~\)

\begin{itemize}
\tightlist
\item
  \emph{\textcolor{red}{Linkage:}} Complete, single, average centroid.
\end{itemize}

\(~\)

\begin{itemize}
\tightlist
\item
  \emph{\textcolor{red}{Dissimilarity measure:}} Euclidian distance,
  correlation. \emph{Other similarity/distance measures?}
  \footnote{ Note: Correlation is actually a similarity measure, not a distance measure. Implication?}
\end{itemize}
\end{block}
\end{frame}

\begin{frame}
\begin{block}{Hierarchical clustering -- example}
\protect\hypertarget{hierarchical-clustering-example}{}
\(~\)

\includegraphics{fig12_12.png}

Note: The representation on the right is not possible in
high-dimensional space (i.e., if we have \(X_1, X_2, X_3, ...., X_p\)).
\end{block}
\end{frame}

\begin{frame}
\begin{block}{Hierarchical clustering -- example}
\protect\hypertarget{hierarchical-clustering-example-1}{}
\(~\)

An exam question from 2022:

\includegraphics{exam_question3b_2023.png}
\end{block}
\end{frame}

\begin{frame}
\begin{block}{Pros and cons of clusterization methods / practical
issues}
\protect\hypertarget{pros-and-cons-of-clusterization-methods-practical-issues}{}
\(~\)

\(~\)

\(~\)

\(~\)

\(~\)

\(~\)

\(~\)

\(~\)

\(~\)

\(~\)
\end{block}
\end{frame}

\begin{frame}{References}
\protect\hypertarget{references}{}
\hypertarget{refs}{}
\begin{CSLReferences}{1}{0}
\leavevmode\vadjust pre{\hypertarget{ref-ISL}{}}%
James, Gareth, Daniela Witten, Trevor Hastie, and Robert Tibshirani.
2021. \emph{An Introduction to Statistical Learning}. Springer.

\end{CSLReferences}
\end{frame}

\end{document}
