\documentclass[]{article}
\usepackage{lmodern}
\usepackage{amssymb,amsmath}
\usepackage{ifxetex,ifluatex}
\usepackage{fixltx2e} % provides \textsubscript
\ifnum 0\ifxetex 1\fi\ifluatex 1\fi=0 % if pdftex
  \usepackage[T1]{fontenc}
  \usepackage[utf8]{inputenc}
\else % if luatex or xelatex
  \ifxetex
    \usepackage{mathspec}
  \else
    \usepackage{fontspec}
  \fi
  \defaultfontfeatures{Ligatures=TeX,Scale=MatchLowercase}
\fi
% use upquote if available, for straight quotes in verbatim environments
\IfFileExists{upquote.sty}{\usepackage{upquote}}{}
% use microtype if available
\IfFileExists{microtype.sty}{%
\usepackage{microtype}
\UseMicrotypeSet[protrusion]{basicmath} % disable protrusion for tt fonts
}{}
\usepackage[margin=1in]{geometry}
\usepackage{hyperref}
\PassOptionsToPackage{usenames,dvipsnames}{color} % color is loaded by hyperref
\hypersetup{unicode=true,
            pdftitle={Module 2: Recommended Exercises},
            pdfauthor={Martina Hall, Michail Spitieris, Stefanie Muff, Department of Mathematical Sciences, NTNU},
            colorlinks=true,
            linkcolor=Maroon,
            citecolor=Blue,
            urlcolor=blue,
            breaklinks=true}
\urlstyle{same}  % don't use monospace font for urls
\usepackage{color}
\usepackage{fancyvrb}
\newcommand{\VerbBar}{|}
\newcommand{\VERB}{\Verb[commandchars=\\\{\}]}
\DefineVerbatimEnvironment{Highlighting}{Verbatim}{commandchars=\\\{\}}
% Add ',fontsize=\small' for more characters per line
\usepackage{framed}
\definecolor{shadecolor}{RGB}{248,248,248}
\newenvironment{Shaded}{\begin{snugshade}}{\end{snugshade}}
\newcommand{\KeywordTok}[1]{\textcolor[rgb]{0.13,0.29,0.53}{\textbf{#1}}}
\newcommand{\DataTypeTok}[1]{\textcolor[rgb]{0.13,0.29,0.53}{#1}}
\newcommand{\DecValTok}[1]{\textcolor[rgb]{0.00,0.00,0.81}{#1}}
\newcommand{\BaseNTok}[1]{\textcolor[rgb]{0.00,0.00,0.81}{#1}}
\newcommand{\FloatTok}[1]{\textcolor[rgb]{0.00,0.00,0.81}{#1}}
\newcommand{\ConstantTok}[1]{\textcolor[rgb]{0.00,0.00,0.00}{#1}}
\newcommand{\CharTok}[1]{\textcolor[rgb]{0.31,0.60,0.02}{#1}}
\newcommand{\SpecialCharTok}[1]{\textcolor[rgb]{0.00,0.00,0.00}{#1}}
\newcommand{\StringTok}[1]{\textcolor[rgb]{0.31,0.60,0.02}{#1}}
\newcommand{\VerbatimStringTok}[1]{\textcolor[rgb]{0.31,0.60,0.02}{#1}}
\newcommand{\SpecialStringTok}[1]{\textcolor[rgb]{0.31,0.60,0.02}{#1}}
\newcommand{\ImportTok}[1]{#1}
\newcommand{\CommentTok}[1]{\textcolor[rgb]{0.56,0.35,0.01}{\textit{#1}}}
\newcommand{\DocumentationTok}[1]{\textcolor[rgb]{0.56,0.35,0.01}{\textbf{\textit{#1}}}}
\newcommand{\AnnotationTok}[1]{\textcolor[rgb]{0.56,0.35,0.01}{\textbf{\textit{#1}}}}
\newcommand{\CommentVarTok}[1]{\textcolor[rgb]{0.56,0.35,0.01}{\textbf{\textit{#1}}}}
\newcommand{\OtherTok}[1]{\textcolor[rgb]{0.56,0.35,0.01}{#1}}
\newcommand{\FunctionTok}[1]{\textcolor[rgb]{0.00,0.00,0.00}{#1}}
\newcommand{\VariableTok}[1]{\textcolor[rgb]{0.00,0.00,0.00}{#1}}
\newcommand{\ControlFlowTok}[1]{\textcolor[rgb]{0.13,0.29,0.53}{\textbf{#1}}}
\newcommand{\OperatorTok}[1]{\textcolor[rgb]{0.81,0.36,0.00}{\textbf{#1}}}
\newcommand{\BuiltInTok}[1]{#1}
\newcommand{\ExtensionTok}[1]{#1}
\newcommand{\PreprocessorTok}[1]{\textcolor[rgb]{0.56,0.35,0.01}{\textit{#1}}}
\newcommand{\AttributeTok}[1]{\textcolor[rgb]{0.77,0.63,0.00}{#1}}
\newcommand{\RegionMarkerTok}[1]{#1}
\newcommand{\InformationTok}[1]{\textcolor[rgb]{0.56,0.35,0.01}{\textbf{\textit{#1}}}}
\newcommand{\WarningTok}[1]{\textcolor[rgb]{0.56,0.35,0.01}{\textbf{\textit{#1}}}}
\newcommand{\AlertTok}[1]{\textcolor[rgb]{0.94,0.16,0.16}{#1}}
\newcommand{\ErrorTok}[1]{\textcolor[rgb]{0.64,0.00,0.00}{\textbf{#1}}}
\newcommand{\NormalTok}[1]{#1}
\usepackage{graphicx,grffile}
\makeatletter
\def\maxwidth{\ifdim\Gin@nat@width>\linewidth\linewidth\else\Gin@nat@width\fi}
\def\maxheight{\ifdim\Gin@nat@height>\textheight\textheight\else\Gin@nat@height\fi}
\makeatother
% Scale images if necessary, so that they will not overflow the page
% margins by default, and it is still possible to overwrite the defaults
% using explicit options in \includegraphics[width, height, ...]{}
\setkeys{Gin}{width=\maxwidth,height=\maxheight,keepaspectratio}
\IfFileExists{parskip.sty}{%
\usepackage{parskip}
}{% else
\setlength{\parindent}{0pt}
\setlength{\parskip}{6pt plus 2pt minus 1pt}
}
\setlength{\emergencystretch}{3em}  % prevent overfull lines
\providecommand{\tightlist}{%
  \setlength{\itemsep}{0pt}\setlength{\parskip}{0pt}}
\setcounter{secnumdepth}{0}
% Redefines (sub)paragraphs to behave more like sections
\ifx\paragraph\undefined\else
\let\oldparagraph\paragraph
\renewcommand{\paragraph}[1]{\oldparagraph{#1}\mbox{}}
\fi
\ifx\subparagraph\undefined\else
\let\oldsubparagraph\subparagraph
\renewcommand{\subparagraph}[1]{\oldsubparagraph{#1}\mbox{}}
\fi

%%% Use protect on footnotes to avoid problems with footnotes in titles
\let\rmarkdownfootnote\footnote%
\def\footnote{\protect\rmarkdownfootnote}

%%% Change title format to be more compact
\usepackage{titling}

% Create subtitle command for use in maketitle
\providecommand{\subtitle}[1]{
  \posttitle{
    \begin{center}\large#1\end{center}
    }
}

\setlength{\droptitle}{-2em}

  \title{Module 2: Recommended Exercises}
    \pretitle{\vspace{\droptitle}\centering\huge}
  \posttitle{\par}
  \subtitle{TMA4268 Statistical Learning V2020}
  \author{Martina Hall, Michail Spitieris, Stefanie Muff, Department of
Mathematical Sciences, NTNU}
    \preauthor{\centering\large\emph}
  \postauthor{\par}
      \predate{\centering\large\emph}
  \postdate{\par}
    \date{January 17, 2020}


\begin{document}
\maketitle

\section{Theoretical exercises}\label{theoretical-exercises}

\subsection{Problem 1}\label{problem-1}

Describe a real-life application in which classification might be
useful. Identify the response and the predictors. Is the goal inference
or prediction?

\subsection{Problem 2}\label{problem-2}

Describe a real-life application in which regression might be useful.
Identify the response and the predictors. Is the goal inference or
prediction?

\subsection{Problem 3}\label{problem-3}

Take a look at Figure 2.9 in the course book (p.31).

\begin{enumerate}
\def\labelenumi{\alph{enumi})}
\tightlist
\item
  Will a flexible or rigid method typically have the highest test error?
\item
  Does a small variance imply an overfit or rather an underfit to the
  data?
\item
  Relate the problem of over-and underfitting to the bias-variance
  trade-off.
\end{enumerate}

\subsection{Problem 4 -- Exercise 2.4.9 from ISL textbook
(modified)}\label{problem-4-exercise-2.4.9-from-isl-textbook-modified}

This exercise involves the \texttt{Auto} dataset from the \texttt{ISLR}
library. Load the data into your R session by running the following
commands

\begin{Shaded}
\begin{Highlighting}[]
\KeywordTok{library}\NormalTok{(ISLR)}
\KeywordTok{data}\NormalTok{(Auto)}
\end{Highlighting}
\end{Shaded}

PS: if the \texttt{ISLR} package is not installed (library function
gives error) you can install it by running
\texttt{install.packages("ISLR")} before you load the package the first
time.

\begin{enumerate}
\def\labelenumi{\alph{enumi})}
\item
  View the data. What are the dimensions of the data? Which predictors
  are quantitative and which are qualitative?
\item
  What is the range (min, max) of each quantitative predictor? Hint: use
  the \texttt{range()} function. For more advanced users, check out
  \texttt{sapply()}.
\item
  What is the mean and standard deviation of each quantitative
  predictor?
\item
  Now, make a new dataset called \texttt{ReducedAuto} where you remove
  the 10th through 85th observations. What is the range, mean and
  standard deviation of the quantitative predictors in this reduced set?
\item
  Using the full dataset, investigate the quantitative predictors
  graphically using a scatterplot. Do you see any strong relationships
  between the predictors? Hint: try out the \texttt{ggpairs()} function
  from the \texttt{GGally} package.
\item
  Suppose we wish to predict gas milage (\texttt{mpg}) on the basis of
  the other variables (both quantitative and qualitative). Make some
  plots showing the relationships between \texttt{mpg} and the
  qualitative predictors (hint: \texttt{geom\_boxplot()}). Which
  predictors would you consider helpful when predicting \texttt{mpg}?
\item
  The correlation of two variables \(X\) and \(Y\) are defined as
  \[ \text{cor}(X,Y) = \frac{\text{cov}(X,Y)}{\sigma_X\sigma_Y}.\] Both
  the correlation matrix and covariance matrix are easily assessed in R
  with the \texttt{cor()} and \texttt{cov()} functions. Use only the
  covariance matrix to find the correlation between \texttt{mpg} and
  \texttt{displacement}, \texttt{mpg} and \texttt{horsepower}, and
  \texttt{mpg} and \texttt{weight}. Do your results coincide with the
  correlation matrix you find using \texttt{cor(Auto{[},quant{]})}?
\end{enumerate}

\begin{Shaded}
\begin{Highlighting}[]
\NormalTok{quant =}\StringTok{ }\KeywordTok{c}\NormalTok{(}\DecValTok{1}\NormalTok{, }\DecValTok{3}\NormalTok{, }\DecValTok{4}\NormalTok{, }\DecValTok{5}\NormalTok{, }\DecValTok{6}\NormalTok{, }\DecValTok{7}\NormalTok{)}
\NormalTok{covMat =}\StringTok{ }\KeywordTok{cov}\NormalTok{(Auto[, quant])}
\end{Highlighting}
\end{Shaded}

\section{Problem 5 -- Multivariate normal
distribution}\label{problem-5-multivariate-normal-distribution}

The pdf of a multivariate normal distribution is on the form
\[ f(\boldsymbol{x}) = \frac{1}{(2\pi)^{p/2}|\boldsymbol{\Sigma|}} \exp\{-\frac{1}{2}(\boldsymbol{x-\mu})^T\boldsymbol{\Sigma}^{-1}(\boldsymbol{x-\mu)}\},\]
where \(\bf{x}\) is a random vector of size \(p\times 1\),
\(\boldsymbol{\mu}\) is the mean vector of size \(p\times 1\) and
\(\boldsymbol{\Sigma}\) is the covariance matrix of size \(p\times p\).

\begin{enumerate}
\def\labelenumi{\alph{enumi})}
\tightlist
\item
  Use the \texttt{mvrnorm()} function from the \texttt{MASS} library to
  simulate 1000 values from multivariate normal distributions with
\item
  \[ \boldsymbol{\mu} = \begin{pmatrix}
  2 \\
  3 
  \end{pmatrix} \quad \text{and} \quad \boldsymbol{\Sigma} = \begin{pmatrix}
  1 & 0\\
  0 & 1
  \end{pmatrix},\]
\end{enumerate}

\begin{enumerate}
\def\labelenumi{\roman{enumi})}
\setcounter{enumi}{1}
\item
  \[ \boldsymbol{\mu} = \begin{pmatrix}
  2 \\
  3 
  \end{pmatrix} \quad \text{and} \quad \boldsymbol{\Sigma} = \begin{pmatrix}
  1 & 0\\
  0 & 5
  \end{pmatrix},\]
\item
  \[ \boldsymbol{\mu} = \begin{pmatrix}
  2 \\
  3 
  \end{pmatrix} \quad \text{and} \quad \boldsymbol{\Sigma} = \begin{pmatrix}
  1 & 2\\
  2 & 5
  \end{pmatrix},\]
\item
  \[ \boldsymbol{\mu} = \begin{pmatrix}
  2 \\
  3 
  \end{pmatrix} \quad \text{and} \quad \boldsymbol{\Sigma} = \begin{pmatrix}
  1 & -2\\
  -2 & 1
  \end{pmatrix}.\]
\end{enumerate}

\begin{enumerate}
\def\labelenumi{\alph{enumi})}
\setcounter{enumi}{1}
\tightlist
\item
  Plot the four sets of simulated values. Can you see which plot belongs
  to which distribution?
\end{enumerate}

\normalsize
\clearpage

\subsection{Problem 6: Theory and practice - training and test MSE;
bias-variance}\label{problem-6-theory-and-practice---training-and-test-mse-bias-variance}

We will now look closely into the simulations and calculations performed
for the training error (MSEtrain), test error (MSEtest), and the
bias-variance trade-off in lecture 1 of module 2.

Simulation setup:

\begin{itemize}
\tightlist
\item
  True function \(f(x)=x^2\) with normal noise
  \(\varepsilon \sim N(0,2^2)\).
\item
  \(x= -2.0, -1.9, ... ,4.0\) (grid with 61 values).
\item
  Parametric models are fitted (polynomials of degree 1 to degree 20).
\item
  M=100 simulations.
\end{itemize}

\subsubsection{a) Problem set-up}\label{a-problem-set-up}

\begin{itemize}
\tightlist
\item
  Look at the code below and run it yourself. Explain what is done (you
  do not need not understand the code in detail).
\item
  We will learn more about the \texttt{lm} function in M3 - now just
  think of this as fitting a polynomial regression and predict gives the
  fitted curve in our grid points. \texttt{predarray} is just a way to
  save \(M\) simulations of 61 gridpoints in \(x\) and 20 polynomial
  models.
\end{itemize}

\begin{Shaded}
\begin{Highlighting}[]
\KeywordTok{library}\NormalTok{(ggplot2)}
\KeywordTok{library}\NormalTok{(ggpubr)}
\KeywordTok{set.seed}\NormalTok{(}\DecValTok{2}\NormalTok{)  }\CommentTok{# to reproduce}
\NormalTok{M =}\StringTok{ }\DecValTok{100}  \CommentTok{# repeated samplings, x fixed }
\NormalTok{nord =}\StringTok{ }\DecValTok{20}  \CommentTok{# order of polynoms}
\NormalTok{x =}\StringTok{ }\KeywordTok{seq}\NormalTok{(}\DataTypeTok{from =} \DecValTok{-2}\NormalTok{, }\DataTypeTok{to =} \DecValTok{4}\NormalTok{, }\DataTypeTok{by =} \FloatTok{0.1}\NormalTok{)}
\NormalTok{truefunc =}\StringTok{ }\ControlFlowTok{function}\NormalTok{(x) \{}
    \KeywordTok{return}\NormalTok{(x}\OperatorTok{^}\DecValTok{2}\NormalTok{)}
\NormalTok{\}}
\NormalTok{true_y =}\StringTok{ }\KeywordTok{truefunc}\NormalTok{(x)}
\NormalTok{error =}\StringTok{ }\KeywordTok{matrix}\NormalTok{(}\KeywordTok{rnorm}\NormalTok{(}\KeywordTok{length}\NormalTok{(x) }\OperatorTok{*}\StringTok{ }\NormalTok{M, }\DataTypeTok{mean =} \DecValTok{0}\NormalTok{, }\DataTypeTok{sd =} \DecValTok{2}\NormalTok{), }\DataTypeTok{nrow =}\NormalTok{ M, }\DataTypeTok{byrow =} \OtherTok{TRUE}\NormalTok{)}
\NormalTok{ymat =}\StringTok{ }\KeywordTok{matrix}\NormalTok{(}\KeywordTok{rep}\NormalTok{(true_y, M), }\DataTypeTok{byrow =}\NormalTok{ T, }\DataTypeTok{nrow =}\NormalTok{ M) }\OperatorTok{+}\StringTok{ }\NormalTok{error}
\NormalTok{predarray =}\StringTok{ }\KeywordTok{array}\NormalTok{(}\OtherTok{NA}\NormalTok{, }\DataTypeTok{dim =} \KeywordTok{c}\NormalTok{(M, }\KeywordTok{length}\NormalTok{(x), nord))}
\ControlFlowTok{for}\NormalTok{ (i }\ControlFlowTok{in} \DecValTok{1}\OperatorTok{:}\NormalTok{M) \{}
    \ControlFlowTok{for}\NormalTok{ (j }\ControlFlowTok{in} \DecValTok{1}\OperatorTok{:}\NormalTok{nord) \{}
\NormalTok{        predarray[i, , j] =}\StringTok{ }\KeywordTok{predict}\NormalTok{(}\KeywordTok{lm}\NormalTok{(ymat[i, ] }\OperatorTok{~}\StringTok{ }\KeywordTok{poly}\NormalTok{(x, j, }\DataTypeTok{raw =} \OtherTok{TRUE}\NormalTok{)))}
\NormalTok{    \}}
\NormalTok{\}}
\CommentTok{# M matrices of size length(x) times nord first, only look at}
\CommentTok{# variablity in the M fits and plot M curves where we had 1 for}
\CommentTok{# plotting need to stack the matrices underneath eachother and make}
\CommentTok{# new variable 'rep'}
\NormalTok{stackmat =}\StringTok{ }\OtherTok{NULL}
\ControlFlowTok{for}\NormalTok{ (i }\ControlFlowTok{in} \DecValTok{1}\OperatorTok{:}\NormalTok{M) \{}
\NormalTok{    stackmat =}\StringTok{ }\KeywordTok{rbind}\NormalTok{(stackmat, }\KeywordTok{cbind}\NormalTok{(x, }\KeywordTok{rep}\NormalTok{(i, }\KeywordTok{length}\NormalTok{(x)), predarray[i, }
\NormalTok{        , ]))}
\NormalTok{\}}
\CommentTok{# dim(stackmat)}
\KeywordTok{colnames}\NormalTok{(stackmat) =}\StringTok{ }\KeywordTok{c}\NormalTok{(}\StringTok{"x"}\NormalTok{, }\StringTok{"rep"}\NormalTok{, }\KeywordTok{paste}\NormalTok{(}\StringTok{"poly"}\NormalTok{, }\DecValTok{1}\OperatorTok{:}\DecValTok{20}\NormalTok{, }\DataTypeTok{sep =} \StringTok{""}\NormalTok{))}
\NormalTok{sdf =}\StringTok{ }\KeywordTok{as.data.frame}\NormalTok{(stackmat)  }\CommentTok{#NB have poly1-20 now - but first only use 1,2,20}
\CommentTok{# to add true curve using stat_function - easiest solution}
\NormalTok{true_x =}\StringTok{ }\NormalTok{x}
\NormalTok{yrange =}\StringTok{ }\KeywordTok{range}\NormalTok{(}\KeywordTok{apply}\NormalTok{(sdf, }\DecValTok{2}\NormalTok{, range)[, }\DecValTok{3}\OperatorTok{:}\DecValTok{22}\NormalTok{])}
\NormalTok{p1 =}\StringTok{ }\KeywordTok{ggplot}\NormalTok{(}\DataTypeTok{data =}\NormalTok{ sdf, }\KeywordTok{aes}\NormalTok{(}\DataTypeTok{x =}\NormalTok{ x, }\DataTypeTok{y =}\NormalTok{ poly1, }\DataTypeTok{group =}\NormalTok{ rep, }\DataTypeTok{colour =}\NormalTok{ rep)) }\OperatorTok{+}\StringTok{ }
\StringTok{    }\KeywordTok{scale_y_continuous}\NormalTok{(}\DataTypeTok{limits =}\NormalTok{ yrange) }\OperatorTok{+}\StringTok{ }\KeywordTok{geom_line}\NormalTok{()}
\NormalTok{p1 =}\StringTok{ }\NormalTok{p1 }\OperatorTok{+}\StringTok{ }\KeywordTok{stat_function}\NormalTok{(}\DataTypeTok{fun =}\NormalTok{ truefunc, }\DataTypeTok{lwd =} \FloatTok{1.3}\NormalTok{, }\DataTypeTok{colour =} \StringTok{"black"}\NormalTok{) }\OperatorTok{+}\StringTok{ }
\StringTok{    }\KeywordTok{ggtitle}\NormalTok{(}\StringTok{"poly1"}\NormalTok{)}
\NormalTok{p2 =}\StringTok{ }\KeywordTok{ggplot}\NormalTok{(}\DataTypeTok{data =}\NormalTok{ sdf, }\KeywordTok{aes}\NormalTok{(}\DataTypeTok{x =}\NormalTok{ x, }\DataTypeTok{y =}\NormalTok{ poly2, }\DataTypeTok{group =}\NormalTok{ rep, }\DataTypeTok{colour =}\NormalTok{ rep)) }\OperatorTok{+}\StringTok{ }
\StringTok{    }\KeywordTok{scale_y_continuous}\NormalTok{(}\DataTypeTok{limits =}\NormalTok{ yrange) }\OperatorTok{+}\StringTok{ }\KeywordTok{geom_line}\NormalTok{()}
\NormalTok{p2 =}\StringTok{ }\NormalTok{p2 }\OperatorTok{+}\StringTok{ }\KeywordTok{stat_function}\NormalTok{(}\DataTypeTok{fun =}\NormalTok{ truefunc, }\DataTypeTok{lwd =} \FloatTok{1.3}\NormalTok{, }\DataTypeTok{colour =} \StringTok{"black"}\NormalTok{) }\OperatorTok{+}\StringTok{ }
\StringTok{    }\KeywordTok{ggtitle}\NormalTok{(}\StringTok{"poly2"}\NormalTok{)}
\NormalTok{p10 =}\StringTok{ }\KeywordTok{ggplot}\NormalTok{(}\DataTypeTok{data =}\NormalTok{ sdf, }\KeywordTok{aes}\NormalTok{(}\DataTypeTok{x =}\NormalTok{ x, }\DataTypeTok{y =}\NormalTok{ poly10, }\DataTypeTok{group =}\NormalTok{ rep, }\DataTypeTok{colour =}\NormalTok{ rep)) }\OperatorTok{+}\StringTok{ }
\StringTok{    }\KeywordTok{scale_y_continuous}\NormalTok{(}\DataTypeTok{limits =}\NormalTok{ yrange) }\OperatorTok{+}\StringTok{ }\KeywordTok{geom_line}\NormalTok{()}
\NormalTok{p10 =}\StringTok{ }\NormalTok{p10 }\OperatorTok{+}\StringTok{ }\KeywordTok{stat_function}\NormalTok{(}\DataTypeTok{fun =}\NormalTok{ truefunc, }\DataTypeTok{lwd =} \FloatTok{1.3}\NormalTok{, }\DataTypeTok{colour =} \StringTok{"black"}\NormalTok{) }\OperatorTok{+}\StringTok{ }
\StringTok{    }\KeywordTok{ggtitle}\NormalTok{(}\StringTok{"poly10"}\NormalTok{)}
\NormalTok{p20 =}\StringTok{ }\KeywordTok{ggplot}\NormalTok{(}\DataTypeTok{data =}\NormalTok{ sdf, }\KeywordTok{aes}\NormalTok{(}\DataTypeTok{x =}\NormalTok{ x, }\DataTypeTok{y =}\NormalTok{ poly20, }\DataTypeTok{group =}\NormalTok{ rep, }\DataTypeTok{colour =}\NormalTok{ rep)) }\OperatorTok{+}\StringTok{ }
\StringTok{    }\KeywordTok{scale_y_continuous}\NormalTok{(}\DataTypeTok{limits =}\NormalTok{ yrange) }\OperatorTok{+}\StringTok{ }\KeywordTok{geom_line}\NormalTok{()}
\NormalTok{p20 =}\StringTok{ }\NormalTok{p20 }\OperatorTok{+}\StringTok{ }\KeywordTok{stat_function}\NormalTok{(}\DataTypeTok{fun =}\NormalTok{ truefunc, }\DataTypeTok{lwd =} \FloatTok{1.3}\NormalTok{, }\DataTypeTok{colour =} \StringTok{"black"}\NormalTok{) }\OperatorTok{+}\StringTok{ }
\StringTok{    }\KeywordTok{ggtitle}\NormalTok{(}\StringTok{"poly20"}\NormalTok{)}
\KeywordTok{ggarrange}\NormalTok{(p1, p2, p10, p20)}
\end{Highlighting}
\end{Shaded}

\begin{itemize}
\tightlist
\item
  What do you observe in this plot? Which polynomial fits the best to
  the true curve?
\end{itemize}

\begin{center}\rule{0.5\linewidth}{\linethickness}\end{center}

\subsubsection{b) Train and test MSE}\label{b-train-and-test-mse}

\begin{itemize}
\tightlist
\item
  First we produce predictions at each grid point based on our training
  data (\texttt{x} and \texttt{ymat})
\item
  but we also draw new observations to calculate testMSE - see
  \texttt{testymat}
\item
  observe how trainMSE and testMSE is calculated
\item
  run the code
\end{itemize}

\begin{Shaded}
\begin{Highlighting}[]
\KeywordTok{set.seed}\NormalTok{(}\DecValTok{2}\NormalTok{)  }\CommentTok{# to reproduce}
\NormalTok{M =}\StringTok{ }\DecValTok{100}  \CommentTok{# repeated samplings,x fixed but new errors}
\NormalTok{nord =}\StringTok{ }\DecValTok{20}
\NormalTok{x =}\StringTok{ }\KeywordTok{seq}\NormalTok{(}\DataTypeTok{from =} \DecValTok{-2}\NormalTok{, }\DataTypeTok{to =} \DecValTok{4}\NormalTok{, }\DataTypeTok{by =} \FloatTok{0.1}\NormalTok{)}
\NormalTok{truefunc =}\StringTok{ }\ControlFlowTok{function}\NormalTok{(x) \{}
    \KeywordTok{return}\NormalTok{(x}\OperatorTok{^}\DecValTok{2}\NormalTok{)}
\NormalTok{\}}
\NormalTok{true_y =}\StringTok{ }\KeywordTok{truefunc}\NormalTok{(x)}
\NormalTok{error =}\StringTok{ }\KeywordTok{matrix}\NormalTok{(}\KeywordTok{rnorm}\NormalTok{(}\KeywordTok{length}\NormalTok{(x) }\OperatorTok{*}\StringTok{ }\NormalTok{M, }\DataTypeTok{mean =} \DecValTok{0}\NormalTok{, }\DataTypeTok{sd =} \DecValTok{2}\NormalTok{), }\DataTypeTok{nrow =}\NormalTok{ M, }\DataTypeTok{byrow =} \OtherTok{TRUE}\NormalTok{)}
\NormalTok{testerror =}\StringTok{ }\KeywordTok{matrix}\NormalTok{(}\KeywordTok{rnorm}\NormalTok{(}\KeywordTok{length}\NormalTok{(x) }\OperatorTok{*}\StringTok{ }\NormalTok{M, }\DataTypeTok{mean =} \DecValTok{0}\NormalTok{, }\DataTypeTok{sd =} \DecValTok{2}\NormalTok{), }\DataTypeTok{nrow =}\NormalTok{ M, }
    \DataTypeTok{byrow =} \OtherTok{TRUE}\NormalTok{)}
\NormalTok{ymat =}\StringTok{ }\KeywordTok{matrix}\NormalTok{(}\KeywordTok{rep}\NormalTok{(true_y, M), }\DataTypeTok{byrow =}\NormalTok{ T, }\DataTypeTok{nrow =}\NormalTok{ M) }\OperatorTok{+}\StringTok{ }\NormalTok{error}
\NormalTok{testymat =}\StringTok{ }\KeywordTok{matrix}\NormalTok{(}\KeywordTok{rep}\NormalTok{(true_y, M), }\DataTypeTok{byrow =}\NormalTok{ T, }\DataTypeTok{nrow =}\NormalTok{ M) }\OperatorTok{+}\StringTok{ }\NormalTok{testerror}
\NormalTok{predarray =}\StringTok{ }\KeywordTok{array}\NormalTok{(}\OtherTok{NA}\NormalTok{, }\DataTypeTok{dim =} \KeywordTok{c}\NormalTok{(M, }\KeywordTok{length}\NormalTok{(x), nord))}
\ControlFlowTok{for}\NormalTok{ (i }\ControlFlowTok{in} \DecValTok{1}\OperatorTok{:}\NormalTok{M) \{}
    \ControlFlowTok{for}\NormalTok{ (j }\ControlFlowTok{in} \DecValTok{1}\OperatorTok{:}\NormalTok{nord) \{}
\NormalTok{        predarray[i, , j] =}\StringTok{ }\KeywordTok{predict}\NormalTok{(}\KeywordTok{lm}\NormalTok{(ymat[i, ] }\OperatorTok{~}\StringTok{ }\KeywordTok{poly}\NormalTok{(x, j, }\DataTypeTok{raw =} \OtherTok{TRUE}\NormalTok{)))}
\NormalTok{    \}}
\NormalTok{\}}
\NormalTok{trainMSE =}\StringTok{ }\KeywordTok{matrix}\NormalTok{(}\DataTypeTok{ncol =}\NormalTok{ nord, }\DataTypeTok{nrow =}\NormalTok{ M)}
\NormalTok{testMSE =}\StringTok{ }\KeywordTok{matrix}\NormalTok{(}\DataTypeTok{ncol =}\NormalTok{ nord, }\DataTypeTok{nrow =}\NormalTok{ M)}
\ControlFlowTok{for}\NormalTok{ (i }\ControlFlowTok{in} \DecValTok{1}\OperatorTok{:}\NormalTok{M) \{}
\NormalTok{    trainMSE[i, ] =}\StringTok{ }\KeywordTok{apply}\NormalTok{((predarray[i, , ] }\OperatorTok{-}\StringTok{ }\NormalTok{ymat[i, ])}\OperatorTok{^}\DecValTok{2}\NormalTok{, }\DecValTok{2}\NormalTok{, mean)}
\NormalTok{    testMSE[i, ] =}\StringTok{ }\KeywordTok{apply}\NormalTok{((predarray[i, , ] }\OperatorTok{-}\StringTok{ }\NormalTok{testymat[i, ])}\OperatorTok{^}\DecValTok{2}\NormalTok{, }\DecValTok{2}\NormalTok{, mean)}
\NormalTok{\}}
\end{Highlighting}
\end{Shaded}

\begin{itemize}
\tightlist
\item
  Then we plot train and testMSE - first for one train + test data set,
  then for 99 more.
\end{itemize}

\begin{Shaded}
\begin{Highlighting}[]
\KeywordTok{library}\NormalTok{(ggplot2)}
\KeywordTok{library}\NormalTok{(ggpubr)}
\CommentTok{# format suitable for plotting}
\NormalTok{stackmat =}\StringTok{ }\OtherTok{NULL}
\ControlFlowTok{for}\NormalTok{ (i }\ControlFlowTok{in} \DecValTok{1}\OperatorTok{:}\NormalTok{M) \{}
\NormalTok{    stackmat =}\StringTok{ }\KeywordTok{rbind}\NormalTok{(stackmat, }\KeywordTok{cbind}\NormalTok{(}\KeywordTok{rep}\NormalTok{(i, nord), }\DecValTok{1}\OperatorTok{:}\NormalTok{nord, trainMSE[i, }
\NormalTok{        ], testMSE[i, ]))}
\NormalTok{\}}
\KeywordTok{colnames}\NormalTok{(stackmat) =}\StringTok{ }\KeywordTok{c}\NormalTok{(}\StringTok{"rep"}\NormalTok{, }\StringTok{"poly"}\NormalTok{, }\StringTok{"trainMSE"}\NormalTok{, }\StringTok{"testMSE"}\NormalTok{)}
\NormalTok{sdf =}\StringTok{ }\KeywordTok{as.data.frame}\NormalTok{(stackmat)}
\NormalTok{yrange =}\StringTok{ }\KeywordTok{range}\NormalTok{(sdf[, }\DecValTok{3}\OperatorTok{:}\DecValTok{4}\NormalTok{])}
\NormalTok{p1 =}\StringTok{ }\KeywordTok{ggplot}\NormalTok{(}\DataTypeTok{data =}\NormalTok{ sdf[}\DecValTok{1}\OperatorTok{:}\NormalTok{nord, ], }\KeywordTok{aes}\NormalTok{(}\DataTypeTok{x =}\NormalTok{ poly, }\DataTypeTok{y =}\NormalTok{ trainMSE)) }\OperatorTok{+}\StringTok{ }\KeywordTok{scale_y_continuous}\NormalTok{(}\DataTypeTok{limits =}\NormalTok{ yrange) }\OperatorTok{+}\StringTok{ }
\StringTok{    }\KeywordTok{geom_line}\NormalTok{()}
\NormalTok{pall =}\StringTok{ }\KeywordTok{ggplot}\NormalTok{(}\DataTypeTok{data =}\NormalTok{ sdf, }\KeywordTok{aes}\NormalTok{(}\DataTypeTok{x =}\NormalTok{ poly, }\DataTypeTok{group =}\NormalTok{ rep, }\DataTypeTok{y =}\NormalTok{ trainMSE, }\DataTypeTok{colour =}\NormalTok{ rep)) }\OperatorTok{+}\StringTok{ }
\StringTok{    }\KeywordTok{scale_y_continuous}\NormalTok{(}\DataTypeTok{limits =}\NormalTok{ yrange) }\OperatorTok{+}\StringTok{ }\KeywordTok{geom_line}\NormalTok{()}
\NormalTok{testp1 =}\StringTok{ }\KeywordTok{ggplot}\NormalTok{(}\DataTypeTok{data =}\NormalTok{ sdf[}\DecValTok{1}\OperatorTok{:}\NormalTok{nord, ], }\KeywordTok{aes}\NormalTok{(}\DataTypeTok{x =}\NormalTok{ poly, }\DataTypeTok{y =}\NormalTok{ testMSE)) }\OperatorTok{+}\StringTok{ }\KeywordTok{scale_y_continuous}\NormalTok{(}\DataTypeTok{limits =}\NormalTok{ yrange) }\OperatorTok{+}\StringTok{ }
\StringTok{    }\KeywordTok{geom_line}\NormalTok{()}
\NormalTok{testpall =}\StringTok{ }\KeywordTok{ggplot}\NormalTok{(}\DataTypeTok{data =}\NormalTok{ sdf, }\KeywordTok{aes}\NormalTok{(}\DataTypeTok{x =}\NormalTok{ poly, }\DataTypeTok{group =}\NormalTok{ rep, }\DataTypeTok{y =}\NormalTok{ testMSE, }
    \DataTypeTok{colour =}\NormalTok{ rep)) }\OperatorTok{+}\StringTok{ }\KeywordTok{scale_y_continuous}\NormalTok{(}\DataTypeTok{limits =}\NormalTok{ yrange) }\OperatorTok{+}\StringTok{ }\KeywordTok{geom_line}\NormalTok{()}
\KeywordTok{ggarrange}\NormalTok{(p1, pall, testp1, testpall)}
\end{Highlighting}
\end{Shaded}

\begin{itemize}
\tightlist
\item
  More plots: first boxplot and then mean for train and test MSE
\end{itemize}

\begin{Shaded}
\begin{Highlighting}[]
\KeywordTok{library}\NormalTok{(reshape2)}
\NormalTok{df =}\StringTok{ }\KeywordTok{melt}\NormalTok{(sdf, }\DataTypeTok{id =} \KeywordTok{c}\NormalTok{(}\StringTok{"poly"}\NormalTok{, }\StringTok{"rep"}\NormalTok{))[, }\DecValTok{-2}\NormalTok{]}
\KeywordTok{colnames}\NormalTok{(df)[}\DecValTok{2}\NormalTok{] =}\StringTok{ "MSEtype"}
\KeywordTok{ggplot}\NormalTok{(}\DataTypeTok{data =}\NormalTok{ df, }\KeywordTok{aes}\NormalTok{(}\DataTypeTok{x =} \KeywordTok{as.factor}\NormalTok{(poly), }\DataTypeTok{y =}\NormalTok{ value)) }\OperatorTok{+}\StringTok{ }\KeywordTok{geom_boxplot}\NormalTok{(}\KeywordTok{aes}\NormalTok{(}\DataTypeTok{fill =}\NormalTok{ MSEtype))}
\end{Highlighting}
\end{Shaded}

\begin{Shaded}
\begin{Highlighting}[]
\NormalTok{trainMSEmean =}\StringTok{ }\KeywordTok{apply}\NormalTok{(trainMSE, }\DecValTok{2}\NormalTok{, mean)}
\NormalTok{testMSEmean =}\StringTok{ }\KeywordTok{apply}\NormalTok{(testMSE, }\DecValTok{2}\NormalTok{, mean)}
\NormalTok{meandf =}\StringTok{ }\KeywordTok{melt}\NormalTok{(}\KeywordTok{data.frame}\NormalTok{(}\KeywordTok{cbind}\NormalTok{(}\DataTypeTok{poly =} \DecValTok{1}\OperatorTok{:}\NormalTok{nord, trainMSEmean, testMSEmean)), }
    \DataTypeTok{id =} \StringTok{"poly"}\NormalTok{)}
\KeywordTok{ggplot}\NormalTok{(}\DataTypeTok{data =}\NormalTok{ meandf, }\KeywordTok{aes}\NormalTok{(}\DataTypeTok{x =}\NormalTok{ poly, }\DataTypeTok{y =}\NormalTok{ value, }\DataTypeTok{colour =}\NormalTok{ variable)) }\OperatorTok{+}\StringTok{ }
\StringTok{    }\KeywordTok{geom_line}\NormalTok{()}
\end{Highlighting}
\end{Shaded}

\begin{itemize}
\tightlist
\item
  Which value of the polynomial gives the smallest mean testMSE?
\item
  Which gives the smallest mean trainMSE?
\item
  Which would you use to predict a new value of \(y\)?
\end{itemize}

\subsubsection{c) Bias and variance - we use the
truth!}\label{c-bias-and-variance---we-use-the-truth}

Finally, we want to see how the expected quadratic loss can be
decomposed into

\begin{itemize}
\tightlist
\item
  irreducible error: \(\text{Var}(\varepsilon)=4\)
\item
  squared bias: difference between mean of estimated parametric model
  chosen and the true underlying curve (\texttt{truefunc})
\item
  variance: variance of the estimated parametric model
\end{itemize}

Notice that the test data is not used - only predicted values in each x
grid point.

Study and run the code. Explain the plots produced.

\begin{Shaded}
\begin{Highlighting}[]
\NormalTok{meanmat =}\StringTok{ }\KeywordTok{matrix}\NormalTok{(}\DataTypeTok{ncol =} \KeywordTok{length}\NormalTok{(x), }\DataTypeTok{nrow =}\NormalTok{ nord)}
\NormalTok{varmat =}\StringTok{ }\KeywordTok{matrix}\NormalTok{(}\DataTypeTok{ncol =} \KeywordTok{length}\NormalTok{(x), }\DataTypeTok{nrow =}\NormalTok{ nord)}
\ControlFlowTok{for}\NormalTok{ (j }\ControlFlowTok{in} \DecValTok{1}\OperatorTok{:}\NormalTok{nord) \{}
\NormalTok{    meanmat[j, ] =}\StringTok{ }\KeywordTok{apply}\NormalTok{(predarray[, , j], }\DecValTok{2}\NormalTok{, mean)  }\CommentTok{# we now take the mean over the M simulations - to mimic E and Var at each x value and each poly model}
\NormalTok{    varmat[j, ] =}\StringTok{ }\KeywordTok{apply}\NormalTok{(predarray[, , j], }\DecValTok{2}\NormalTok{, var)}
\NormalTok{\}}
\CommentTok{# nord times length(x)}
\NormalTok{bias2mat =}\StringTok{ }\NormalTok{(meanmat }\OperatorTok{-}\StringTok{ }\KeywordTok{matrix}\NormalTok{(}\KeywordTok{rep}\NormalTok{(true_y, nord), }\DataTypeTok{byrow =} \OtherTok{TRUE}\NormalTok{, }\DataTypeTok{nrow =}\NormalTok{ nord))}\OperatorTok{^}\DecValTok{2}  \CommentTok{#here the truth is finally used!}
\end{Highlighting}
\end{Shaded}

\begin{itemize}
\tightlist
\item
  Plotting the polys as a function of x
\end{itemize}

\begin{Shaded}
\begin{Highlighting}[]
\NormalTok{df =}\StringTok{ }\KeywordTok{data.frame}\NormalTok{(}\KeywordTok{rep}\NormalTok{(x, }\DataTypeTok{each =}\NormalTok{ nord), }\KeywordTok{rep}\NormalTok{(}\DecValTok{1}\OperatorTok{:}\NormalTok{nord, }\KeywordTok{length}\NormalTok{(x)), }\KeywordTok{c}\NormalTok{(bias2mat), }
    \KeywordTok{c}\NormalTok{(varmat), }\KeywordTok{rep}\NormalTok{(}\DecValTok{4}\NormalTok{, }\KeywordTok{prod}\NormalTok{(}\KeywordTok{dim}\NormalTok{(varmat))))  }\CommentTok{#irr is just 1}
\KeywordTok{colnames}\NormalTok{(df) =}\StringTok{ }\KeywordTok{c}\NormalTok{(}\StringTok{"x"}\NormalTok{, }\StringTok{"poly"}\NormalTok{, }\StringTok{"bias2"}\NormalTok{, }\StringTok{"variance"}\NormalTok{, }\StringTok{"irreducible error"}\NormalTok{)  }\CommentTok{#suitable for plotting}
\NormalTok{df}\OperatorTok{$}\NormalTok{total =}\StringTok{ }\NormalTok{df}\OperatorTok{$}\NormalTok{bias2 }\OperatorTok{+}\StringTok{ }\NormalTok{df}\OperatorTok{$}\NormalTok{variance }\OperatorTok{+}\StringTok{ }\NormalTok{df}\OperatorTok{$}\StringTok{`}\DataTypeTok{irreducible error}\StringTok{`}
\NormalTok{hdf =}\StringTok{ }\KeywordTok{melt}\NormalTok{(df, }\DataTypeTok{id =} \KeywordTok{c}\NormalTok{(}\StringTok{"x"}\NormalTok{, }\StringTok{"poly"}\NormalTok{))}
\NormalTok{hdf1 =}\StringTok{ }\NormalTok{hdf[hdf}\OperatorTok{$}\NormalTok{poly }\OperatorTok{==}\StringTok{ }\DecValTok{1}\NormalTok{, ]}
\NormalTok{hdf2 =}\StringTok{ }\NormalTok{hdf[hdf}\OperatorTok{$}\NormalTok{poly }\OperatorTok{==}\StringTok{ }\DecValTok{2}\NormalTok{, ]}
\NormalTok{hdf10 =}\StringTok{ }\NormalTok{hdf[hdf}\OperatorTok{$}\NormalTok{poly }\OperatorTok{==}\StringTok{ }\DecValTok{10}\NormalTok{, ]}
\NormalTok{hdf20 =}\StringTok{ }\NormalTok{hdf[hdf}\OperatorTok{$}\NormalTok{poly }\OperatorTok{==}\StringTok{ }\DecValTok{20}\NormalTok{, ]}
\NormalTok{p1 =}\StringTok{ }\KeywordTok{ggplot}\NormalTok{(}\DataTypeTok{data =}\NormalTok{ hdf1, }\KeywordTok{aes}\NormalTok{(}\DataTypeTok{x =}\NormalTok{ x, }\DataTypeTok{y =}\NormalTok{ value, }\DataTypeTok{colour =}\NormalTok{ variable)) }\OperatorTok{+}\StringTok{ }
\StringTok{    }\KeywordTok{geom_line}\NormalTok{() }\OperatorTok{+}\StringTok{ }\KeywordTok{ggtitle}\NormalTok{(}\StringTok{"poly1"}\NormalTok{)}
\NormalTok{p2 =}\StringTok{ }\KeywordTok{ggplot}\NormalTok{(}\DataTypeTok{data =}\NormalTok{ hdf2, }\KeywordTok{aes}\NormalTok{(}\DataTypeTok{x =}\NormalTok{ x, }\DataTypeTok{y =}\NormalTok{ value, }\DataTypeTok{colour =}\NormalTok{ variable)) }\OperatorTok{+}\StringTok{ }
\StringTok{    }\KeywordTok{geom_line}\NormalTok{() }\OperatorTok{+}\StringTok{ }\KeywordTok{ggtitle}\NormalTok{(}\StringTok{"poly2"}\NormalTok{)}
\NormalTok{p10 =}\StringTok{ }\KeywordTok{ggplot}\NormalTok{(}\DataTypeTok{data =}\NormalTok{ hdf10, }\KeywordTok{aes}\NormalTok{(}\DataTypeTok{x =}\NormalTok{ x, }\DataTypeTok{y =}\NormalTok{ value, }\DataTypeTok{colour =}\NormalTok{ variable)) }\OperatorTok{+}\StringTok{ }
\StringTok{    }\KeywordTok{geom_line}\NormalTok{() }\OperatorTok{+}\StringTok{ }\KeywordTok{ggtitle}\NormalTok{(}\StringTok{"poly10"}\NormalTok{)}
\NormalTok{p20 =}\StringTok{ }\KeywordTok{ggplot}\NormalTok{(}\DataTypeTok{data =}\NormalTok{ hdf20, }\KeywordTok{aes}\NormalTok{(}\DataTypeTok{x =}\NormalTok{ x, }\DataTypeTok{y =}\NormalTok{ value, }\DataTypeTok{colour =}\NormalTok{ variable)) }\OperatorTok{+}\StringTok{ }
\StringTok{    }\KeywordTok{geom_line}\NormalTok{() }\OperatorTok{+}\StringTok{ }\KeywordTok{ggtitle}\NormalTok{(}\StringTok{"poly20"}\NormalTok{)}
\KeywordTok{ggarrange}\NormalTok{(p1, p2, p10, p20)}
\end{Highlighting}
\end{Shaded}

\begin{itemize}
\tightlist
\item
  Now plotting effect of more complex model at 4 chosen values of x,
  compare to Figures in 2.12 on page 36 in ISL (our textbook).
\end{itemize}

\begin{Shaded}
\begin{Highlighting}[]
\NormalTok{hdfatxa =}\StringTok{ }\NormalTok{hdf[hdf}\OperatorTok{$}\NormalTok{x }\OperatorTok{==}\StringTok{ }\DecValTok{-1}\NormalTok{, ]}
\NormalTok{hdfatxb =}\StringTok{ }\NormalTok{hdf[hdf}\OperatorTok{$}\NormalTok{x }\OperatorTok{==}\StringTok{ }\FloatTok{0.5}\NormalTok{, ]}
\NormalTok{hdfatxc =}\StringTok{ }\NormalTok{hdf[hdf}\OperatorTok{$}\NormalTok{x }\OperatorTok{==}\StringTok{ }\DecValTok{2}\NormalTok{, ]}
\NormalTok{hdfatxd =}\StringTok{ }\NormalTok{hdf[hdf}\OperatorTok{$}\NormalTok{x }\OperatorTok{==}\StringTok{ }\FloatTok{3.5}\NormalTok{, ]}
\NormalTok{pa =}\StringTok{ }\KeywordTok{ggplot}\NormalTok{(}\DataTypeTok{data =}\NormalTok{ hdfatxa, }\KeywordTok{aes}\NormalTok{(}\DataTypeTok{x =}\NormalTok{ poly, }\DataTypeTok{y =}\NormalTok{ value, }\DataTypeTok{colour =}\NormalTok{ variable)) }\OperatorTok{+}\StringTok{ }
\StringTok{    }\KeywordTok{geom_line}\NormalTok{() }\OperatorTok{+}\StringTok{ }\KeywordTok{ggtitle}\NormalTok{(}\StringTok{"x0=-1"}\NormalTok{)}
\NormalTok{pb =}\StringTok{ }\KeywordTok{ggplot}\NormalTok{(}\DataTypeTok{data =}\NormalTok{ hdfatxb, }\KeywordTok{aes}\NormalTok{(}\DataTypeTok{x =}\NormalTok{ poly, }\DataTypeTok{y =}\NormalTok{ value, }\DataTypeTok{colour =}\NormalTok{ variable)) }\OperatorTok{+}\StringTok{ }
\StringTok{    }\KeywordTok{geom_line}\NormalTok{() }\OperatorTok{+}\StringTok{ }\KeywordTok{ggtitle}\NormalTok{(}\StringTok{"x0=0.5"}\NormalTok{)}
\NormalTok{pc =}\StringTok{ }\KeywordTok{ggplot}\NormalTok{(}\DataTypeTok{data =}\NormalTok{ hdfatxc, }\KeywordTok{aes}\NormalTok{(}\DataTypeTok{x =}\NormalTok{ poly, }\DataTypeTok{y =}\NormalTok{ value, }\DataTypeTok{colour =}\NormalTok{ variable)) }\OperatorTok{+}\StringTok{ }
\StringTok{    }\KeywordTok{geom_line}\NormalTok{() }\OperatorTok{+}\StringTok{ }\KeywordTok{ggtitle}\NormalTok{(}\StringTok{"x0=2"}\NormalTok{)}
\NormalTok{pd =}\StringTok{ }\KeywordTok{ggplot}\NormalTok{(}\DataTypeTok{data =}\NormalTok{ hdfatxd, }\KeywordTok{aes}\NormalTok{(}\DataTypeTok{x =}\NormalTok{ poly, }\DataTypeTok{y =}\NormalTok{ value, }\DataTypeTok{colour =}\NormalTok{ variable)) }\OperatorTok{+}\StringTok{ }
\StringTok{    }\KeywordTok{geom_line}\NormalTok{() }\OperatorTok{+}\StringTok{ }\KeywordTok{ggtitle}\NormalTok{(}\StringTok{"x0=3.5"}\NormalTok{)}
\KeywordTok{ggarrange}\NormalTok{(pa, pb, pc, pd)}
\end{Highlighting}
\end{Shaded}

\begin{itemize}
\tightlist
\item
  Study the final plot you produced: when the flexibility increases
  (poly increase), what happens with

  \begin{enumerate}
  \def\labelenumi{\roman{enumi})}
  \tightlist
  \item
    the squared bias,
  \item
    the variance,
  \item
    the irreducible error?
  \end{enumerate}
\end{itemize}

\begin{center}\rule{0.5\linewidth}{\linethickness}\end{center}

\subsubsection{d) Repeat a-c}\label{d-repeat-a-c}

\begin{itemize}
\tightlist
\item
  Then try to change the true function \texttt{truefunc} to something
  else - mayby order 3? What does this do the the plots produced? Maybe
  you then also want to plot poly3?
\item
  Also try to change the standard deviation of the noise added to the
  curve (now it is sd=2). What happens if you change this to sd=1 or
  sd=3?
\item
  Or, change to the true function that is not a polynomial?
\end{itemize}

\begin{center}\rule{0.5\linewidth}{\linethickness}\end{center}

\section{Acknowledgements}\label{acknowledgements}

We thank Mette Langaas and her PhD students (in particular Julia Debik)
from 2018 and 2019 for building up the original version of this exercise
sheet.


\end{document}
