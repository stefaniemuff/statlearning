\documentclass[]{article}
\usepackage{lmodern}
\usepackage{amssymb,amsmath}
\usepackage{ifxetex,ifluatex}
\usepackage{fixltx2e} % provides \textsubscript
\ifnum 0\ifxetex 1\fi\ifluatex 1\fi=0 % if pdftex
  \usepackage[T1]{fontenc}
  \usepackage[utf8]{inputenc}
\else % if luatex or xelatex
  \ifxetex
    \usepackage{mathspec}
  \else
    \usepackage{fontspec}
  \fi
  \defaultfontfeatures{Ligatures=TeX,Scale=MatchLowercase}
\fi
% use upquote if available, for straight quotes in verbatim environments
\IfFileExists{upquote.sty}{\usepackage{upquote}}{}
% use microtype if available
\IfFileExists{microtype.sty}{%
\usepackage{microtype}
\UseMicrotypeSet[protrusion]{basicmath} % disable protrusion for tt fonts
}{}
\usepackage[margin=1in]{geometry}
\usepackage{hyperref}
\PassOptionsToPackage{usenames,dvipsnames}{color} % color is loaded by hyperref
\hypersetup{unicode=true,
            pdftitle={Part I: R and RStudio},
            pdfauthor={Martina Hall, Michail Spitieris, Stefanie Muff, Department of Mathematical Sciences, NTNU},
            colorlinks=true,
            linkcolor=Maroon,
            citecolor=Blue,
            urlcolor=blue,
            breaklinks=true}
\urlstyle{same}  % don't use monospace font for urls
\usepackage{color}
\usepackage{fancyvrb}
\newcommand{\VerbBar}{|}
\newcommand{\VERB}{\Verb[commandchars=\\\{\}]}
\DefineVerbatimEnvironment{Highlighting}{Verbatim}{commandchars=\\\{\}}
% Add ',fontsize=\small' for more characters per line
\usepackage{framed}
\definecolor{shadecolor}{RGB}{248,248,248}
\newenvironment{Shaded}{\begin{snugshade}}{\end{snugshade}}
\newcommand{\KeywordTok}[1]{\textcolor[rgb]{0.13,0.29,0.53}{\textbf{#1}}}
\newcommand{\DataTypeTok}[1]{\textcolor[rgb]{0.13,0.29,0.53}{#1}}
\newcommand{\DecValTok}[1]{\textcolor[rgb]{0.00,0.00,0.81}{#1}}
\newcommand{\BaseNTok}[1]{\textcolor[rgb]{0.00,0.00,0.81}{#1}}
\newcommand{\FloatTok}[1]{\textcolor[rgb]{0.00,0.00,0.81}{#1}}
\newcommand{\ConstantTok}[1]{\textcolor[rgb]{0.00,0.00,0.00}{#1}}
\newcommand{\CharTok}[1]{\textcolor[rgb]{0.31,0.60,0.02}{#1}}
\newcommand{\SpecialCharTok}[1]{\textcolor[rgb]{0.00,0.00,0.00}{#1}}
\newcommand{\StringTok}[1]{\textcolor[rgb]{0.31,0.60,0.02}{#1}}
\newcommand{\VerbatimStringTok}[1]{\textcolor[rgb]{0.31,0.60,0.02}{#1}}
\newcommand{\SpecialStringTok}[1]{\textcolor[rgb]{0.31,0.60,0.02}{#1}}
\newcommand{\ImportTok}[1]{#1}
\newcommand{\CommentTok}[1]{\textcolor[rgb]{0.56,0.35,0.01}{\textit{#1}}}
\newcommand{\DocumentationTok}[1]{\textcolor[rgb]{0.56,0.35,0.01}{\textbf{\textit{#1}}}}
\newcommand{\AnnotationTok}[1]{\textcolor[rgb]{0.56,0.35,0.01}{\textbf{\textit{#1}}}}
\newcommand{\CommentVarTok}[1]{\textcolor[rgb]{0.56,0.35,0.01}{\textbf{\textit{#1}}}}
\newcommand{\OtherTok}[1]{\textcolor[rgb]{0.56,0.35,0.01}{#1}}
\newcommand{\FunctionTok}[1]{\textcolor[rgb]{0.00,0.00,0.00}{#1}}
\newcommand{\VariableTok}[1]{\textcolor[rgb]{0.00,0.00,0.00}{#1}}
\newcommand{\ControlFlowTok}[1]{\textcolor[rgb]{0.13,0.29,0.53}{\textbf{#1}}}
\newcommand{\OperatorTok}[1]{\textcolor[rgb]{0.81,0.36,0.00}{\textbf{#1}}}
\newcommand{\BuiltInTok}[1]{#1}
\newcommand{\ExtensionTok}[1]{#1}
\newcommand{\PreprocessorTok}[1]{\textcolor[rgb]{0.56,0.35,0.01}{\textit{#1}}}
\newcommand{\AttributeTok}[1]{\textcolor[rgb]{0.77,0.63,0.00}{#1}}
\newcommand{\RegionMarkerTok}[1]{#1}
\newcommand{\InformationTok}[1]{\textcolor[rgb]{0.56,0.35,0.01}{\textbf{\textit{#1}}}}
\newcommand{\WarningTok}[1]{\textcolor[rgb]{0.56,0.35,0.01}{\textbf{\textit{#1}}}}
\newcommand{\AlertTok}[1]{\textcolor[rgb]{0.94,0.16,0.16}{#1}}
\newcommand{\ErrorTok}[1]{\textcolor[rgb]{0.64,0.00,0.00}{\textbf{#1}}}
\newcommand{\NormalTok}[1]{#1}
\usepackage{longtable,booktabs}
\usepackage{graphicx,grffile}
\makeatletter
\def\maxwidth{\ifdim\Gin@nat@width>\linewidth\linewidth\else\Gin@nat@width\fi}
\def\maxheight{\ifdim\Gin@nat@height>\textheight\textheight\else\Gin@nat@height\fi}
\makeatother
% Scale images if necessary, so that they will not overflow the page
% margins by default, and it is still possible to overwrite the defaults
% using explicit options in \includegraphics[width, height, ...]{}
\setkeys{Gin}{width=\maxwidth,height=\maxheight,keepaspectratio}
\IfFileExists{parskip.sty}{%
\usepackage{parskip}
}{% else
\setlength{\parindent}{0pt}
\setlength{\parskip}{6pt plus 2pt minus 1pt}
}
\setlength{\emergencystretch}{3em}  % prevent overfull lines
\providecommand{\tightlist}{%
  \setlength{\itemsep}{0pt}\setlength{\parskip}{0pt}}
\setcounter{secnumdepth}{0}
% Redefines (sub)paragraphs to behave more like sections
\ifx\paragraph\undefined\else
\let\oldparagraph\paragraph
\renewcommand{\paragraph}[1]{\oldparagraph{#1}\mbox{}}
\fi
\ifx\subparagraph\undefined\else
\let\oldsubparagraph\subparagraph
\renewcommand{\subparagraph}[1]{\oldsubparagraph{#1}\mbox{}}
\fi

%%% Use protect on footnotes to avoid problems with footnotes in titles
\let\rmarkdownfootnote\footnote%
\def\footnote{\protect\rmarkdownfootnote}

%%% Change title format to be more compact
\usepackage{titling}

% Create subtitle command for use in maketitle
\providecommand{\subtitle}[1]{
  \posttitle{
    \begin{center}\large#1\end{center}
    }
}

\setlength{\droptitle}{-2em}

  \title{Part I: R and RStudio}
    \pretitle{\vspace{\droptitle}\centering\huge}
  \posttitle{\par}
  \subtitle{TMA4268 Statistical Learning V2020. Module 1: INTRODUCTION TO
STATISTICAL LEARNING}
  \author{Martina Hall, Michail Spitieris, Stefanie Muff, Department of
Mathematical Sciences, NTNU}
    \preauthor{\centering\large\emph}
  \postauthor{\par}
      \predate{\centering\large\emph}
  \postdate{\par}
    \date{January 6th, 2020}


\begin{document}
\maketitle

(Latest changes: 03.01.20: first version for 2020)

R is a free software environment for statistical computing and graphics.
It runs on a wide variety of UNIX platforms, Windows and MacOS. R can be
downloaded from \url{http://www.r-project.org/}.

We recommend that you run R using the RStudio IDE (integrated
development environment). RStudio can be downloaded from
\url{http://rstudio.org/}.\\
Notice: you need to download both R and RStudio.

If you need help on installing R and RStudio on you laptop computer,
contact \href{mailto:orakel@ntnu.no}{\nolinkurl{orakel@ntnu.no}}. If you
want to work at Nullrommet or Banachrommet at Matteland, R and RStudio
is already installed for you.

\section{Part A: Using RStudio - what are the different
windows?}\label{part-a-using-rstudio---what-are-the-different-windows}

Start RStudio. Then you (probably) have the following four windows.

\begin{itemize}
\item
  \textbf{Source} (aka script window) - upper left window: where you
  write your code and keep track of your work.
\item
  \textbf{Console} - lower left window: where the R commands are
  executed (so here is where you R installation lives). Sometimes also
  referred to as command window.
\item
  \textbf{Environment/History/Connections/Presentation} - upper right
  window: the objects that you have in your workspace, and the commands
  you have executed, and more.
\item
  \textbf{Files/Plots/Packages/Help/Viewer}- lower right: overview of
  your files, the plots you produce, the packages you have installed and
  loaded, and more.
\end{itemize}

\textbf{Source window}: (Make the source window active.) To start
writing a script press File- New File- R Script. To open an existing
file, press File- Open File- and select the file you want to open. The
file will open in the source window. To save this file, press File- Save
as- and go to the working directory there you want to put your TMA4268 R
files and save the file as ``name''.R (example: \texttt{myRintro.R}).
Files with R code usually have extension \texttt{.R}.

\textbf{Console window}: (Make the console window active.) To see your
working directory, you can write \texttt{getwd()}, and you will get your
location as output. You can also set your working directory to a certain
folder of choise by writing \texttt{setwd("location")} (Example:
\texttt{setwd("M:/Documents/TMA4268/")}). Now you are certain that all
your files will be put in this folder.

\textbf{Quitting}: It is always important to be able to quit a program:
when you are finished you may choose RStudio-Quit Rstudio (top menu
outside of the windows). Alternatively, you may write \texttt{q()}in the
console window to quit R (the parenthesis is because q is a function
that can have arguments to be given within the parentheses and you call
the function without any arguments). You will be asked if you want to
save your script and workspave. If you want to reuse your script later
(and of cause you want to do that - we aim at reproduciable research),
you should save it! If you answer yes to ``Save workspace image'' all
the objects you have created are found in a \texttt{.RData} file (more
about objects soon). This could be useful if you don't want to run all
the commands in the script again, because if you start R in the same
working directory all the objects you have created will be automatically
available to you. However, we recommend to \textbf{not} save
\texttt{.RData} files unless you really need them for later use (could
take up much space!). In order to ensure reproducible code, \textbf{we
recommend to turn off the option to automatically save the workspace} by
going into Tools-\textgreater{}Global
Options..-\textgreater{}General-\textgreater{}Save workspace to .RData
on exit-\textgreater{} Never.

You can download the \textbf{RStudio IDE cheat sheet}:
\url{https://github.com/rstudio/cheatsheets/raw/master/rstudio-ide.pdf}

\section{Part B: Trying out
R-commands}\label{part-b-trying-out-r-commands}

To exceute your commands, you can either type directly in the console or
run the commands from the source window. In the \textbf{source window},
you can run the current line by pressing Ctrl and Enter (Windows) or CMD
and Enter (MacOS), or you can run selected lines by marking them and
pressing Ctrl + Enter. You can also use the Run-button in the top right
corner of the window to run selected lines or commands, and the
Source-button in the top right corner to run everything in your Source
window. \textbf{We recomend to always use the source window and save the
script, in this way your code will not disappear!}

\textbf{Q}: Open a new script and save it as \texttt{"MyRbeginner.R"}.
Write the following commands into your script and execute them
one-by-one. What have you done mathematically here?

\begin{Shaded}
\begin{Highlighting}[]
\DecValTok{2} \OperatorTok{+}\StringTok{ }\DecValTok{3}
\DecValTok{2} \OperatorTok{*}\StringTok{ }\DecValTok{6}
\DecValTok{3} \OperatorTok{*}\StringTok{ }\DecValTok{10}\OperatorTok{^}\DecValTok{4} \OperatorTok{-}\StringTok{ }\DecValTok{3} \OperatorTok{*}\StringTok{ }\DecValTok{5}\OperatorTok{^}\DecValTok{2}
\DecValTok{10}\OperatorTok{^}\DecValTok{2} \OperatorTok{-}\StringTok{ }\DecValTok{1}
\DecValTok{10}\OperatorTok{^}\NormalTok{(}\DecValTok{2} \OperatorTok{-}\StringTok{ }\DecValTok{1}\NormalTok{)}
\KeywordTok{sqrt}\NormalTok{(}\DecValTok{9}\NormalTok{)}
\KeywordTok{log}\NormalTok{(}\DecValTok{3}\NormalTok{, }\DataTypeTok{base =} \DecValTok{10}\NormalTok{)}
\StringTok{`}\DataTypeTok{?}\StringTok{`}\NormalTok{(log)}
\NormalTok{log}
\KeywordTok{log10}\NormalTok{(}\DecValTok{3}\NormalTok{)}
\KeywordTok{log}\NormalTok{(}\DecValTok{3}\NormalTok{)}
\KeywordTok{exp}\NormalTok{(}\DecValTok{34}\NormalTok{)}
\KeywordTok{gamma}\NormalTok{(}\DecValTok{3}\NormalTok{)}
\KeywordTok{factorial}\NormalTok{(}\DecValTok{5}\NormalTok{)}
\KeywordTok{choose}\NormalTok{(}\DecValTok{10}\NormalTok{, }\DecValTok{4}\NormalTok{)}
\DecValTok{1}\OperatorTok{:}\DecValTok{4}
\KeywordTok{c}\NormalTok{(}\DecValTok{1}\NormalTok{, }\DecValTok{2}\NormalTok{, }\DecValTok{3}\NormalTok{, }\DecValTok{4}\NormalTok{)}
\KeywordTok{seq}\NormalTok{(}\DataTypeTok{from =} \DecValTok{1}\NormalTok{, }\DataTypeTok{to =} \DecValTok{4}\NormalTok{, }\DataTypeTok{by =} \DecValTok{1}\NormalTok{)}
\KeywordTok{sum}\NormalTok{(}\DecValTok{1}\OperatorTok{:}\DecValTok{5}\NormalTok{)}
\KeywordTok{prod}\NormalTok{(}\DecValTok{1}\OperatorTok{:}\DecValTok{5}\NormalTok{)}
\NormalTok{heights =}\StringTok{ }\KeywordTok{c}\NormalTok{(}\DecValTok{192}\NormalTok{, }\DecValTok{185}\NormalTok{, }\DecValTok{174}\NormalTok{, }\DecValTok{195}\NormalTok{, }\DecValTok{173}\NormalTok{)}
\NormalTok{shoes =}\StringTok{ }\KeywordTok{c}\NormalTok{(}\DecValTok{46}\NormalTok{, }\DecValTok{43}\NormalTok{, }\DecValTok{40}\NormalTok{, }\DecValTok{45}\NormalTok{, }\DecValTok{40}\NormalTok{)}
\NormalTok{ratio <-}\StringTok{ }\NormalTok{heights}\OperatorTok{/}\NormalTok{shoes}
\NormalTok{ratio}
\end{Highlighting}
\end{Shaded}

Here we have created three objects: \texttt{heights}, \texttt{shoes} and
\texttt{ratio}. Observe: we can both use \texttt{=} and
\texttt{\textless{}-} for assigning content to an object. Notice now
that the objects you assigned values to
(\texttt{heights,\ shows,\ ratio}) appear in the Enviroment window
(sorted as Data, Values or Functions, but you should only have Values so
far).

The function \texttt{c} combines values into a vector (concatenate).
Also, all the commands you have run are reported in the History window.

If you want to add comments, you do that by starting with a hashtag
symbol:

\begin{Shaded}
\begin{Highlighting}[]
\CommentTok{# now we quit}
\KeywordTok{q}\NormalTok{()}
\end{Highlighting}
\end{Shaded}

Save your work - and we will try to open it later.

\section{Part C: Vectors and
matrices}\label{part-c-vectors-and-matrices}

R can handle both numeric and non numeric data. The concatenate
\texttt{c}-function can handle both numeric and non-numeric data, but be
careful when mixing them.

\textbf{Q}: Go through theses commands and see what is produced.

\begin{Shaded}
\begin{Highlighting}[]
\NormalTok{x =}\StringTok{ }\KeywordTok{c}\NormalTok{(}\DecValTok{1}\NormalTok{, }\DecValTok{2}\NormalTok{, }\DecValTok{3}\NormalTok{)}
\KeywordTok{typeof}\NormalTok{(x)}
\NormalTok{y =}\StringTok{ }\KeywordTok{c}\NormalTok{(}\StringTok{"a"}\NormalTok{, }\StringTok{"b"}\NormalTok{, }\StringTok{"c"}\NormalTok{)}
\KeywordTok{typeof}\NormalTok{(y)}
\NormalTok{u =}\StringTok{ }\KeywordTok{c}\NormalTok{(}\StringTok{"1"}\NormalTok{, }\StringTok{"2"}\NormalTok{, }\StringTok{"3"}\NormalTok{)}
\KeywordTok{typeof}\NormalTok{(u)}
\NormalTok{v =}\StringTok{ }\KeywordTok{as.numeric}\NormalTok{(u)}
\KeywordTok{typeof}\NormalTok{(v)}
\NormalTok{z =}\StringTok{ }\KeywordTok{c}\NormalTok{(}\StringTok{"red"}\NormalTok{, }\DecValTok{1}\NormalTok{, }\StringTok{"yellow"}\NormalTok{, }\DecValTok{2}\NormalTok{)}
\KeywordTok{typeof}\NormalTok{(z)}
\CommentTok{# w = z - 1 this gives error}
\end{Highlighting}
\end{Shaded}

\begin{verbatim}
## [1] "double"
## [1] "character"
## [1] "character"
## [1] "double"
## [1] "character"
\end{verbatim}

Logical operators are also available, \texttt{==} for equality,
\texttt{!=} for not equal to, \texttt{\textgreater{}=} for greater than
or equal to, etc.

\begin{Shaded}
\begin{Highlighting}[]
\NormalTok{gender =}\StringTok{ }\KeywordTok{factor}\NormalTok{(}\KeywordTok{c}\NormalTok{(}\StringTok{"male"}\NormalTok{, }\StringTok{"female"}\NormalTok{, }\StringTok{"female"}\NormalTok{, }\StringTok{"male"}\NormalTok{))}
\NormalTok{gender}
\KeywordTok{sum}\NormalTok{(gender }\OperatorTok{==}\StringTok{ "male"}\NormalTok{)}
\KeywordTok{table}\NormalTok{(gender)}
\end{Highlighting}
\end{Shaded}

\begin{verbatim}
## [1] male   female female male  
## Levels: female male
## [1] 2
## gender
## female   male 
##      2      2
\end{verbatim}

Some useful code to work with vectors. First we will look at three
equivalent ways to generate a vector \(x=(1,2,3,4,5)^\top\), and then do
some (logical) operations. After each line, you can type \texttt{x} into
the console to see how \texttt{x} was modified:

\begin{Shaded}
\begin{Highlighting}[]
\NormalTok{x =}\StringTok{ }\DecValTok{1}\OperatorTok{:}\DecValTok{5}
\NormalTok{x =}\StringTok{ }\KeywordTok{seq}\NormalTok{(}\DataTypeTok{from =} \DecValTok{1}\NormalTok{, }\DataTypeTok{to =} \DecValTok{5}\NormalTok{, }\DataTypeTok{length =} \DecValTok{5}\NormalTok{)}
\NormalTok{x =}\StringTok{ }\KeywordTok{c}\NormalTok{(}\DecValTok{1}\NormalTok{, }\DecValTok{2}\NormalTok{, }\DecValTok{3}\NormalTok{, }\DecValTok{4}\NormalTok{, }\DecValTok{5}\NormalTok{)}
\DecValTok{2} \OperatorTok\StringTok{ }\NormalTok{x}
\DecValTok{6} \OperatorTok\StringTok{ }\NormalTok{x}
\NormalTok{x[}\DecValTok{2}\NormalTok{]}
\NormalTok{x[}\DecValTok{2}\NormalTok{] =}\StringTok{ }\DecValTok{10}
\NormalTok{x[}\DecValTok{3}\OperatorTok{:}\DecValTok{4}\NormalTok{] =}\StringTok{ }\DecValTok{0}
\NormalTok{x[}\OperatorTok{-}\DecValTok{2}\NormalTok{] =}\StringTok{ }\DecValTok{1}
\NormalTok{x[}\KeywordTok{c}\NormalTok{(}\DecValTok{1}\NormalTok{, }\DecValTok{4}\NormalTok{)] =}\StringTok{ }\DecValTok{4}
\NormalTok{x[x }\OperatorTok{>}\StringTok{ }\DecValTok{4}\NormalTok{] =}\StringTok{ }\DecValTok{10}
\NormalTok{x}
\NormalTok{y =}\StringTok{ }\KeywordTok{log}\NormalTok{(x)}
\NormalTok{z =}\StringTok{ }\KeywordTok{exp}\NormalTok{(y)}
\NormalTok{z =}\StringTok{ }\NormalTok{z }\OperatorTok{+}\StringTok{ }\NormalTok{y}
\NormalTok{y =}\StringTok{ }\NormalTok{x }\OperatorTok{*}\StringTok{ }\NormalTok{y}
\NormalTok{z =}\StringTok{ }\NormalTok{y}\OperatorTok{/}\NormalTok{x}
\NormalTok{a =}\StringTok{ }\KeywordTok{t}\NormalTok{(x) }\OperatorTok\StringTok{ }\NormalTok{y  }\CommentTok{# t(): transpose}
\KeywordTok{min}\NormalTok{(x)}
\KeywordTok{max}\NormalTok{(x)}
\KeywordTok{sum}\NormalTok{(x)}
\KeywordTok{mean}\NormalTok{(x)}
\KeywordTok{var}\NormalTok{(x)}
\KeywordTok{length}\NormalTok{(x)}
\KeywordTok{sort}\NormalTok{(x)}
\KeywordTok{order}\NormalTok{(x)}
\KeywordTok{sort}\NormalTok{(x) }\OperatorTok{==}\StringTok{ }\NormalTok{x[}\KeywordTok{order}\NormalTok{(x)]}
\KeywordTok{sample}\NormalTok{(}\DecValTok{1}\OperatorTok{:}\DecValTok{10}\NormalTok{)}
\KeywordTok{sample}\NormalTok{(}\DecValTok{1}\OperatorTok{:}\DecValTok{10}\NormalTok{, }\DataTypeTok{replace =}\NormalTok{ T)}
\end{Highlighting}
\end{Shaded}

Notice the length of your vectors when doing calculations with two
vectors. Try to understand what the following operations do:

\begin{Shaded}
\begin{Highlighting}[]
\NormalTok{x =}\StringTok{ }\DecValTok{1}\OperatorTok{:}\DecValTok{5}
\NormalTok{y =}\StringTok{ }\DecValTok{2}
\NormalTok{x }\OperatorTok{-}\StringTok{ }\NormalTok{y}
\DecValTok{5} \OperatorTok{*}\StringTok{ }\NormalTok{x}
\NormalTok{z =}\StringTok{ }\DecValTok{10}\OperatorTok{:}\DecValTok{15}
\NormalTok{w =}\StringTok{ }\DecValTok{1}\OperatorTok{:}\DecValTok{2}
\NormalTok{z }\OperatorTok{-}\StringTok{ }\NormalTok{w}
\end{Highlighting}
\end{Shaded}

\begin{verbatim}
## [1] -1  0  1  2  3
## [1]  5 10 15 20 25
## [1]  9  9 11 11 13 13
\end{verbatim}

What happens here?

Now let's look at matrices, and start to generate a \(3\times 2\) matrix
\texttt{A}:

\begin{Shaded}
\begin{Highlighting}[]
\NormalTok{A =}\StringTok{ }\KeywordTok{matrix}\NormalTok{(}\DecValTok{1}\OperatorTok{:}\DecValTok{6}\NormalTok{, }\DataTypeTok{nrow =} \DecValTok{3}\NormalTok{, }\DataTypeTok{ncol =} \DecValTok{2}\NormalTok{)}
\NormalTok{A}
\end{Highlighting}
\end{Shaded}

\begin{verbatim}
##      [,1] [,2]
## [1,]    1    4
## [2,]    2    5
## [3,]    3    6
\end{verbatim}

\begin{Shaded}
\begin{Highlighting}[]
\NormalTok{B =}\StringTok{ }\KeywordTok{matrix}\NormalTok{(}\DecValTok{1}\OperatorTok{:}\DecValTok{6}\NormalTok{, }\DataTypeTok{nrow =} \DecValTok{2}\NormalTok{, }\DataTypeTok{ncol =} \DecValTok{3}\NormalTok{, }\DataTypeTok{byrow =} \OtherTok{TRUE}\NormalTok{)}
\NormalTok{B}
\end{Highlighting}
\end{Shaded}

\begin{verbatim}
##      [,1] [,2] [,3]
## [1,]    1    2    3
## [2,]    4    5    6
\end{verbatim}

We hope you do remember matrix multiplication. If two matrices
\texttt{A} and \texttt{B} fulfil certain dimension criteria, they can be
multiplied. In \texttt{R} this is done using the \texttt{\%*\%}
operation. Try out what the following lines of code do:

\begin{Shaded}
\begin{Highlighting}[]
\NormalTok{A }\OperatorTok\StringTok{ }\NormalTok{B  }\CommentTok{# matrix multiplication}
\NormalTok{A }\OperatorTok{*}\StringTok{ }\KeywordTok{t}\NormalTok{(B)}
\NormalTok{A }\OperatorTok\StringTok{ }\KeywordTok{t}\NormalTok{(A)}
\NormalTok{A}\OperatorTok{^}\DecValTok{2}
\end{Highlighting}
\end{Shaded}

\begin{verbatim}
##      [,1] [,2] [,3]
## [1,]   17   22   27
## [2,]   22   29   36
## [3,]   27   36   45
##      [,1] [,2]
## [1,]    1   16
## [2,]    4   25
## [3,]    9   36
##      [,1] [,2] [,3]
## [1,]   17   22   27
## [2,]   22   29   36
## [3,]   27   36   45
##      [,1] [,2]
## [1,]    1   16
## [2,]    4   25
## [3,]    9   36
\end{verbatim}

The functions \texttt{cbind} (column bind) and \texttt{rbind} (row bind)
can also be used to create matrices:

\begin{Shaded}
\begin{Highlighting}[]
\NormalTok{x1 =}\StringTok{ }\DecValTok{1}\OperatorTok{:}\DecValTok{3}
\NormalTok{x2 =}\StringTok{ }\KeywordTok{c}\NormalTok{(}\DecValTok{7}\NormalTok{, }\DecValTok{6}\NormalTok{, }\DecValTok{6}\NormalTok{)}
\NormalTok{x3 =}\StringTok{ }\KeywordTok{c}\NormalTok{(}\DecValTok{12}\NormalTok{, }\DecValTok{19}\NormalTok{, }\DecValTok{21}\NormalTok{)}
\NormalTok{A =}\StringTok{ }\KeywordTok{cbind}\NormalTok{(x1, x2, x3)  }\CommentTok{# Bind vectors x1, x2, and x3 into a matrix.}
\CommentTok{# Treats each as a column.}
\NormalTok{A =}\StringTok{ }\KeywordTok{rbind}\NormalTok{(x1, x2, x3)  }\CommentTok{# Bind vectors x1, x2, and x3 into a matrix.}
\CommentTok{# Treats each as a row.}
\end{Highlighting}
\end{Shaded}

Other matrix commands are

\begin{Shaded}
\begin{Highlighting}[]
\KeywordTok{dim}\NormalTok{(A)  }\CommentTok{# get the dimensions of a matrix}
\KeywordTok{nrow}\NormalTok{(A)  }\CommentTok{# number of rows}
\KeywordTok{ncol}\NormalTok{(A)  }\CommentTok{# number of columns}
\KeywordTok{apply}\NormalTok{(A, }\DecValTok{1}\NormalTok{, sum)  }\CommentTok{# apply the sum function to the rows of A}
\KeywordTok{apply}\NormalTok{(A, }\DecValTok{2}\NormalTok{, sum)  }\CommentTok{# apply the sum function to the columns of A}
\KeywordTok{sum}\NormalTok{(}\KeywordTok{diag}\NormalTok{(A))  }\CommentTok{# trace of A}
\NormalTok{A =}\StringTok{ }\KeywordTok{diag}\NormalTok{(}\DecValTok{1}\OperatorTok{:}\DecValTok{3}\NormalTok{)}
\KeywordTok{solve}\NormalTok{(A)  }\CommentTok{# inverse of A, in general solve(A,b) solves Ax=b wrt x}
\KeywordTok{det}\NormalTok{(A)  }\CommentTok{# determinant of A}
\end{Highlighting}
\end{Shaded}

\section{Part D: Basics about
plotting}\label{part-d-basics-about-plotting}

In \texttt{R} there are nowadays two paradigms for how to plot: The
base-R plotting universe, and the (more modern) \texttt{ggplot()}
approach using the \texttt{ggplot2} package in R. The latter will be
discussed more in the \texttt{Rplots} exercise part, and is therefore
only briefly mentioned below. In the base-R version, the functions
\texttt{plot()}, \texttt{histogram()} and \texttt{boxplot()} are the
most frequently used functions. Although these functions are (by some)
considered to be a bit old-fashioned, they are often useful to get a
quick first impression of your data (and in fact, some people continue
to use them as their main starting point to plot).

To create a plot, we simulate some data and plot them:

\begin{Shaded}
\begin{Highlighting}[]
\NormalTok{x <-}\StringTok{ }\KeywordTok{seq}\NormalTok{(}\OperatorTok{-}\DecValTok{4}\NormalTok{, }\DecValTok{4}\NormalTok{, }\DataTypeTok{length =} \DecValTok{500}\NormalTok{)}
\NormalTok{y <-}\StringTok{ }\NormalTok{x}\OperatorTok{^}\DecValTok{2} \OperatorTok{-}\StringTok{ }\DecValTok{1}
\KeywordTok{plot}\NormalTok{(x, y, }\DataTypeTok{type =} \StringTok{"l"}\NormalTok{, }\DataTypeTok{main =} \StringTok{"My plot"}\NormalTok{, }\DataTypeTok{xlab =} \StringTok{"x"}\NormalTok{, }\DataTypeTok{ylab =} \StringTok{"y"}\NormalTok{)}
\KeywordTok{abline}\NormalTok{(}\DataTypeTok{v =} \DecValTok{3}\NormalTok{)}
\KeywordTok{abline}\NormalTok{(}\DataTypeTok{h =} \DecValTok{5}\NormalTok{)}
\end{Highlighting}
\end{Shaded}

\includegraphics{Rbeginner_files/figure-latex/unnamed-chunk-12-1.pdf}

To draw the plot in the way you want, check the help pages of the plot
function to see which input values you can change to make your plot look
the way you want to.

The package \texttt{ggplot2} is a powerful tool for making nice plots.
In this package, the function \texttt{ggplot(data,aes(x=x,y=y))} makes
the foundation of the plot with the data \texttt{data} and features of
the plot are added with \texttt{+geom\_point()},
\texttt{+geom\_boxplot()}, \texttt{+xlab("Temperature")} ect. If you
want to learn more about the grammar of graphics you should start by
reading \href{http://r4ds.had.co.nz/data-visualisation.html}{Chapter 3
of the book \texttt{R\ for\ Data\ Science}}.

The list below shows some basic tools for plotting using both
\texttt{plot()}and \texttt{ggplot()} (the last from the \texttt{ggplot2}
package). For \texttt{ggplot()} we must first save the data into a
object and build the foundation of the plot with \texttt{ggplot()}
before we run these functions. See ``Visualizations in R'' for more
plotting and check out \url{https://ggplot2.tidyverse.org/reference/}
for more functions of the \texttt{ggplot2} package.

\begin{longtable}[]{@{}lll@{}}
\toprule
Description & Base Graphics & ggplot2\tabularnewline
\midrule
\endhead
Plot y versus x using points & plot(x, y) & +
geom\_point()\tabularnewline
Boxplot of x & boxplot(x) & + geom\_boxplot()\tabularnewline
Histogram of x & hist(x) & + geom\_histogram()\tabularnewline
\bottomrule
\end{longtable}

Plotting will be an important part of any statistical analysis course.

\section{Part E: Writing a simple
function}\label{part-e-writing-a-simple-function}

Like in other programming languages, R allows you to write functions.
Functions are a collection of commands that are bundled. This allows you
to avoid copy-past around code, and you are certainly familiar with this
idea from Python.

When starting a function R, you should start with the name of the
function and state if the function takes input values. Then you write
the function code inside the branches \({}\). Remember to return the
output of the function using \texttt{return()}.

\begin{Shaded}
\begin{Highlighting}[]
\NormalTok{myfunction <-}\StringTok{ }\ControlFlowTok{function}\NormalTok{(x, y) }\CommentTok{# myfunction is the name, x and y are the names of the inputs}
\NormalTok{\{}
\NormalTok{    n <-}\StringTok{ }\KeywordTok{c}\NormalTok{(}\KeywordTok{length}\NormalTok{(x), }\KeywordTok{length}\NormalTok{(y))}
\NormalTok{    m <-}\StringTok{ }\KeywordTok{c}\NormalTok{(}\KeywordTok{sum}\NormalTok{(x), }\KeywordTok{sum}\NormalTok{(y))}
\NormalTok{    p <-}\StringTok{ }\NormalTok{m}\OperatorTok{/}\NormalTok{n}
    \KeywordTok{return}\NormalTok{(p)}
\NormalTok{\}}
\end{Highlighting}
\end{Shaded}

\textbf{Q}: What does the function return?

To start using the function, you must first run it through the console
so that it is in your enviroment (mark and run). Then you call the
function name and give your inputs like this.

\begin{Shaded}
\begin{Highlighting}[]
\NormalTok{a =}\StringTok{ }\DecValTok{1}\OperatorTok{:}\DecValTok{10}
\NormalTok{b =}\StringTok{ }\KeywordTok{seq}\NormalTok{(}\DataTypeTok{from =} \FloatTok{0.1}\NormalTok{, }\DataTypeTok{to =} \DecValTok{1}\NormalTok{, }\DataTypeTok{length =} \DecValTok{10}\NormalTok{)}
\NormalTok{p =}\StringTok{ }\KeywordTok{myfunction}\NormalTok{(}\DataTypeTok{x =}\NormalTok{ a, }\DataTypeTok{y =}\NormalTok{ b)  }\CommentTok{#assign output to a variable p}
\NormalTok{p}
\end{Highlighting}
\end{Shaded}

\begin{verbatim}
## [1] 5.50 0.55
\end{verbatim}

You can also use \texttt{if/else} sentences, \texttt{for/while}-loops
and \texttt{print()}.

\begin{Shaded}
\begin{Highlighting}[]
\NormalTok{lett =}\StringTok{ }\KeywordTok{c}\NormalTok{(}\StringTok{"a"}\NormalTok{, }\StringTok{"b"}\NormalTok{, }\StringTok{"c"}\NormalTok{, }\StringTok{"d"}\NormalTok{, }\StringTok{"e"}\NormalTok{, }\StringTok{"f"}\NormalTok{, }\StringTok{"g"}\NormalTok{, }\StringTok{"h"}\NormalTok{)}
\ControlFlowTok{for}\NormalTok{ (i }\ControlFlowTok{in} \DecValTok{1}\OperatorTok{:}\KeywordTok{length}\NormalTok{(lett)) \{}
    \KeywordTok{print}\NormalTok{(}\StringTok{"Now we work with:"}\NormalTok{)}
    \KeywordTok{print}\NormalTok{(lett[i])}
    \ControlFlowTok{if}\NormalTok{ (lett[i] }\OperatorTok{==}\StringTok{ "b"}\NormalTok{) \{}
        \KeywordTok{print}\NormalTok{(lett[i])}
\NormalTok{    \} }\ControlFlowTok{else}\NormalTok{ \{}
        \ControlFlowTok{if}\NormalTok{ (lett[i] }\OperatorTok{==}\StringTok{ "d"}\NormalTok{) \{}
            \KeywordTok{print}\NormalTok{(lett[i])}
\NormalTok{        \} }\ControlFlowTok{else}\NormalTok{ \{}
            \KeywordTok{print}\NormalTok{(}\StringTok{"not b or d"}\NormalTok{)}
\NormalTok{        \}}
\NormalTok{    \}}
\NormalTok{\}}
\end{Highlighting}
\end{Shaded}

While-loops can be written in a similar manner, using \texttt{while}
instad of \texttt{for}.

\section{Part F: Lists and data
frames}\label{part-f-lists-and-data-frames}

Lists and data frames are good tools for storing and accessing your
data.

\subsection{Lists}\label{lists}

Using a list, there is no restrictions to the type of data you want to
store.

\begin{Shaded}
\begin{Highlighting}[]
\NormalTok{a =}\StringTok{ }\KeywordTok{c}\NormalTok{(}\StringTok{"male"}\NormalTok{, }\StringTok{"female"}\NormalTok{, }\StringTok{"male"}\NormalTok{, }\StringTok{"male"}\NormalTok{)}
\NormalTok{b =}\StringTok{ }\KeywordTok{matrix}\NormalTok{(}\KeywordTok{c}\NormalTok{(}\DecValTok{1}\OperatorTok{:}\DecValTok{6}\NormalTok{), }\DataTypeTok{ncol =} \DecValTok{2}\NormalTok{)}
\NormalTok{c =}\StringTok{ }\KeywordTok{rnorm}\NormalTok{(}\DecValTok{100}\NormalTok{, }\DataTypeTok{mean =} \DecValTok{0}\NormalTok{, }\DataTypeTok{sd =} \DecValTok{1}\NormalTok{)}
\NormalTok{my_list =}\StringTok{ }\KeywordTok{list}\NormalTok{(}\DataTypeTok{a =}\NormalTok{ a, }\DataTypeTok{b =}\NormalTok{ b, }\DataTypeTok{c =}\NormalTok{ c)}
\end{Highlighting}
\end{Shaded}

The list \texttt{my\_list} now consists of three objects, \texttt{a,\ b}
and \texttt{c}. To access the data in you list, you write

\begin{Shaded}
\begin{Highlighting}[]
\NormalTok{my_list[[}\DecValTok{1}\NormalTok{]]  }\CommentTok{#a}
\NormalTok{my_list[[}\DecValTok{2}\NormalTok{]]  }\CommentTok{#b}
\NormalTok{my_list[[}\DecValTok{3}\NormalTok{]]  }\CommentTok{#c}
\end{Highlighting}
\end{Shaded}

or

\begin{Shaded}
\begin{Highlighting}[]
\NormalTok{my_list}\OperatorTok{$}\NormalTok{a  }\CommentTok{#a}
\NormalTok{my_list}\OperatorTok{$}\NormalTok{b  }\CommentTok{#b}
\NormalTok{my_list}\OperatorTok{$}\NormalTok{c  }\CommentTok{#c}
\end{Highlighting}
\end{Shaded}

To access the second element in the object \texttt{a}, you write
\texttt{my\_list{[}{[}1{]}{]}{[}2{]}} or \texttt{my\_list\$a{[}2{]}}.

\subsection{Data frames}\label{data-frames}

When using a data frame, you need all your elements in the data frame to
be of equal length.

\begin{Shaded}
\begin{Highlighting}[]
\NormalTok{Sick =}\StringTok{ }\KeywordTok{c}\NormalTok{(}\DecValTok{0}\NormalTok{, }\DecValTok{1}\NormalTok{, }\DecValTok{1}\NormalTok{, }\DecValTok{0}\NormalTok{, }\DecValTok{0}\NormalTok{, }\DecValTok{0}\NormalTok{, }\DecValTok{1}\NormalTok{, }\DecValTok{0}\NormalTok{)}
\NormalTok{Age =}\StringTok{ }\KeywordTok{c}\NormalTok{(}\DecValTok{50}\NormalTok{, }\DecValTok{15}\NormalTok{, }\DecValTok{39}\NormalTok{, }\DecValTok{35}\NormalTok{, }\DecValTok{26}\NormalTok{, }\DecValTok{20}\NormalTok{, }\DecValTok{10}\NormalTok{, }\DecValTok{69}\NormalTok{)}
\NormalTok{Sex =}\StringTok{ }\KeywordTok{factor}\NormalTok{(}\KeywordTok{c}\NormalTok{(}\StringTok{"male"}\NormalTok{, }\StringTok{"female"}\NormalTok{, }\StringTok{"female"}\NormalTok{, }\StringTok{"male"}\NormalTok{, }\StringTok{"male"}\NormalTok{, }\StringTok{"male"}\NormalTok{, }\StringTok{"female"}\NormalTok{, }
    \StringTok{"male"}\NormalTok{))}
\NormalTok{df =}\StringTok{ }\KeywordTok{data.frame}\NormalTok{(}\DataTypeTok{Sick =}\NormalTok{ Sick, }\DataTypeTok{Age =}\NormalTok{ Age, }\DataTypeTok{Sex =}\NormalTok{ Sex)}
\end{Highlighting}
\end{Shaded}

To access the vectors in the data frame,

\begin{Shaded}
\begin{Highlighting}[]
\NormalTok{df}\OperatorTok{$}\NormalTok{Sick}
\NormalTok{df}\OperatorTok{$}\NormalTok{Age}
\NormalTok{df}\OperatorTok{$}\NormalTok{Sex}
\end{Highlighting}
\end{Shaded}

Similar to a list, we access elements in the data frame using
\texttt{df\$Sex{[}2{]}}. If your data frame is very large, it is easier
to view typing \texttt{View(df)}.

\section{Part G: Loading and writing
files}\label{part-g-loading-and-writing-files}

There are several ways to read and write diffferent file types in R.
txt-files (tab separated) and csv-files (comma separated) are the most
common ones, and we show here how to write and read such files. Remember
to check the format of your file before reading it into R. You may also
choose Enviroment (window upper right) -Import Data set - and get help.

\subsection{csv-files (comma
separated)}\label{csv-files-comma-separated}

You can save your dataframe \texttt{df} from part F by writing it out as
a csv-file using the command \texttt{write.csv()}. Use the Help-window
and search for \texttt{write.csv} or type \texttt{?write.csv} in the
console to see which arguments to include.

\begin{Shaded}
\begin{Highlighting}[]
\KeywordTok{write.csv}\NormalTok{(df, }\DataTypeTok{file =} \StringTok{"MyFirstFile.csv"}\NormalTok{, }\DataTypeTok{row.names =} \OtherTok{FALSE}\NormalTok{)}
\end{Highlighting}
\end{Shaded}

The file is now saved in your working directory. To save it somewhere
else, you can either use
\texttt{setwd("\textasciitilde{}/your\_selected\_folder/")} before
writing the file, or include the path when saving it
\texttt{write.csv(df,file="\textasciitilde{}/your\_selected\_folder/MyFirstFile.csv")}.

To read a csv-file into R, we use a similar command \texttt{read.csv()}.
Remember to name the data when reading it, else it will not be stored in
your environment.

\begin{Shaded}
\begin{Highlighting}[]
\KeywordTok{getwd}\NormalTok{()  }\CommentTok{#path of your working directory}
\KeywordTok{list.files}\NormalTok{()  }\CommentTok{#files in the folder }
\NormalTok{myDf =}\StringTok{ }\KeywordTok{read.csv}\NormalTok{(}\DataTypeTok{file =} \StringTok{"MyFirstFile.csv"}\NormalTok{, }\DataTypeTok{header =} \OtherTok{TRUE}\NormalTok{)}
\end{Highlighting}
\end{Shaded}

To read a file from another folder, you can either use
\texttt{setwd("\textasciitilde{}/your\_selected\_folder/")} before
reading the file, or include the path when reading it
\texttt{read.csv(file="\textasciitilde{}/your\_selected\_folder/MyFirstFile.csv",\ header=TRUE)}.

\subsection{txt-files (tab separated)}\label{txt-files-tab-separated}

If you want to save your data as a txt-file, you can use the command
\texttt{write.table()}. With this function you can choose which format
you want to save the file as with the argument \texttt{sep=}. For a
txt-file, set \texttt{sep="\textbackslash{}t"} and name your file .txt.
(You can also use this for csv - \texttt{sep=","}). Use the Help-window
and search for \texttt{write.table} or type \texttt{?write.table} in the
console to see which arguments to include.

\begin{Shaded}
\begin{Highlighting}[]
\KeywordTok{write.table}\NormalTok{(df, }\DataTypeTok{file =} \StringTok{"MyFirstFile.txt"}\NormalTok{, }\DataTypeTok{sep =} \StringTok{"}\CharTok{\textbackslash{}t}\StringTok{"}\NormalTok{, }\DataTypeTok{row.names =} \OtherTok{FALSE}\NormalTok{)}
\end{Highlighting}
\end{Shaded}

To read the txt-file into R, use the command \texttt{read.table()} and
remember to set the argument \texttt{sep="\textbackslash{}t"}.

\begin{Shaded}
\begin{Highlighting}[]
\NormalTok{myDf =}\StringTok{ }\KeywordTok{read.table}\NormalTok{(}\DataTypeTok{file =} \StringTok{"MyFirstFile.txt"}\NormalTok{, }\DataTypeTok{sep =} \StringTok{"}\CharTok{\textbackslash{}t}\StringTok{"}\NormalTok{, }\DataTypeTok{header =} \OtherTok{TRUE}\NormalTok{)}
\end{Highlighting}
\end{Shaded}

\subsection{Printing to and reading from other file
formats}\label{printing-to-and-reading-from-other-file-formats}

There exists many packages to read different type of input data. A quick
google search will guide you! Reading \texttt{xls} files can be done
using the package \texttt{readxl}, but we recommend to NEVER use this
format. Always write your files as txt or csv, and if you have to read a
\texttt{xls} file, be very careful about the format of the file! It's
all too easy to mess up with Excel files.

\subsection{\texorpdfstring{Executing the commands in an R-file with
\texttt{source}}{Executing the commands in an R-file with source}}\label{executing-the-commands-in-an-r-file-with-source}

Open again the file \texttt{myRintro.R} in your source window. Then
either write:

\begin{Shaded}
\begin{Highlighting}[]
\KeywordTok{source}\NormalTok{(}\StringTok{"myRintro.R"}\NormalTok{)  }\CommentTok{#given that your working directory is wher myRintro.R is saved}
\end{Highlighting}
\end{Shaded}

or source with the source button in the upper right corner of the source
window. Alternatively -- and that's what we like in our efficient
everyday workflow -- check out the shortcut for your OS by clicking on
code -\textgreater{} source in the RStudio menu.

It is also possible to source a file from the internet, for example a
version of Part B can be sources from the TMA4268 catalog:

\begin{Shaded}
\begin{Highlighting}[]
\KeywordTok{source}\NormalTok{(}\StringTok{"https://www.math.ntnu.no/emner/TMA4268/2019v/1Intro/RintroPartB.R"}\NormalTok{, }
    \DataTypeTok{echo =} \OtherTok{TRUE}\NormalTok{)}
\end{Highlighting}
\end{Shaded}

Here \texttt{echo=TRUE} echoes the commands being run- in addition to
the results of the commands.

\subsection{Exporting plots - some
alternatives}\label{exporting-plots---some-alternatives}

We will talk more about generating random data from different
distribution in \texttt{Rintermediate.R}. However, the following
commands draws 100 realizations from the standard normal distribution
and makes a boxplot. Write these commands and run them.

\begin{Shaded}
\begin{Highlighting}[]
\NormalTok{ds =}\StringTok{ }\KeywordTok{rnorm}\NormalTok{(}\DecValTok{100}\NormalTok{)}
\KeywordTok{boxplot}\NormalTok{(ds)}
\end{Highlighting}
\end{Shaded}

\includegraphics{Rbeginner_files/figure-latex/unnamed-chunk-27-1.pdf}

Now, we want to export this plot - maybe to be put on a webpage or just
for fun (we will use R Markdown for our compulsory exercises and will
then not need to export plots).

You may save the plot (for example as pdf or svg) by pressing Export in
the Plots window, or alternatively you may write

\begin{Shaded}
\begin{Highlighting}[]
\KeywordTok{dev.copy2pdf}\NormalTok{(}\DataTypeTok{file =} \StringTok{"box.pdf"}\NormalTok{)}
\end{Highlighting}
\end{Shaded}

This will produce a pdf-file that is saved to your working directory.

A third solution

\begin{Shaded}
\begin{Highlighting}[]
\KeywordTok{pdf}\NormalTok{(}\StringTok{"box.pdf"}\NormalTok{)}
\KeywordTok{boxplot}\NormalTok{(ds)}
\KeywordTok{dev.off}\NormalTok{()}
\end{Highlighting}
\end{Shaded}

to make a file named box.pdf with the boxplot, then it is possible to
save many plots together in one pdf-file - just add more plots before
closing the pdf-file with \texttt{dev.off()}. For more on exporting
plots, take a look at
\url{http://www.sthda.com/english/wiki/creating-and-saving-graphs-r-base-graphs}.

\section{Part H: Functions and
packages}\label{part-h-functions-and-packages}

We have already seen how to make functions in Part E and we have used
some existing functions like \texttt{boxplot()}, \texttt{sample()} and
\texttt{rnorm()}. R is a free and open source program where everyone can
contribute with making functions, and there exists \emph{a lot}.
Functions are available through packages (collection of functions and/or
data) that can be installed and loaded into R. Some are already included
in the default R session, like the package \texttt{stats} that includes
many basic functions for doing statistics. Making your own functions is
an important part of programming and statistical analysis, but using
existing functions and packages often makes life easier.

Every statistical researcher who would like to to get their new
statistitical methods used will make an R package and distribute it with
their article (on the new method). Most books also come with R packages
with data sets and functions. Our ISL book has the package
\texttt{ISLR}, hosted on the most widely used service for R packages:
CRAN. See the official page for the package here:
\url{https://cran.r-project.org/web/packages/ISLR/index.html}.

To install an R package from CRAN you go to the Packages tab and see if
the packages is already available on your computer. If you see ISLR in
this list just press the square next to ISLR to load the package into R.

If you don't see ISLR you will have to download it from CRAN. Do this by
either pressing Install on the top left corner of the Packages window,
CRAN is already filled as ``Install from'' and then write ``ISLR'' as
the name of the package to install, and nice to have chosen ``install
dependencies'' (then packages that ISLR depend on will also be
installed). Then press ``Install''. You might have to select from
different mirrors for CRAN - choose Norway, and you are good to go. Then
``ISLR'' should pop up in the list of packages installed, and you tick
(in the square) to load the package into R.

Alternatively -- and the easier way that we recommend -- you write in
the console (or source) window:

\begin{Shaded}
\begin{Highlighting}[]
\KeywordTok{install.packages}\NormalTok{(}\StringTok{"ISLR"}\NormalTok{)}
\KeywordTok{library}\NormalTok{(ISLR)  }\CommentTok{# to make the package available in the current session}
\end{Highlighting}
\end{Shaded}

Now the packages is installed and loaded to the current session.
Remember that whenever starting a new session, you need to reload the
packages you want to use, using the \texttt{library()} function, or
ticking the square next to ISLR in the Packages window. You don't need
to install it again, it is already on your computer.

\begin{Shaded}
\begin{Highlighting}[]
\KeywordTok{library}\NormalTok{(}\DataTypeTok{help =} \StringTok{"ISLR"}\NormalTok{)}
\end{Highlighting}
\end{Shaded}

\textbf{Q}: Look at the contents of the \texttt{ISLR} to see that only
data sets are available - you may also see that by selecting the name
ISLR in the Packages window. To know more about the data set named
\texttt{NCI60} either just select the data set in the Packages window,
or write \texttt{help(NCI60)} or \texttt{?NCI60} after ISLR is loaded.
What can you say about \texttt{NCI60}?

Another package that we will use is \texttt{car}.

\textbf{Q}: Install the \texttt{car} package from CRAN, check the
content of the package (data sets and functions) and investigate

We will be using a lot of packages in this course, and by using the .Rmd
version of our course material you can see which packages we load and
use. We would assume that you have installed these packages if you want
to reproduce that statistical analyses on the module pages.

Before start using the functions of the package, it is often a good idea
to visit the help pages of the package to see which functions and data
sets are available, how they are used, what they calculate, and the
output they give, etc.. These pages are found in the Help window to the
left or typing \texttt{?name} in the console, (ex. \texttt{?mean}).

In the stats package, you find functions for making and evaluating
distributions. We use the function \texttt{rnorm} to sample independent
data from the univariate normal distribution.

\begin{Shaded}
\begin{Highlighting}[]
\NormalTok{rnorm  }\CommentTok{#lists the function code}
\StringTok{`}\DataTypeTok{?}\StringTok{`}\NormalTok{(rnorm  }\CommentTok{#help pages for the function}
\NormalTok{)}
\KeywordTok{rnorm}\NormalTok{()  }\CommentTok{#gives error}
\KeywordTok{rnorm}\NormalTok{(}\DataTypeTok{n =} \DecValTok{100}\NormalTok{, }\DataTypeTok{mean =} \DecValTok{0}\NormalTok{, }\DataTypeTok{sd =} \DecValTok{1}\NormalTok{)  }\CommentTok{#draw random samples from this distribution}
\StringTok{`}\DataTypeTok{?}\StringTok{`}\NormalTok{(lm  }\CommentTok{# more to see, will be what we use to perform linear regression}
\NormalTok{)}
\end{Highlighting}
\end{Shaded}

\section{What is next?}\label{what-is-next}

You may now move to \texttt{Rintermediate} on our
\href{https://wiki.math.ntnu.no/tma4268/2020v/subpage1}{course page} to
see how R can be used on topics that should already be familiar to you
from TMA4240/TMA4245 Statistics - or similar courses. Or, if you did
ST1201 you can look at an overview of how the methods in ST1201 can be
performed in R:
\href{https://www.math.ntnu.no/emner/TMA4268/2019v/1Intro/ST1201inR.html}{ST1201inR.html}

\section{Acknowledgements}\label{acknowledgements}

We thank Mette Langaas and her PhD students from 2018 and 2019 for
building up the original version of this sheet.


\end{document}
