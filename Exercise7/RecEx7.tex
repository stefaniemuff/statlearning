\documentclass[]{article}
\usepackage{lmodern}
\usepackage{amssymb,amsmath}
\usepackage{ifxetex,ifluatex}
\usepackage{fixltx2e} % provides \textsubscript
\ifnum 0\ifxetex 1\fi\ifluatex 1\fi=0 % if pdftex
  \usepackage[T1]{fontenc}
  \usepackage[utf8]{inputenc}
\else % if luatex or xelatex
  \ifxetex
    \usepackage{mathspec}
  \else
    \usepackage{fontspec}
  \fi
  \defaultfontfeatures{Ligatures=TeX,Scale=MatchLowercase}
\fi
% use upquote if available, for straight quotes in verbatim environments
\IfFileExists{upquote.sty}{\usepackage{upquote}}{}
% use microtype if available
\IfFileExists{microtype.sty}{%
\usepackage{microtype}
\UseMicrotypeSet[protrusion]{basicmath} % disable protrusion for tt fonts
}{}
\usepackage[margin=1in]{geometry}
\usepackage{hyperref}
\PassOptionsToPackage{usenames,dvipsnames}{color} % color is loaded by hyperref
\hypersetup{unicode=true,
            pdftitle={Module 7: Recommended Exercises},
            pdfauthor={Andreas Strand, Martina Hall, Michail Spitieris, Stefanie Muff, Department of Mathematical Sciences, NTNU},
            colorlinks=true,
            linkcolor=Maroon,
            citecolor=Blue,
            urlcolor=blue,
            breaklinks=true}
\urlstyle{same}  % don't use monospace font for urls
\usepackage{color}
\usepackage{fancyvrb}
\newcommand{\VerbBar}{|}
\newcommand{\VERB}{\Verb[commandchars=\\\{\}]}
\DefineVerbatimEnvironment{Highlighting}{Verbatim}{commandchars=\\\{\}}
% Add ',fontsize=\small' for more characters per line
\usepackage{framed}
\definecolor{shadecolor}{RGB}{248,248,248}
\newenvironment{Shaded}{\begin{snugshade}}{\end{snugshade}}
\newcommand{\KeywordTok}[1]{\textcolor[rgb]{0.13,0.29,0.53}{\textbf{#1}}}
\newcommand{\DataTypeTok}[1]{\textcolor[rgb]{0.13,0.29,0.53}{#1}}
\newcommand{\DecValTok}[1]{\textcolor[rgb]{0.00,0.00,0.81}{#1}}
\newcommand{\BaseNTok}[1]{\textcolor[rgb]{0.00,0.00,0.81}{#1}}
\newcommand{\FloatTok}[1]{\textcolor[rgb]{0.00,0.00,0.81}{#1}}
\newcommand{\ConstantTok}[1]{\textcolor[rgb]{0.00,0.00,0.00}{#1}}
\newcommand{\CharTok}[1]{\textcolor[rgb]{0.31,0.60,0.02}{#1}}
\newcommand{\SpecialCharTok}[1]{\textcolor[rgb]{0.00,0.00,0.00}{#1}}
\newcommand{\StringTok}[1]{\textcolor[rgb]{0.31,0.60,0.02}{#1}}
\newcommand{\VerbatimStringTok}[1]{\textcolor[rgb]{0.31,0.60,0.02}{#1}}
\newcommand{\SpecialStringTok}[1]{\textcolor[rgb]{0.31,0.60,0.02}{#1}}
\newcommand{\ImportTok}[1]{#1}
\newcommand{\CommentTok}[1]{\textcolor[rgb]{0.56,0.35,0.01}{\textit{#1}}}
\newcommand{\DocumentationTok}[1]{\textcolor[rgb]{0.56,0.35,0.01}{\textbf{\textit{#1}}}}
\newcommand{\AnnotationTok}[1]{\textcolor[rgb]{0.56,0.35,0.01}{\textbf{\textit{#1}}}}
\newcommand{\CommentVarTok}[1]{\textcolor[rgb]{0.56,0.35,0.01}{\textbf{\textit{#1}}}}
\newcommand{\OtherTok}[1]{\textcolor[rgb]{0.56,0.35,0.01}{#1}}
\newcommand{\FunctionTok}[1]{\textcolor[rgb]{0.00,0.00,0.00}{#1}}
\newcommand{\VariableTok}[1]{\textcolor[rgb]{0.00,0.00,0.00}{#1}}
\newcommand{\ControlFlowTok}[1]{\textcolor[rgb]{0.13,0.29,0.53}{\textbf{#1}}}
\newcommand{\OperatorTok}[1]{\textcolor[rgb]{0.81,0.36,0.00}{\textbf{#1}}}
\newcommand{\BuiltInTok}[1]{#1}
\newcommand{\ExtensionTok}[1]{#1}
\newcommand{\PreprocessorTok}[1]{\textcolor[rgb]{0.56,0.35,0.01}{\textit{#1}}}
\newcommand{\AttributeTok}[1]{\textcolor[rgb]{0.77,0.63,0.00}{#1}}
\newcommand{\RegionMarkerTok}[1]{#1}
\newcommand{\InformationTok}[1]{\textcolor[rgb]{0.56,0.35,0.01}{\textbf{\textit{#1}}}}
\newcommand{\WarningTok}[1]{\textcolor[rgb]{0.56,0.35,0.01}{\textbf{\textit{#1}}}}
\newcommand{\AlertTok}[1]{\textcolor[rgb]{0.94,0.16,0.16}{#1}}
\newcommand{\ErrorTok}[1]{\textcolor[rgb]{0.64,0.00,0.00}{\textbf{#1}}}
\newcommand{\NormalTok}[1]{#1}
\usepackage{graphicx,grffile}
\makeatletter
\def\maxwidth{\ifdim\Gin@nat@width>\linewidth\linewidth\else\Gin@nat@width\fi}
\def\maxheight{\ifdim\Gin@nat@height>\textheight\textheight\else\Gin@nat@height\fi}
\makeatother
% Scale images if necessary, so that they will not overflow the page
% margins by default, and it is still possible to overwrite the defaults
% using explicit options in \includegraphics[width, height, ...]{}
\setkeys{Gin}{width=\maxwidth,height=\maxheight,keepaspectratio}
\IfFileExists{parskip.sty}{%
\usepackage{parskip}
}{% else
\setlength{\parindent}{0pt}
\setlength{\parskip}{6pt plus 2pt minus 1pt}
}
\setlength{\emergencystretch}{3em}  % prevent overfull lines
\providecommand{\tightlist}{%
  \setlength{\itemsep}{0pt}\setlength{\parskip}{0pt}}
\setcounter{secnumdepth}{0}
% Redefines (sub)paragraphs to behave more like sections
\ifx\paragraph\undefined\else
\let\oldparagraph\paragraph
\renewcommand{\paragraph}[1]{\oldparagraph{#1}\mbox{}}
\fi
\ifx\subparagraph\undefined\else
\let\oldsubparagraph\subparagraph
\renewcommand{\subparagraph}[1]{\oldsubparagraph{#1}\mbox{}}
\fi

%%% Use protect on footnotes to avoid problems with footnotes in titles
\let\rmarkdownfootnote\footnote%
\def\footnote{\protect\rmarkdownfootnote}

%%% Change title format to be more compact
\usepackage{titling}

% Create subtitle command for use in maketitle
\providecommand{\subtitle}[1]{
  \posttitle{
    \begin{center}\large#1\end{center}
    }
}

\setlength{\droptitle}{-2em}

  \title{Module 7: Recommended Exercises}
    \pretitle{\vspace{\droptitle}\centering\huge}
  \posttitle{\par}
  \subtitle{TMA4268 Statistical Learning V2020}
  \author{Andreas Strand, Martina Hall, Michail Spitieris, Stefanie Muff,
Department of Mathematical Sciences, NTNU}
    \preauthor{\centering\large\emph}
  \postauthor{\par}
      \predate{\centering\large\emph}
  \postdate{\par}
    \date{February 27, 2020}


\begin{document}
\maketitle

\subsection{Problem 1}\label{problem-1}

Let us take a look at the \texttt{Auto} data set. We want to model miles
per gallon \texttt{mpg} by engine horsepower \texttt{horsepower}.
Separate the observations into training and test. A training set is
plotted below.

Perform polynomial regression of degree 1, 2, 3 and 4. Use
\texttt{lines()} to add the fitted values to the plot below.

Also plot the test error by polynomial degree.

\begin{Shaded}
\begin{Highlighting}[]
\KeywordTok{library}\NormalTok{(ISLR)}
\CommentTok{# extract only the two variables from Auto}
\NormalTok{ds =}\StringTok{ }\NormalTok{Auto[}\KeywordTok{c}\NormalTok{(}\StringTok{"horsepower"}\NormalTok{, }\StringTok{"mpg"}\NormalTok{)]}
\NormalTok{n =}\StringTok{ }\KeywordTok{nrow}\NormalTok{(ds)}
\CommentTok{# which degrees we will look at}
\NormalTok{deg =}\StringTok{ }\DecValTok{1}\OperatorTok{:}\DecValTok{4}
\KeywordTok{set.seed}\NormalTok{(}\DecValTok{1}\NormalTok{)}
\CommentTok{# training ids for training set}
\NormalTok{tr =}\StringTok{ }\KeywordTok{sample.int}\NormalTok{(n, n}\OperatorTok{/}\DecValTok{2}\NormalTok{)}
\CommentTok{# plot of training data}
\KeywordTok{plot}\NormalTok{(ds[tr, ], }\DataTypeTok{col =} \StringTok{"darkgrey"}\NormalTok{, }\DataTypeTok{main =} \StringTok{"Polynomial regression"}\NormalTok{)}
\end{Highlighting}
\end{Shaded}

\includegraphics{RecEx7_files/figure-latex/unnamed-chunk-1-1.pdf}

\subsection{Problem 2}\label{problem-2}

We will continue working with the \texttt{Auto} data set. The variable
\texttt{origin} is 1,2 or 3, corresponding to American, European or
Japanese origin. Use \texttt{factor(origin)} for conversion to a factor
variable. Predict \texttt{mpg} by origin with a linear model. Plot the
fitted values and approximative 95 percent confidence intervals.
Selecting \texttt{se\ =\ T} in \texttt{predict()} gives standard errors
of the prediction.

\begin{itemize}
\item
  Hint: make a new dataframe of the three origins (as factors) and use
  this new data in your predict function.
\item
  Hint: to plot the confidence intervals, you can add
  \texttt{geom\_segment(aes(x=origin,\ y=lwr,\ xend\ =\ origin,\ yend=upr))}
  to your ggplot, where \texttt{origin}, \texttt{lwr} and \texttt{upr}
  comes from a dataframe with \texttt{lwr} as the lower bound and
  \texttt{upr} as the upper bound.
\end{itemize}

\subsection{Problem 3}\label{problem-3}

Now, let us look at the \texttt{Wage} data set. The section on
\href{https://htmlpreview.github.io/?https://github.com/stefaniemuff/statlearning/blob/master/7BeyondLinear/7BeyondLinear.html}{Additive
Models} explains how we can create an AM by adding components together.
One component we saw is a natural spline in \texttt{year} with one knot.
Derive the expression for the design matrix \(\mathbf X_2\) from the
natural spline basis: \[
b_1(x_i) = x_i, \quad b_{k+2}(x_i) = d_k(x_i)-d_K(x_i),\; k = 0, \ldots, K - 1,\\
\] \[
d_k(x_i) = \frac{(x_i-c_k)^3_+-(x_i-c_{K+1})^3_+}{c_{K+1}-c_k}.
\]

\subsection{Problem 4}\label{problem-4}

We will continue working with the same AM as in problem 3. The R call
\texttt{model.matrix(\textasciitilde{}\ bs(age,knots=c(40,60))\ +\ ns(year,knots=2006)\ +\ education)}
gives a design matrix for the AM. This matrix is what \texttt{gam()}
uses. However, it does not equal our design matrix for AM
\(\mathbf X = (\mathbf{1}, \mathbf{X}_1, \mathbf{X}_2, \mathbf{X}_3)\).
The predicted responses will still be the same.

Write code that produces \(\mathbf X\). The code below may be useful.

\begin{Shaded}
\begin{Highlighting}[]
\CommentTok{# X_1}
\NormalTok{mybs =}\StringTok{ }\ControlFlowTok{function}\NormalTok{(x, knots) \{}
    \KeywordTok{cbind}\NormalTok{(x, x}\OperatorTok{^}\DecValTok{2}\NormalTok{, x}\OperatorTok{^}\DecValTok{3}\NormalTok{, }\KeywordTok{sapply}\NormalTok{(knots, }\ControlFlowTok{function}\NormalTok{(y) }\KeywordTok{pmax}\NormalTok{(}\DecValTok{0}\NormalTok{, x }\OperatorTok{-}\StringTok{ }\NormalTok{y)}\OperatorTok{^}\DecValTok{3}\NormalTok{))}
\NormalTok{\}}

\NormalTok{d =}\StringTok{ }\ControlFlowTok{function}\NormalTok{(c, cK, x) (}\KeywordTok{pmax}\NormalTok{(}\DecValTok{0}\NormalTok{, x }\OperatorTok{-}\StringTok{ }\NormalTok{c)}\OperatorTok{^}\DecValTok{3} \OperatorTok{-}\StringTok{ }\KeywordTok{pmax}\NormalTok{(}\DecValTok{0}\NormalTok{, x }\OperatorTok{-}\StringTok{ }\NormalTok{cK)}\OperatorTok{^}\DecValTok{3}\NormalTok{)}\OperatorTok{/}\NormalTok{(cK }\OperatorTok{-}\StringTok{ }\NormalTok{c)}
\CommentTok{# X_2}
\NormalTok{myns =}\StringTok{ }\ControlFlowTok{function}\NormalTok{(x, knots) \{}
\NormalTok{    kn =}\StringTok{ }\KeywordTok{c}\NormalTok{(}\KeywordTok{min}\NormalTok{(x), knots, }\KeywordTok{max}\NormalTok{(x))}
\NormalTok{    K =}\StringTok{ }\KeywordTok{length}\NormalTok{(kn)}
\NormalTok{    sub =}\StringTok{ }\KeywordTok{d}\NormalTok{(kn[K }\OperatorTok{-}\StringTok{ }\DecValTok{1}\NormalTok{], kn[K], x)}
    \KeywordTok{cbind}\NormalTok{(x, }\KeywordTok{sapply}\NormalTok{(kn[}\DecValTok{1}\OperatorTok{:}\NormalTok{(K }\OperatorTok{-}\StringTok{ }\DecValTok{2}\NormalTok{)], d, kn[K], x) }\OperatorTok{-}\StringTok{ }\NormalTok{sub)}
\NormalTok{\}}
\CommentTok{# X_3}
\NormalTok{myfactor =}\StringTok{ }\ControlFlowTok{function}\NormalTok{(x) }\KeywordTok{model.matrix}\NormalTok{(}\OperatorTok{~}\NormalTok{x)[, }\OperatorTok{-}\DecValTok{1}\NormalTok{]}
\end{Highlighting}
\end{Shaded}

If the code is valid, the predicted response
\(\hat{\mathbf y} = \mathbf X(\mathbf X^\mathsf{T} \mathbf X)^{-1} \mathbf X^\mathsf{T} \mathbf y\)
should be the same as when using the built-in R function.

\begin{Shaded}
\begin{Highlighting}[]
\CommentTok{# install.packages('gam'')}
\KeywordTok{library}\NormalTok{(gam)}
\KeywordTok{library}\NormalTok{(ISLR)}
\CommentTok{# your X-matrix}
\NormalTok{X =}\StringTok{ }\CommentTok{# fitted model with our X}
\NormalTok{myhat =}\StringTok{ }\KeywordTok{lm}\NormalTok{(wage }\OperatorTok{~}\StringTok{ }\NormalTok{X }\OperatorTok{-}\StringTok{ }\DecValTok{1}\NormalTok{)}\OperatorTok{$}\NormalTok{fit}
\CommentTok{# fitted model with gam}
\NormalTok{yhat =}\StringTok{ }\KeywordTok{gam}\NormalTok{(wage }\OperatorTok{~}\StringTok{ }\KeywordTok{bs}\NormalTok{(age, }\DataTypeTok{knots =} \KeywordTok{c}\NormalTok{(}\DecValTok{40}\NormalTok{, }\DecValTok{60}\NormalTok{)) }\OperatorTok{+}\StringTok{ }\KeywordTok{ns}\NormalTok{(year, }\DataTypeTok{knots =} \DecValTok{2006}\NormalTok{) }\OperatorTok{+}\StringTok{ }\NormalTok{education)}\OperatorTok{$}\NormalTok{fit}
\CommentTok{# are they equal?}
\KeywordTok{all.equal}\NormalTok{(myhat, yhat)}
\end{Highlighting}
\end{Shaded}

How can \texttt{myhat} equal \texttt{yhat} when the design matrices
differ?

\subsection{Problem 5}\label{problem-5}

In this exercise we take a quick look at different non-linear regression
methods. We continue using the Auto dataset from above, but with more
variables.

Fit an additive model using the function \texttt{gam} from package
\texttt{gam}. Call the result \texttt{gamobject}.

\begin{itemize}
\tightlist
\item
  \texttt{mpg} is the response,
\item
  \texttt{displace} is a cubic spline (hint: \texttt{bs}) with one knot
  at 290,
\item
  \texttt{horsepower} is a polynomial of degree 2 (hint: \texttt{poly}),
\item
  \texttt{weight} is a linear function,
\item
  \texttt{acceleration} is a smoothing spline with \texttt{df=3} (hint:
  \texttt{s}),
\item
  \texttt{origin} is a step function (what we previously have called
  dummy variable coding).
\end{itemize}

Plot the resulting curves (hint: first set \texttt{par(mfrow=c(2,3)} and
then \texttt{plot(gamobject,se=TRUE,col="blue")}). Comment on what you
see.

\subsection{Problem 6 (Advanced)}\label{problem-6-advanced}

Back to Wage-data. In the part where we discussed smoothing splines
there is a section explaining how to compute \(\mathbf S\), where
\(\hat{\mathbf y} = \mathbf S \mathbf y\). This is implemented below,
with \(\mathbf x\) as unique observations of \texttt{age} and
\(\mathbf y\) the coresponding \texttt{wage}.

\begin{Shaded}
\begin{Highlighting}[]
\NormalTok{K =}\StringTok{ }\ControlFlowTok{function}\NormalTok{(x) \{}
\NormalTok{    xi =}\StringTok{ }\KeywordTok{sort}\NormalTok{(}\KeywordTok{unique}\NormalTok{(x))}
\NormalTok{    n =}\StringTok{ }\KeywordTok{length}\NormalTok{(xi)}
\NormalTok{    h =}\StringTok{ }\NormalTok{xi[}\OperatorTok{-}\DecValTok{1}\NormalTok{] }\OperatorTok{-}\StringTok{ }\NormalTok{xi[}\OperatorTok{-}\NormalTok{n]}
\NormalTok{    i =}\StringTok{ }\KeywordTok{seq.int}\NormalTok{(n }\OperatorTok{-}\StringTok{ }\DecValTok{2}\NormalTok{)}
\NormalTok{    D =}\StringTok{ }\KeywordTok{diag}\NormalTok{(}\DecValTok{1}\OperatorTok{/}\NormalTok{h[i], }\DataTypeTok{ncol =}\NormalTok{ n)}
\NormalTok{    D[}\KeywordTok{cbind}\NormalTok{(i, i }\OperatorTok{+}\StringTok{ }\DecValTok{1}\NormalTok{)] =}\StringTok{ }\OperatorTok{-}\DecValTok{1}\OperatorTok{/}\NormalTok{h[i] }\OperatorTok{-}\StringTok{ }\DecValTok{1}\OperatorTok{/}\NormalTok{h[i }\OperatorTok{+}\StringTok{ }\DecValTok{1}\NormalTok{]}
\NormalTok{    D[}\KeywordTok{cbind}\NormalTok{(i, i }\OperatorTok{+}\StringTok{ }\DecValTok{2}\NormalTok{)] =}\StringTok{ }\DecValTok{1}\OperatorTok{/}\NormalTok{h[i }\OperatorTok{+}\StringTok{ }\DecValTok{1}\NormalTok{]}
\NormalTok{    W =}\StringTok{ }\KeywordTok{diag}\NormalTok{(h[i] }\OperatorTok{+}\StringTok{ }\NormalTok{h[i }\OperatorTok{+}\StringTok{ }\DecValTok{1}\NormalTok{]}\OperatorTok{/}\DecValTok{3}\NormalTok{)}
\NormalTok{    W[}\KeywordTok{cbind}\NormalTok{(i[}\OperatorTok{-}\DecValTok{1}\NormalTok{], i[}\OperatorTok{-}\DecValTok{1}\NormalTok{] }\OperatorTok{-}\StringTok{ }\DecValTok{1}\NormalTok{)] =}\StringTok{ }\NormalTok{h[i[}\OperatorTok{-}\DecValTok{1}\NormalTok{]]}\OperatorTok{/}\DecValTok{6}
\NormalTok{    W[}\KeywordTok{cbind}\NormalTok{(i[}\OperatorTok{-}\DecValTok{1}\NormalTok{] }\OperatorTok{-}\StringTok{ }\DecValTok{1}\NormalTok{, i[}\OperatorTok{-}\DecValTok{1}\NormalTok{])] =}\StringTok{ }\NormalTok{h[i[}\OperatorTok{-}\DecValTok{1}\NormalTok{]]}\OperatorTok{/}\DecValTok{6}
    \KeywordTok{t}\NormalTok{(D) }\OperatorTok\StringTok{ }\KeywordTok{solve}\NormalTok{(W) }\OperatorTok\StringTok{ }\NormalTok{D}
\NormalTok{\}}

\NormalTok{x =}\StringTok{ }\KeywordTok{sort}\NormalTok{(}\KeywordTok{unique}\NormalTok{(age))}
\NormalTok{y =}\StringTok{ }\NormalTok{wage[}\KeywordTok{order}\NormalTok{(age[}\OperatorTok{!}\KeywordTok{duplicated}\NormalTok{(age)])]}
\NormalTok{eig =}\StringTok{ }\KeywordTok{eigen}\NormalTok{(}\KeywordTok{K}\NormalTok{(x))}
\NormalTok{U =}\StringTok{ }\NormalTok{eig}\OperatorTok{$}\NormalTok{vectors}
\NormalTok{d =}\StringTok{ }\NormalTok{eig}\OperatorTok{$}\NormalTok{values}
\NormalTok{lambda =}\StringTok{ }\DecValTok{2000}
\NormalTok{S =}\StringTok{ }\NormalTok{U }\OperatorTok\StringTok{ }\KeywordTok{diag}\NormalTok{(}\DecValTok{1}\OperatorTok{/}\NormalTok{(}\DecValTok{1} \OperatorTok{+}\StringTok{ }\NormalTok{lambda }\OperatorTok{*}\StringTok{ }\NormalTok{d)) }\OperatorTok\StringTok{ }\KeywordTok{t}\NormalTok{(U)}
\end{Highlighting}
\end{Shaded}

Use \(\mathbf S\) from this code to compute \(\hat{\mathbf y}\). Also
compute \(\hat{\mathbf y}\) using \texttt{smooth.spline()} with the
correct degrees of freedom. Finally, plot both sets of fitted values and
observe that they are equal.

\begin{Shaded}
\begin{Highlighting}[]
\NormalTok{myhat =}\StringTok{ }\NormalTok{yhat =}\StringTok{ }\KeywordTok{smooth.spline}\NormalTok{(x, y, }\DataTypeTok{df =}\NormalTok{ )}\OperatorTok{$}\NormalTok{y}

\KeywordTok{plot}\NormalTok{(x, y, }\DataTypeTok{main =} \StringTok{"Comparison of fitted values"}\NormalTok{)}
\NormalTok{co =}\StringTok{ }\KeywordTok{c}\NormalTok{(}\StringTok{"blue"}\NormalTok{, }\StringTok{"red"}\NormalTok{)}
\NormalTok{w =}\StringTok{ }\KeywordTok{c}\NormalTok{(}\DecValTok{5}\NormalTok{, }\DecValTok{2}\NormalTok{)}
\CommentTok{# add lines of fitted values to the plot}
\KeywordTok{lines}\NormalTok{(x, myhat, }\DataTypeTok{lwd =}\NormalTok{ w[}\DecValTok{1}\NormalTok{], }\DataTypeTok{col =}\NormalTok{ co[}\DecValTok{1}\NormalTok{])  }\CommentTok{#lwd = linewidth}
\KeywordTok{lines}\NormalTok{(x, yhat, }\DataTypeTok{lwd =}\NormalTok{ w[}\DecValTok{2}\NormalTok{], }\DataTypeTok{col =}\NormalTok{ co[}\DecValTok{2}\NormalTok{])}
\KeywordTok{legend}\NormalTok{(}\StringTok{"topright"}\NormalTok{, }\DataTypeTok{legend =} \KeywordTok{c}\NormalTok{(}\StringTok{"myhat"}\NormalTok{, }\StringTok{"yhat"}\NormalTok{), }\DataTypeTok{col =}\NormalTok{ co, }\DataTypeTok{lwd =}\NormalTok{ w, }\DataTypeTok{inset =} \KeywordTok{c}\NormalTok{(}\DecValTok{0}\NormalTok{, }
    \OperatorTok{-}\FloatTok{0.19}\NormalTok{), }\DataTypeTok{xpd =}\NormalTok{ T)}
\end{Highlighting}
\end{Shaded}


\end{document}
