\documentclass[10pt,ignorenonframetext,]{beamer}
\setbeamertemplate{caption}[numbered]
\setbeamertemplate{caption label separator}{: }
\setbeamercolor{caption name}{fg=normal text.fg}
\beamertemplatenavigationsymbolsempty
\usepackage{lmodern}
\usepackage{amssymb,amsmath}
\usepackage{ifxetex,ifluatex}
\usepackage{fixltx2e} % provides \textsubscript
\ifnum 0\ifxetex 1\fi\ifluatex 1\fi=0 % if pdftex
  \usepackage[T1]{fontenc}
  \usepackage[utf8]{inputenc}
\else % if luatex or xelatex
  \ifxetex
    \usepackage{mathspec}
  \else
    \usepackage{fontspec}
  \fi
  \defaultfontfeatures{Ligatures=TeX,Scale=MatchLowercase}
\fi
\usetheme[]{Singapore}
\usefonttheme{serif}
% use upquote if available, for straight quotes in verbatim environments
\IfFileExists{upquote.sty}{\usepackage{upquote}}{}
% use microtype if available
\IfFileExists{microtype.sty}{%
\usepackage{microtype}
\UseMicrotypeSet[protrusion]{basicmath} % disable protrusion for tt fonts
}{}
\newif\ifbibliography
\hypersetup{
            pdftitle={Module 11: Neural Networks},
            pdfauthor={Stefanie Muff, Department of Mathematical Sciences, NTNU},
            colorlinks=true,
            linkcolor=Maroon,
            citecolor=Blue,
            urlcolor=blue,
            breaklinks=true}
\urlstyle{same}  % don't use monospace font for urls
\usepackage{color}
\usepackage{fancyvrb}
\newcommand{\VerbBar}{|}
\newcommand{\VERB}{\Verb[commandchars=\\\{\}]}
\DefineVerbatimEnvironment{Highlighting}{Verbatim}{commandchars=\\\{\}}
% Add ',fontsize=\small' for more characters per line
\usepackage{framed}
\definecolor{shadecolor}{RGB}{248,248,248}
\newenvironment{Shaded}{\begin{snugshade}}{\end{snugshade}}
\newcommand{\KeywordTok}[1]{\textcolor[rgb]{0.13,0.29,0.53}{\textbf{#1}}}
\newcommand{\DataTypeTok}[1]{\textcolor[rgb]{0.13,0.29,0.53}{#1}}
\newcommand{\DecValTok}[1]{\textcolor[rgb]{0.00,0.00,0.81}{#1}}
\newcommand{\BaseNTok}[1]{\textcolor[rgb]{0.00,0.00,0.81}{#1}}
\newcommand{\FloatTok}[1]{\textcolor[rgb]{0.00,0.00,0.81}{#1}}
\newcommand{\ConstantTok}[1]{\textcolor[rgb]{0.00,0.00,0.00}{#1}}
\newcommand{\CharTok}[1]{\textcolor[rgb]{0.31,0.60,0.02}{#1}}
\newcommand{\SpecialCharTok}[1]{\textcolor[rgb]{0.00,0.00,0.00}{#1}}
\newcommand{\StringTok}[1]{\textcolor[rgb]{0.31,0.60,0.02}{#1}}
\newcommand{\VerbatimStringTok}[1]{\textcolor[rgb]{0.31,0.60,0.02}{#1}}
\newcommand{\SpecialStringTok}[1]{\textcolor[rgb]{0.31,0.60,0.02}{#1}}
\newcommand{\ImportTok}[1]{#1}
\newcommand{\CommentTok}[1]{\textcolor[rgb]{0.56,0.35,0.01}{\textit{#1}}}
\newcommand{\DocumentationTok}[1]{\textcolor[rgb]{0.56,0.35,0.01}{\textbf{\textit{#1}}}}
\newcommand{\AnnotationTok}[1]{\textcolor[rgb]{0.56,0.35,0.01}{\textbf{\textit{#1}}}}
\newcommand{\CommentVarTok}[1]{\textcolor[rgb]{0.56,0.35,0.01}{\textbf{\textit{#1}}}}
\newcommand{\OtherTok}[1]{\textcolor[rgb]{0.56,0.35,0.01}{#1}}
\newcommand{\FunctionTok}[1]{\textcolor[rgb]{0.00,0.00,0.00}{#1}}
\newcommand{\VariableTok}[1]{\textcolor[rgb]{0.00,0.00,0.00}{#1}}
\newcommand{\ControlFlowTok}[1]{\textcolor[rgb]{0.13,0.29,0.53}{\textbf{#1}}}
\newcommand{\OperatorTok}[1]{\textcolor[rgb]{0.81,0.36,0.00}{\textbf{#1}}}
\newcommand{\BuiltInTok}[1]{#1}
\newcommand{\ExtensionTok}[1]{#1}
\newcommand{\PreprocessorTok}[1]{\textcolor[rgb]{0.56,0.35,0.01}{\textit{#1}}}
\newcommand{\AttributeTok}[1]{\textcolor[rgb]{0.77,0.63,0.00}{#1}}
\newcommand{\RegionMarkerTok}[1]{#1}
\newcommand{\InformationTok}[1]{\textcolor[rgb]{0.56,0.35,0.01}{\textbf{\textit{#1}}}}
\newcommand{\WarningTok}[1]{\textcolor[rgb]{0.56,0.35,0.01}{\textbf{\textit{#1}}}}
\newcommand{\AlertTok}[1]{\textcolor[rgb]{0.94,0.16,0.16}{#1}}
\newcommand{\ErrorTok}[1]{\textcolor[rgb]{0.64,0.00,0.00}{\textbf{#1}}}
\newcommand{\NormalTok}[1]{#1}
\usepackage{longtable,booktabs}
\usepackage{caption}
% These lines are needed to make table captions work with longtable:
\makeatletter
\def\fnum@table{\tablename~\thetable}
\makeatother
\usepackage{graphicx,grffile}
\makeatletter
\def\maxwidth{\ifdim\Gin@nat@width>\linewidth\linewidth\else\Gin@nat@width\fi}
\def\maxheight{\ifdim\Gin@nat@height>\textheight0.8\textheight\else\Gin@nat@height\fi}
\makeatother
% Scale images if necessary, so that they will not overflow the page
% margins by default, and it is still possible to overwrite the defaults
% using explicit options in \includegraphics[width, height, ...]{}
\setkeys{Gin}{width=\maxwidth,height=\maxheight,keepaspectratio}

% Prevent slide breaks in the middle of a paragraph:
\widowpenalties 1 10000
\raggedbottom

\AtBeginPart{
  \let\insertpartnumber\relax
  \let\partname\relax
  \frame{\partpage}
}
\AtBeginSection{
  \ifbibliography
  \else
    \let\insertsectionnumber\relax
    \let\sectionname\relax
    \frame{\sectionpage}
  \fi
}
\AtBeginSubsection{
  \let\insertsubsectionnumber\relax
  \let\subsectionname\relax
  \frame{\subsectionpage}
}

\setlength{\parindent}{0pt}
\setlength{\parskip}{6pt plus 2pt minus 1pt}
\setlength{\emergencystretch}{3em}  % prevent overfull lines
\providecommand{\tightlist}{%
  \setlength{\itemsep}{0pt}\setlength{\parskip}{0pt}}
\setcounter{secnumdepth}{0}
\usepackage{multirow}

\title{Module 11: Neural Networks}
\subtitle{TMA4268 Statistical Learning V2020}
\author{Stefanie Muff, Department of Mathematical Sciences, NTNU}
\date{April xx, 2020}

\begin{document}
\frame{\titlepage}

\begin{frame}{Introduction}

\begin{itemize}
\item
  Neural networks (NN) were first introduced in the 1990's.
\item
  Shift from statistics to computer science and machine learning, as
  they are highly parameterized
\item
  Statisticians were skeptical: ``It's just a nonlinear model''.
\item
  After the first hype, NNs were pushed aside by boosting and support
  vector machines.
\item
  Revival since 2010: The emergence of
  \emph{\textcolor{red}{Deep learning}} as a consequence of improved
  computer resources, some innovations, and applications to image and
  video classification, and speech and text processing
\end{itemize}

\end{frame}

\begin{frame}

\begin{block}{Learning material for this module}

(on the reading list)

\vspace{2mm}

\begin{itemize}
\tightlist
\item
  \href{https://www.math.ntnu.no/emner/TMA4268/2019v/notes/M11L1notes.pdf}{Classnotes
  25.03.2019}
\item
  \href{https://www.math.ntnu.no/emner/TMA4268/2019v/notes/M11L2notes.pdf}{Classnotes
  28.03.2019}
\end{itemize}

See also \emph{References and further reading} (last slide), for further
reading material.

\end{block}

\end{frame}

\begin{frame}

\begin{block}{What will you learn?}

\begin{itemize}
\tightlist
\item
  Why a module on neural networks?
\item
  Translating from statistical to neural networks language

  \begin{itemize}
  \tightlist
  \item
    linear regression
  \item
    logistic regression
  \item
    multiclass (multinomial) regression
  \end{itemize}
\item
  Feedforward networks

  \begin{itemize}
  \tightlist
  \item
    one hidden layer
  \item
    universal approximation theorem
  \end{itemize}
\end{itemize}

\end{block}

\end{frame}

\begin{frame}

\begin{itemize}
\tightlist
\item
  Neural network parts

  \begin{itemize}
  \tightlist
  \item
    model: architecture, activation functions
  \item
    method: loss fuction
  \item
    algorithm: how to estimate parameters, gradient descent and
    back-propagation
  \item
    recent developents
  \end{itemize}
\item
  Deep learning

  \begin{itemize}
  \tightlist
  \item
    the timeline
  \item
    Keras
  \end{itemize}
\item
  References
\end{itemize}

\end{frame}

\begin{frame}

\begin{block}{Why a module on neural networks?}

\begin{itemize}
\tightlist
\item
  Every day you read about the success of AI, machine learning --- and
  in particular deep learning.
\item
  In the last five years the field of deep learning has gone from low
  level performance to excellent performance --- particularly in image
  recognition and speech transcription.
\end{itemize}

\end{block}

\end{frame}

\begin{frame}

\begin{itemize}
\tightlist
\item
  Deep learning is based on a layered artificial neural network
  structure.
\end{itemize}

But, what is a \emph{neural network}?

\end{frame}

\begin{frame}

\centering
\includegraphics[width=0.50000\textwidth]{Neuron3.png}

Neuron and myelinated axon, with signal flow from inputs at dendrites to
outputs at axon terminals. Image credits: By Egm4313.s12 (Prof.~Loc
Vu-Quoc) \url{https://commons.wikimedia.org/w/index.php?curid=72816083}

\end{frame}

\begin{frame}

\begin{itemize}
\tightlist
\item
  Understanding the basic building blocks of artificial neural network,
  and how they are connected to concepts in mathematical statistics
  might be useful!
\item
  The focus here will mainly be to relate what we have learned from
  regression and classification in this course to the basic framework of
  artificial neural network,
\item
  and we will touch upon the new and hot stuff.
\end{itemize}

\end{frame}

\begin{frame}

\begin{itemize}
\tightlist
\item
  There are several (self-study) learning resources (some listed under
  `further references') that the student my turn to for further
  knowledge into deep learning, but this presentation is heavily based
  on Chollet and Allaire (2018), with added formulas and theory.
\item
  There is a new IT3030
  \href{https://www.ntnu.no/studier/emner/IT3030\#tab=omEmnet}{deep
  learning course at NTNU}.
\end{itemize}

\centering
\includegraphics[width=0.30000\textwidth]{DeepLearningwithR.jpeg}
\includegraphics[width=0.30000\textwidth]{DeepLearning.jpeg}

\end{frame}

\begin{frame}

\begin{block}{AI, machine learning and statistics}

\begin{itemize}
\item
  Artificial intelligence dates back to the 1950s, and can be seen as
  \emph{the effort to automate intellectual tasks normally performed by
  humans} (page 4, Chollet and Allaire (2018)).
\item
  AI was first based on hardcoded rules, but turned out to be
  intractable for solving more complex, fuzzy problems.
\item
  With the field of \emph{machine learning} the shift is that a system
  is \emph{trained} rather than explicitly programmed.
\end{itemize}

\end{block}

\end{frame}

\begin{frame}

\begin{block}{Machine learning}

\begin{itemize}
\item
  Machine learning is related to mathematical statistics, but differ in
  many ways (as we will soon see).
\item
  In short, machine learning attachs much larger, and more complex data
  sets than what is usually done in statistics.
\item
  The focus in machine learning is not on mathematical theory, but is
  more oriented towards \emph{engineering}, and ideas are proven
  \emph{empirically} rather than theoretically (which is the case in
  mathematical statistics).
\item
  According to Chollet and Allaire (2018) (page 19):

  \begin{quote}Machine learning isn't mathematics or physics, where major advancements can be done with a pen and a piece of paper. It's an engineering science.\end{quote}
\end{itemize}

\end{block}

\end{frame}

\begin{frame}[fragile]{From statistics to artificial neural networks}

Recapitulate from Module 3 with the bodyfat dataset that contained the
following variables.

\begin{itemize}
\tightlist
\item
  \texttt{bodyfat}: \% of body fat.
\item
  \texttt{age}: age of the person.
\item
  \texttt{weight}: body weighth.
\item
  \texttt{height}: body height.
\item
  \texttt{neck}: neck thickness.
\item
  \texttt{bmi}: bmi.
\item
  \texttt{abdomen}: circumference of abdomen.
\item
  \texttt{hip}: circumference of hip.
\end{itemize}

\end{frame}

\begin{frame}[fragile]

We will now look at modelling the \texttt{bodyfat} as response and using
all other variables as covariates - this will give us

\begin{itemize}
\tightlist
\item
  one numerical output (response), and
\item
  seven covariates
\item
  one intercept
\end{itemize}

Let \(n\) be the number of observations in the training set, here
\(n=243\).

\end{frame}

\begin{frame}

\begin{block}{Multiple linear regression model}

(from Module 3)

\(~\)

We assume

\begin{equation}
 Y_i=\beta_0 + \beta_1 x_{i1}+\beta_2 x_{i2}+\cdots + \beta_p x_{ip}+\varepsilon_i={\boldsymbol x}_i^T{\boldsymbol \beta}+\varepsilon_i \ ,
\end{equation}

for \(i=1,\ldots,n\), where \(x_{ij}\) is the value \(j\)th predictor
for the \(i\)th datapoint, and
\(\boldsymbol{\beta}^\top = (\beta_0,\beta_1,\ldots,\beta_p)\) the
regression coeffficients.

\(~\)

\(~\)

We used the compact matrix notation for all observations
\(i=1,\ldots,n\) together:
\[{\boldsymbol Y}={\boldsymbol {X}} \boldsymbol{\beta}+{\boldsymbol{\varepsilon}}  \ .\]

\end{block}

\end{frame}

\begin{frame}

Assumptions:

\begin{enumerate}
\def\labelenumi{\arabic{enumi}.}
\tightlist
\item
  \(\text{E}(\boldsymbol{\varepsilon})=\bf{0}\).
\item
  \(\text{Cov}(\boldsymbol{\varepsilon})=\text{E}(\boldsymbol{\varepsilon}\boldsymbol{\varepsilon}^T)=\sigma^2\bf{I}\).
\item
  The design matrix has full rank, \(\text{rank}({\boldsymbol X})=p+1\).
  (We assume \(n>>(p+1)\).)
\end{enumerate}

The classical \emph{normal} linear regression model is obtained if
additionally

\begin{enumerate}
\def\labelenumi{\arabic{enumi}.}
\setcounter{enumi}{3}
\tightlist
\item
  \(\boldsymbol\varepsilon\sim N_n({\boldsymbol 0},\sigma^2 {\boldsymbol I})\)
  holds. Here \(N_n\) denotes the \(n\)-dimensional multivarate normal
  distribution.
\end{enumerate}

\end{frame}

\begin{frame}

\begin{block}{From statistical model to network architecture}

\(~\)

We need \emph{new concepts}:

\begin{itemize}
\item
  Our covariates are now presented as \emph{input nodes} in an
  \emph{input layer}.
\item
  The intercept is presented as a node, and is called a \emph{bias}
  node. (New meaning to us!)
\item
  The response is presented as one \emph{output node} in an \emph{output
  layer}.
\item
  The regression coefficients are called \emph{weights} and are often
  written on the lines (arrows) from the inputs to the output node.
\item
  All lines going into the output node signifies that we multiply the
  covariate values in the input nodes with the weights (regression
  coefficients), and then sum. This sum can be sent through a socalled
  \emph{activition function}. The activation function for linear
  regression is just the identity function.
\end{itemize}

\end{block}

\end{frame}

\begin{frame}[fragile]

\scriptsize

\begin{verbatim}
## # weights:  8
## initial  value 4665345.995957 
## iter  10 value 4474.902315
## final  value 4415.454124 
## converged
\end{verbatim}

\begin{center}\includegraphics[width=0.8\linewidth]{11Nnet_files/figure-beamer/motivating-1} \end{center}

\end{frame}

\begin{frame}

The regression function is then rewritten from the linear regression
case

\begin{equation*}
 Y_i=\beta_0 + \beta_1 x_{i1}+\beta_2 x_{i2}+\cdots + \beta_p x_{ip}+\varepsilon_i \ ,
\end{equation*}

to the neural network case

\begin{equation*}
y_1({\boldsymbol x}_i)=\phi_o(w_0+w_1 x_{i1}+\cdots + w_p x_{ip}) \ ,
\end{equation*}

where \(\phi_o(x)=x\).

\end{frame}

\begin{frame}

\begin{itemize}
\item
  We do not say anything of what is random and fixed, and do not make
  any assumption distribution of a random variable.
\item
  In the statistics world we would have written
  \(\hat{y}_1({\boldsymbol x}_i)\) to specify that we are estimating a
  predicted value of the response for the given covariate value. To be
  able to distinguish this predicted response from the observed response
  we use the notation:
  \[ \hat{y}_1({\boldsymbol x}_i)=\phi_o(w_0+w_1 x_{i1}+\cdots + w_p x_{ip})\]
\end{itemize}

The only difference to our MLR model is then that we would have called
the \(w\)s \(\hat{\beta}\)s instead.

\end{frame}

\begin{frame}

\begin{block}{Statistical parameter estimation}

\vspace{2mm}

In multiple linear regression, the parameters \(\boldsymbol\beta\) are
estimated with maximum likelihood and least squares. These two methods
give the same estimator when we assume the normal linear regression
model.

The estimator \(\hat{\boldsymbol \beta}\) is found by minimizing the RSS
for a multiple linear regression model:
\[\begin{aligned} \text{RSS} &=\sum_{i=1}^n (y_i - \hat y_i)^2 = \sum_{i=1}^n (y_i - \hat \beta_0 - \hat \beta_1 x_{i1} - \hat \beta_2 x_{i2} -...-\hat \beta_p x_{ip} )^2 \\
&= \sum_{i=1}^n (y_i-{\boldsymbol x}_i^T \boldsymbol \beta)^2=({\boldsymbol Y}-{\boldsymbol X}\hat{\boldsymbol{\beta}})^T({\boldsymbol Y}-{\boldsymbol X}\hat{\boldsymbol{\beta}}) \ .\end{aligned}\]

This problem has a solution given on closed form:
\[ \hat{\boldsymbol\beta}=({\boldsymbol X}^T{\boldsymbol X})^{-1} {\boldsymbol X}^T {\boldsymbol Y} \ .\]

\end{block}

\end{frame}

\begin{frame}

\begin{block}{Neural networks: loss function and gradient descent}

\vspace{4mm}

We now translate what we did for the regression setup into the neural
networks world: \vspace{2mm}

\begin{enumerate}
\def\labelenumi{\arabic{enumi}.}
\item
  Replace the parameters \(\boldsymbol\beta\) \(\rightarrow\)
  \emph{network weights} \(\boldsymbol w\).
\item
  Replace the RSS in our \emph{training data set} \(\rightarrow\)
  \emph{mean squared error}
  \[ J({\boldsymbol w})=\frac{1}{n}\sum_{i=1}^n (y_i-{\hat{y}_1({\boldsymbol x}_i)})^2 \ ,\]
  where \(J({\boldsymbol w})\) indicates that the unknown parameters are
  the weights \({\boldsymbol w}\).
\end{enumerate}

\end{block}

\end{frame}

\begin{frame}

\begin{enumerate}
\def\labelenumi{\arabic{enumi}.}
\setcounter{enumi}{2}
\tightlist
\item
  Replace \emph{minimizing} the loss function (RSS) via

  \begin{itemize}
  \tightlist
  \item
    calculate the derivative of the loss function with respect to each
    of our parameters
  \item
    solve the \((p+1)\) linear equations.
  \end{itemize}

  \(\rightarrow\)
  \emph{\textcolor{red}{more general minization procedures}} that work
  also when the loss function does not have a closed form.
\end{enumerate}

\end{frame}

\begin{frame}

\begin{block}{Finding optimal weights (schematic description)}

\begin{enumerate}
\def\labelenumi{\arabic{enumi}.}
\item
  Initialize weights by random numerical values.
\item
  Calculate the predicted values \(\hat{y}_1({\boldsymbol x}_i)\) for
  each observation.
\item
  Calculate the loss function \(J(\boldsymbol{w})\), which we want to
  minimize.
\item
  Calculate the \emph{gradient of a function} (it gives the
  \emph{direction of the steepest ascent}) in the \((p+1)\)- dimensional
  space of real numbers (for \(p+1\) weights).
\item
  Update weights in our \(p+1\)-dimensional in the direction of the
  \emph{negative of the gradient} we \emph{decreases the loss function}
  most quickly. The \emph{step length} (``\emph{learning rate}'')
  determines how far we go in an iteration.
\item
  Iterate steps 2--6 until convergence.
\item
  Return final weights.
\end{enumerate}

\end{block}

\end{frame}

\begin{frame}

Figures that give good illustration of the optimization problem.

Chollet and Allaire (2018):

\begin{itemize}
\tightlist
\item
  2.11: SGD down a 1D loess curve
\item
  2.12: Gradient descent down a 2D loss surface
\item
  2.13: local and global minimum
\end{itemize}

(see board)

\end{frame}

\begin{frame}

\begin{block}{Finding optimal weights: Algorithm}

\begin{enumerate}
\def\labelenumi{\arabic{enumi}.}
\tightlist
\item
  Let \(t=0\) and denote the given initial values for the weights
  \({\boldsymbol w}^{(t)}\),
\item
  \emph{propagate} the observations through the network and calculate
  the predictions \({\hat{y}_1({\boldsymbol x}_i)}\)
\item
  calculate the loss function \(J({\boldsymbol w}^{(0)})\),
\item
  find the gradient (direction) in the \((p+1)\)-dimensional space of
  the weights, and evaluate this at the current weight values
  \(\nabla J({\boldsymbol w}^{(0)})={\frac{\partial J}{\partial {\boldsymbol w}}}({\boldsymbol w}^{(0)})\)
\item
  go with a given step length (learning rate) \(\lambda\) in the
  direction of the negative of the gradient of the loss function to get
  updated values for the weights
  \[{\boldsymbol w}^{(t+1)}={\boldsymbol w}^{(t)} - \lambda \nabla J({\boldsymbol w}^{(t)})\]
\item
  Set \(t=t+1\), go to 2. and continue to 6. several times until you
  arrive at a (local) optimum
\item
  The final values of the weights in that \((p+1)\) dimensional space
  are our parameter estimates and your network is \emph{trained} and can
  be used for prediction on a test set.
\end{enumerate}

\end{block}

\end{frame}

\begin{frame}[fragile]

\begin{block}{Backpropagation}

\vspace{2mm}

\begin{itemize}
\tightlist
\item
  In neural networks the gradient part of the gradient descent algorithm
  is implemented efficiently in an algorithm called
  \emph{backpropagation} (see later).
\end{itemize}

\vspace{4mm}

Here we compare

\begin{itemize}
\tightlist
\item
  the MLR solution with \texttt{lm}
\item
  the neural network solution with \texttt{nnet}
  \footnote{which in fact improves upon the gradient descent with Hessian information and the BFGS-algorithmBFGS is a quasi-Newton method (also known as a variable metric algorithm), specifically that published simultaneously in 1970 by Broyden, Fletcher, Goldfarb and Shanno. This uses function values and gradients to build up a picture of the surface to be optimized.}
\end{itemize}

\end{block}

\end{frame}

\begin{frame}[fragile]

\begin{block}{Parameter estimation vs.~network weights}

\scriptsize

\begin{Shaded}
\begin{Highlighting}[]
\NormalTok{fit =}\StringTok{ }\KeywordTok{lm}\NormalTok{(bodyfat }\OperatorTok{~}\StringTok{ }\NormalTok{age }\OperatorTok{+}\StringTok{ }\NormalTok{weight }\OperatorTok{+}\StringTok{ }\NormalTok{height }\OperatorTok{+}\StringTok{ }\NormalTok{bmi }\OperatorTok{+}\StringTok{ }\NormalTok{neck }\OperatorTok{+}\StringTok{ }\NormalTok{abdomen }\OperatorTok{+}\StringTok{ }
\StringTok{    }\NormalTok{hip, }\DataTypeTok{data =}\NormalTok{ d.bodyfat)}
\NormalTok{fitnnet =}\StringTok{ }\KeywordTok{nnet}\NormalTok{(bodyfat }\OperatorTok{~}\StringTok{ }\NormalTok{age }\OperatorTok{+}\StringTok{ }\NormalTok{weight }\OperatorTok{+}\StringTok{ }\NormalTok{height }\OperatorTok{+}\StringTok{ }\NormalTok{bmi }\OperatorTok{+}\StringTok{ }\NormalTok{neck }\OperatorTok{+}\StringTok{ }\NormalTok{abdomen }\OperatorTok{+}\StringTok{ }
\StringTok{    }\NormalTok{hip, }\DataTypeTok{data =}\NormalTok{ d.bodyfat, }\DataTypeTok{linout =} \OtherTok{TRUE}\NormalTok{, }\DataTypeTok{size =} \DecValTok{0}\NormalTok{, }\DataTypeTok{skip =} \OtherTok{TRUE}\NormalTok{, }\DataTypeTok{maxit =} \DecValTok{1000}\NormalTok{, }
    \DataTypeTok{entropy =} \OtherTok{FALSE}\NormalTok{)}
\end{Highlighting}
\end{Shaded}

\begin{verbatim}
## # weights:  8
## initial  value 1756621.204380 
## iter  10 value 4471.392652
## final  value 4415.453729 
## converged
\end{verbatim}

\begin{Shaded}
\begin{Highlighting}[]
\KeywordTok{cbind}\NormalTok{(fitnnet}\OperatorTok{$}\NormalTok{wts, fit}\OperatorTok{$}\NormalTok{coefficients)}
\end{Highlighting}
\end{Shaded}

\begin{verbatim}
##                      [,1]          [,2]
## (Intercept) -9.748906e+01 -9.748903e+01
## age         -9.607667e-04 -9.607669e-04
## weight      -6.292822e-01 -6.292820e-01
## height       3.974885e-01  3.974884e-01
## bmi          1.785331e+00  1.785330e+00
## neck        -4.945725e-01 -4.945725e-01
## abdomen      8.945189e-01  8.945189e-01
## hip         -1.255549e-01 -1.255549e-01
\end{verbatim}

\end{block}

\end{frame}

\begin{frame}[fragile]

\begin{block}{Logistic regression: Diabetes}

\vspace{2mm}

Aim is to predict if a person has
diabetes\footnote{Logistic regression is the "hello world" of machine learning.}.
The data is from a population of women of Pima Indian heritage in the
US, available in the R \texttt{MASS} package. The following information
is available for each woman:

\begin{itemize}
\tightlist
\item
  \texttt{diabetes}: \texttt{0}= not present, \texttt{1}= present
\item
  \texttt{npreg}: number of pregnancies
\item
  \texttt{glu}: plasma glucose concentration in an oral glucose
  tolerance test
\item
  \texttt{bp}: diastolic blood pressure (mmHg)
\item
  \texttt{skin}: triceps skin fold thickness (mm)
\item
  \texttt{bmi}: body mass index (weight in kg/(height in m)\(^2\))
\item
  \texttt{ped}: diabetes pedigree function.
\item
  \texttt{age}: age in years
\end{itemize}

\end{block}

\end{frame}

\begin{frame}

\begin{block}{The statistical model}

\vspace{2mm}

\begin{itemize}
\item
  \(i=1,\ldots, n\) observations in the training set. We will use \(r\)
  (instead of \(p\)) to be the number of covariates, to avoid confusion
  with the probability \(p\).
\item
  The reponse \(Y_i\) is coded (\(\mathcal{C} = \{1, 0\}\) or \{success,
  failure\}) with
\end{itemize}

\[Y_i = \begin{cases} 1 \text{ with probability } p_i, \\ 0 \text{ with probability } 1-p_i. \end{cases}\]

In logistic regression we \emph{link} together our covariates
\({\boldsymbol x}_i\) with this probability \(p_i\) using a
\emph{logistic function}.

\end{block}

\end{frame}

\begin{frame}

\begin{align*}p_i&= \frac{\exp(\beta_0+\beta_1 x_{i1}+\cdots + \beta_r x_{ir})}{1 + \exp(\beta_0 + \beta_1 x_{i1}+\cdots+\beta_r x_{ir})}\\=&\frac{1}{1+\exp(-\beta_0 - \beta_1 x_{i1}-\cdots-\beta_r x_{ir})}
\end{align*}

This function is S-shaped, and ranges between 0 and 1 (so the \(p_i\) is
between 0 and 1).

\begin{center}\includegraphics[width=0.6\linewidth]{11Nnet_files/figure-beamer/unnamed-chunk-3-1} \end{center}

\end{frame}

\begin{frame}

\begin{block}{Parameter estimation in the statistical model}

(Maximum likelihood) \vspace{2mm}

We assume that pairs of covariates and responses \(\{x_i, y_i\}\) are
measured independently of each other. Given \(n\) such observation
pairs, the likelihood function of a logistic regression model can be
written as:
\[L(\boldsymbol{\beta}) = \prod_{i=1}^n L_i(\boldsymbol{\beta}) = \prod_{i=1}^n f(y_i; \boldsymbol{\beta}) = \prod_{i=1}^n (p_i)^{y_i}(1-p_i)^{1-y_i},\]
where
\(\boldsymbol{\beta} = (\beta_0, \beta_1, \beta_2, \ldots, \beta_r)^T\)
enters into \(p_i\).

It is easier (and equivalent) to maximize the log-likelihood:

\[ \ln(L(\boldsymbol{\beta}))=l(\boldsymbol{\beta}) =\sum_{i=1}^n \Big ( y_i \ln p_i + (1-y_i) \ln(1 - p_i )\Big ) \ .\]

\end{block}

\end{frame}

\begin{frame}

\begin{itemize}
\tightlist
\item
  To maximize the log-likelihood function we find the \(r+1\) partial
  derivatives (to form the gradient), and set equal til 0.
\item
  This gives us a set of \(r+1\) non-linear equations in the
  \(\beta_j\)s.
\item
  This set of equations does not have a closed form solution.
\item
  These equations are therefore solved numerically, using the
  \emph{Newton-Raphson algorithm} (or Fisher Scoring):
  \[\beta^{(t+1)}=\beta^{(t)} + {\boldsymbol F}({\boldsymbol \beta}{(t)})^{-1} s(\beta^{(t)}) \ ,\]
  where the gradient of the log-likelihood
  \({\boldsymbol s}({\boldsymbol \beta})=\frac{\partial l}{\partial \boldsymbol \beta}\)
  is called the score vector, and here the new quantity
  \({\boldsymbol F}({\boldsymbol \beta}^{(t)})^{-1}\) is called the
  inverse \emph{expected Fisher information matrix} and is the expected
  value of the negative of the gradient of the score vector (the
  negative of the Hessian matrix of the loglikelihood), and also the
  covariance matrix of the score
  vector.\footnote{Observe that we here are maximizing so we are going in the direction of the gradient, not the negative of the direction which is needed when we minimize.}
\end{itemize}

\end{frame}

\begin{frame}

\begin{block}{The neural network model: architecture and activation
function}

\vspace{2mm}

\begin{itemize}
\item
  Remember: in the neural network (NN) version of the MLR, we had:
  \[ y_1({\boldsymbol x}_i)=\phi_o(w_0+w_1 x_{i1}+\cdots + w_r x_{ir}) \ , \]
  where \(\phi_o(x)=x\).
\item
  In the NN version of logistic regression we instead have the logistic
  function as the function \(\phi_o\), that is
\end{itemize}

\[ \phi_o(x)=\frac{1}{1+\exp(-x)} \ .\]

This is now referred to as the \emph{sigmoid} activation function and is
often denoted \(\sigma(x)\). Again, we prefer to use
\(\hat{y}_1({\boldsymbol x}_i)\) and get:

\[ \hat{y}_1({\boldsymbol x}_i)=\frac{1}{1+\exp(-(w_0+w_1 x_{i1}+\cdots + w_r x_{ir}))} \in (0,1) \ . \]

\end{block}

\end{frame}

\begin{frame}

\end{frame}

\begin{frame}

\begin{block}{Neural networks: loss function and gradient descent}

For parameter estimation we looked at maximizing the log-likelihood of
the statistical model. For neural networks the negative of the binomial
loglikelihood is a scaled version of the \emph{binomial cross-entropy
loss}.

\[ J({\boldsymbol w})=-\frac{1}{n}\sum_{i=1}^n (y_i\ln({\hat{y}_1({\boldsymbol x}_i)})+(1-y_i)\ln(1-{\hat{y}_1({\boldsymbol x}_i)})\]

The optimization is also now done using gradient descent, but observe
that due to the activation function we can use the chain rule when
calculating the partial derivatives to get the gradient direction. The
Algorithm given for MLR is also applicable now, with the modification to
the activation and loss function given here.

\end{block}

\end{frame}

\begin{frame}[fragile]

\begin{block}{Parameter estimation vs.~network weights}

\scriptsize 

\begin{Shaded}
\begin{Highlighting}[]
\NormalTok{fitlogist =}\StringTok{ }\KeywordTok{glm}\NormalTok{(diabetes }\OperatorTok{~}\StringTok{ }\NormalTok{npreg }\OperatorTok{+}\StringTok{ }\NormalTok{glu }\OperatorTok{+}\StringTok{ }\NormalTok{bp }\OperatorTok{+}\StringTok{ }\NormalTok{skin }\OperatorTok{+}\StringTok{ }\NormalTok{bmi }\OperatorTok{+}\StringTok{ }\NormalTok{ped }\OperatorTok{+}\StringTok{ }\NormalTok{age, }
    \DataTypeTok{data =}\NormalTok{ train, }\DataTypeTok{family =} \KeywordTok{binomial}\NormalTok{(}\DataTypeTok{link =} \StringTok{"logit"}\NormalTok{))}
\KeywordTok{summary}\NormalTok{(fitlogist)}
\end{Highlighting}
\end{Shaded}

\begin{verbatim}
## 
## Call:
## glm(formula = diabetes ~ npreg + glu + bp + skin + bmi + ped + 
##     age, family = binomial(link = "logit"), data = train)
## 
## Deviance Residuals: 
##     Min       1Q   Median       3Q      Max  
## -1.9830  -0.6773  -0.3681   0.6439   2.3154  
## 
## Coefficients:
##              Estimate Std. Error z value Pr(>|z|)    
## (Intercept) -9.773062   1.770386  -5.520 3.38e-08 ***
## npreg        0.103183   0.064694   1.595  0.11073    
## glu          0.032117   0.006787   4.732 2.22e-06 ***
## bp          -0.004768   0.018541  -0.257  0.79707    
## skin        -0.001917   0.022500  -0.085  0.93211    
## bmi          0.083624   0.042827   1.953  0.05087 .  
## ped          1.820410   0.665514   2.735  0.00623 ** 
## age          0.041184   0.022091   1.864  0.06228 .  
## ---
## Signif. codes:  0 '***' 0.001 '**' 0.01 '*' 0.05 '.' 0.1 ' ' 1
## 
## (Dispersion parameter for binomial family taken to be 1)
## 
##     Null deviance: 256.41  on 199  degrees of freedom
## Residual deviance: 178.39  on 192  degrees of freedom
## AIC: 194.39
## 
## Number of Fisher Scoring iterations: 5
\end{verbatim}

\end{block}

\end{frame}

\begin{frame}[fragile]

\scriptsize

\begin{Shaded}
\begin{Highlighting}[]
\KeywordTok{library}\NormalTok{(nnet)}
\KeywordTok{library}\NormalTok{(NeuralNetTools)}
\NormalTok{fitnnet =}\StringTok{ }\KeywordTok{nnet}\NormalTok{(diabetes }\OperatorTok{~}\StringTok{ }\NormalTok{npreg }\OperatorTok{+}\StringTok{ }\NormalTok{glu }\OperatorTok{+}\StringTok{ }\NormalTok{bp }\OperatorTok{+}\StringTok{ }\NormalTok{skin }\OperatorTok{+}\StringTok{ }\NormalTok{bmi }\OperatorTok{+}\StringTok{ }\NormalTok{ped }\OperatorTok{+}\StringTok{ }\NormalTok{age, }
    \DataTypeTok{data =}\NormalTok{ train, }\DataTypeTok{linout =} \OtherTok{FALSE}\NormalTok{, }\DataTypeTok{size =} \DecValTok{0}\NormalTok{, }\DataTypeTok{skip =} \OtherTok{TRUE}\NormalTok{, }\DataTypeTok{maxit =} \DecValTok{1000}\NormalTok{, }
    \DataTypeTok{entropy =} \OtherTok{TRUE}\NormalTok{, }\DataTypeTok{Wts =}\NormalTok{ fitlogist}\OperatorTok{$}\NormalTok{coefficients }\OperatorTok{+}\StringTok{ }\KeywordTok{rnorm}\NormalTok{(}\DecValTok{8}\NormalTok{, }\DecValTok{0}\NormalTok{, }\FloatTok{0.1}\NormalTok{))}
\end{Highlighting}
\end{Shaded}

\begin{verbatim}
## # weights:  8
## initial  value 3755.085763 
## iter  10 value 103.256762
## iter  20 value 89.203364
## final  value 89.195333 
## converged
\end{verbatim}

\begin{Shaded}
\begin{Highlighting}[]
\CommentTok{# entropy=TRUE because default is least squares}
\KeywordTok{cbind}\NormalTok{(fitnnet}\OperatorTok{$}\NormalTok{wts, fitlogist}\OperatorTok{$}\NormalTok{coefficients)}
\end{Highlighting}
\end{Shaded}

\begin{verbatim}
##                     [,1]         [,2]
## (Intercept) -9.773061617 -9.773061533
## npreg        0.103183438  0.103183427
## glu          0.032116812  0.032116823
## bp          -0.004767529 -0.004767542
## skin        -0.001916648 -0.001916632
## bmi          0.083623921  0.083623912
## ped          1.820410792  1.820410367
## age          0.041183522  0.041183529
\end{verbatim}

In the \texttt{nnet} R package a slightly different version is used,
\texttt{entropy=maximum\ conditional\ likelihood} which is half the
deviance for the logistic regression model.

\end{frame}

\begin{frame}[fragile]

\begin{Shaded}
\begin{Highlighting}[]
\KeywordTok{plotnet}\NormalTok{(fitnnet)}
\end{Highlighting}
\end{Shaded}

\includegraphics{11Nnet_files/figure-beamer/unnamed-chunk-7-1.pdf}

\end{frame}

\begin{frame}[fragile]

But, there may also exist local minima.

\begin{Shaded}
\begin{Highlighting}[]
\KeywordTok{set.seed}\NormalTok{(}\DecValTok{123}\NormalTok{)}
\NormalTok{fitnnet =}\StringTok{ }\KeywordTok{nnet}\NormalTok{(diabetes }\OperatorTok{~}\StringTok{ }\NormalTok{npreg }\OperatorTok{+}\StringTok{ }\NormalTok{glu }\OperatorTok{+}\StringTok{ }\NormalTok{bp }\OperatorTok{+}\StringTok{ }\NormalTok{skin }\OperatorTok{+}\StringTok{ }\NormalTok{bmi }\OperatorTok{+}\StringTok{ }\NormalTok{ped }\OperatorTok{+}\StringTok{ }\NormalTok{age, }
    \DataTypeTok{data =}\NormalTok{ train, }\DataTypeTok{linout =} \OtherTok{FALSE}\NormalTok{, }\DataTypeTok{size =} \DecValTok{0}\NormalTok{, }\DataTypeTok{skip =} \OtherTok{TRUE}\NormalTok{, }\DataTypeTok{maxit =} \DecValTok{10000}\NormalTok{, }
    \DataTypeTok{entropy =} \OtherTok{TRUE}\NormalTok{, }\DataTypeTok{Wts =}\NormalTok{ fitlogist}\OperatorTok{$}\NormalTok{coefficients }\OperatorTok{+}\StringTok{ }\KeywordTok{rnorm}\NormalTok{(}\DecValTok{8}\NormalTok{, }\DecValTok{0}\NormalTok{, }\DecValTok{1}\NormalTok{))}
\end{Highlighting}
\end{Shaded}

\begin{verbatim}
## # weights:  8
## initial  value 24315.298582 
## final  value 12526.062906 
## converged
\end{verbatim}

\begin{Shaded}
\begin{Highlighting}[]
\KeywordTok{cbind}\NormalTok{(fitnnet}\OperatorTok{$}\NormalTok{wts, fitlogist}\OperatorTok{$}\NormalTok{coefficients)}
\end{Highlighting}
\end{Shaded}

\begin{verbatim}
##                     [,1]         [,2]
## (Intercept)   -36.733537 -9.773061533
## npreg         -77.126994  0.103183427
## glu         -2984.409175  0.032116823
## bp          -1835.934259 -0.004767542
## skin         -718.072629 -0.001916632
## bmi          -818.561311  0.083623912
## ped            -8.687473  1.820410367
## age          -773.023878  0.041183529
\end{verbatim}

\end{frame}

\begin{frame}[fragile]

\begin{block}{Multiclass regression}

\begin{block}{Which type of iris species?}

The \texttt{iris} flower data set was introduced by the British
statistician and biologist Ronald Fisher in 1936, and we studied that in
module 4 on classification.

\begin{itemize}
\tightlist
\item
  Three plant species \{setosa, virginica, versicolor\} (50 observation
  of each), and
\item
  four features: \texttt{Sepal.Length}, \texttt{Sepal.Width},
  \texttt{Petal.Length} and \texttt{Petal.Width}.
\end{itemize}

The aim is to predict the species of an iris plant.

\end{block}

\end{block}

\end{frame}

\begin{frame}

\begin{block}{Statistical model}

We only briefly mentioned this method in module 4 (there our focus was
on LDA and QDA when we had more than two classes), and the full
theoretical treatment is given in the statistics course TMA4315
Generalized linear models, Module 6 on categorical regression (nominal
response).

\textbf{Assumptions:}

We have independent observation pairs \((Y_i,{\boldsymbol x}_i)\), where
the covariate vector \({\boldsymbol x}_i\) consists of the same
measurements for each response category (that is, not different
covariate types that are measured for each response category.

Each observation can only belong to one response class, \(Y_i=c\) for
\(c=1,\ldots, C\), and we define a dummy variable coding of the response
in a \(C\)-dimensional vector:
\({\boldsymbol y}_i=(0,0,\ldots,0,1,0,\ldots,0)\) with a value of \(1\)
in the \(c\)th element of \({\boldsymbol y}_i\) if the class is \(c\).

\end{block}

\end{frame}

\begin{frame}

The probabilities are \(p_{ic}=P(Y_i=c)\) that the response is category
\(c\) for subject \(i\), where \(\sum_{c=1}^C p_{ic}=1\) (which means
that \(p_{iC}=1-\sum_{c=1}^{C-1} p_{ic}\) and in statistics we don't
include \(p_{iC}\) in the modelling).

To analyse this type of data a generalization of the model for the
logistic regression (for two classes) can be used, and the model we fit
is:

\[p_{ic}=P(Y_i=c)= \frac{\exp({\boldsymbol x}_i^T{\boldsymbol \beta}_c)}{1+\sum_{s=1}^{C-1}\exp({\boldsymbol x}_i^T{\boldsymbol \beta}_s)}\]
\[=
\frac{\exp(\beta_{0c}+\beta_{1c} x_{i1}+\cdots + \beta_{rc} x_{ir})}{1 + \sum_{s=1}^{C-1} \exp(\beta_{0s} + \beta_{1s} x_{i1}+\cdots+\beta_{rs} x_{ir})}\]

See the difference to
\(p_i=P(Y_i=1)= \frac{\exp({\boldsymbol x}_i^T{\boldsymbol \beta})}{1+\exp({\boldsymbol x}_i^T{\boldsymbol \beta})}\)
for two classes.

An observation is classified to the class with the highest probability,
\(\text{argmax}p_{ic}\).

\textbf{Q}: How many parameters are we estimating?

\end{frame}

\begin{frame}

\begin{block}{Neural network architecture and activation function}

The neural network uses the dummy variable coding of the responses, but
call this \emph{one-hot} coding, and builds an output layer with \(C\)
nodes --- and corresponding 0/1 targets (responses).

\end{block}

\end{frame}

\begin{frame}

The activation function for the ouput layer is called \emph{softmax} and
is given as

\begin{align}
\hat{y}_1({\boldsymbol x}_i)&= \frac{\exp({\boldsymbol x}_i^T{\boldsymbol w}_1)}{\sum_{s=1}^{C}\exp({\boldsymbol x}_i^T{\boldsymbol w}_s)}\\
{\hat y_2}({\boldsymbol x}_i)& = \frac{\exp({\boldsymbol x}_i^T{\boldsymbol w}_2)}{\sum_{s=1}^{C}\exp({\boldsymbol x}_i^T{\boldsymbol w}_s)}\\
\vdots & = \vdots \\
{\hat y_C}({\boldsymbol x}_i)&= \frac{\exp({\boldsymbol x}_i^T{\boldsymbol w}_C)}{\sum_{s=1}^{C}\exp({\boldsymbol x}_i^T{\boldsymbol w}_s)}
\end{align}

Where each \({\boldsymbol w}_s\) is a \(r+1\) dimensional vector of
weights.

Observe that there is some redundancy here, since
\(\sum_{c=1}^C {\hat y}_{ci}({\boldsymbol x}_i)=1\), so we could have
had \(C-1\) output nodes, but this is not done.

The focus of neural networks is not to interpret the weights, and there
is no need to assume full rank of a matrix with output nodes.

\textbf{Q:} How many parameters are we estimating?

\end{frame}

\begin{frame}

\includegraphics{11Nnet_files/figure-beamer/unnamed-chunk-9-1.pdf}

\end{frame}

\begin{frame}

\begin{block}{Parameter estimation}

The likelihood of the multinomial regression model can be written as

\[ \ln(L({\boldsymbol \beta})\propto \sum_{i=1}^n \sum_{c=1}^C y_{ic}\ln(p_{ic})\]
where \(p_{iC}=1-p_{i1}-p_{i2}-\cdots -p_{i,C-1}\), and the regression
parameters enters via the \(p_{ic}\)s.

Parameter estimation is done in the same way as for the logistic
regression, with the Fisher scoring algorithm (with score vector and
Fisher information matrix).

\end{block}

\end{frame}

\begin{frame}

\begin{block}{Neural networks: loss function and gradient descent}

For parameter estimation we looked at maximizing the log-likelihood of
the statistical model. For neural networks the negative of the
multinomial loglikelihood is a scaled version of the \emph{categorical
cross-entropy loss}.

\[ J({\boldsymbol w})=-\frac{1}{n}\sum_{i=1}^n\frac{1}{C} \sum_{c=1}^C (y_{ic}\ln({\hat{y}_c({\boldsymbol x}_i)})\]

The optimization is done using gradient descent, with minor changes from
what was done for the logistic regression due to the added sum and the
small change in the activation function.

\end{block}

\end{frame}

\begin{frame}[fragile]

\begin{block}{Parameter estimation vs.~network weights}

\begin{Shaded}
\begin{Highlighting}[]
\KeywordTok{library}\NormalTok{(caret)}
\CommentTok{# multiclass model via neural network}
\NormalTok{fit=}\KeywordTok{multinom}\NormalTok{(species}\OperatorTok{~}\NormalTok{.,}\DataTypeTok{family=}\NormalTok{multinomial, }\DataTypeTok{data=}\NormalTok{iris_train)}
\end{Highlighting}
\end{Shaded}

\begin{verbatim}
## # weights:  18 (10 variable)
## initial  value 54.930614 
## iter  10 value 4.798952
## iter  20 value 1.053616
## iter  30 value 0.018109
## iter  40 value 0.007116
## iter  50 value 0.005460
## iter  60 value 0.001449
## iter  70 value 0.001337
## iter  80 value 0.000769
## iter  90 value 0.000768
## iter 100 value 0.000312
## final  value 0.000312 
## stopped after 100 iterations
\end{verbatim}

\begin{Shaded}
\begin{Highlighting}[]
\KeywordTok{summary}\NormalTok{(fit)}
\end{Highlighting}
\end{Shaded}

\begin{verbatim}
## Call:
## multinom(formula = species ~ ., data = iris_train, family = multinomial)
## 
## Coefficients:
##   (Intercept)  Sepal.L.  Sepal.W.  Petal.L.  Petal.W.
## s    16.52211 7.1991829  13.55332 -41.97642  10.72629
## v  -205.03192 0.9502856 -33.88279  10.51631 140.60878
## 
## Std. Errors:
##   (Intercept) Sepal.L. Sepal.W. Petal.L.   Petal.W.
## s    447.9773 2150.290 1523.092 851.1332   89.60559
## v   3713.8154 3471.479 8225.015 388.6877 1589.66680
## 
## Residual Deviance: 0.0006247145 
## AIC: 20.00062
\end{verbatim}

\begin{Shaded}
\begin{Highlighting}[]
\NormalTok{testclass=}\KeywordTok{predict}\NormalTok{(fit,}\DataTypeTok{new=}\NormalTok{iris_test)}
\KeywordTok{confusionMatrix}\NormalTok{(}\DataTypeTok{data=}\NormalTok{testclass,}\DataTypeTok{reference=}\NormalTok{iris_test}\OperatorTok{$}\NormalTok{species)}\OperatorTok{$}\NormalTok{table}
\end{Highlighting}
\end{Shaded}

\begin{verbatim}
##           Reference
## Prediction  c  s  v
##          c 38  0  4
##          s  0 28  0
##          v  0  0 30
\end{verbatim}

\begin{Shaded}
\begin{Highlighting}[]
\KeywordTok{library}\NormalTok{(NeuralNetTools)}
\CommentTok{#plotnet(iris.nnet)}
\KeywordTok{summary}\NormalTok{(iris.nnet)}
\end{Highlighting}
\end{Shaded}

\begin{verbatim}
## a 4-0-3 network with 15 weights
## options were - skip-layer connections  softmax modelling  decay=5e-04
##  b->o1 i1->o1 i2->o1 i3->o1 i4->o1 
##  17.10  -1.04   0.25   2.68 -11.43 
##  b->o2 i1->o2 i2->o2 i3->o2 i4->o2 
##   2.03   3.64   5.52  -8.30  -5.39 
##  b->o3 i1->o3 i2->o3 i3->o3 i4->o3 
## -19.13  -2.60  -5.78   5.63  16.82
\end{verbatim}

\begin{Shaded}
\begin{Highlighting}[]
\KeywordTok{library}\NormalTok{(caret)}
\KeywordTok{table}\NormalTok{(}\KeywordTok{predict}\NormalTok{(iris.nnet, iris_test, }\DataTypeTok{type =} \StringTok{"class"}\NormalTok{),iris_test}\OperatorTok{$}\NormalTok{species)}
\end{Highlighting}
\end{Shaded}

\begin{verbatim}
##    
##      c  s  v
##   c 37  0  4
##   s  0 28  0
##   v  1  0 30
\end{verbatim}

\end{block}

\end{frame}

\begin{frame}

\begin{block}{Summing up}

\begin{itemize}
\tightlist
\item
  Fitting a multiple linear regression

  \begin{itemize}
  \tightlist
  \item
    can be achived by building a neural network with one input layer and
    one node in the output layer,
  \item
    linear activation function,
  \item
    and mean squared loss,
  \end{itemize}
\item
  Classification with two classes can be performed using logistic
  regression, and this

  \begin{itemize}
  \tightlist
  \item
    by building a neural network with one input layer and one node in
    the output layer,
  \item
    sigmoid activation function,
  \item
    and binary cross-entropy loss.
  \item
    Remember: this will only give linear boundaries between the classes
    (in the output space).
  \end{itemize}
\end{itemize}

\end{block}

\end{frame}

\begin{frame}

\begin{itemize}
\tightlist
\item
  Classification with \(C\) classes, \(C>2\), can be performed using
  multinomial regression,

  \begin{itemize}
  \tightlist
  \item
    by building a neural network with one input layer and \(C\) nodes in
    the output layer,
  \item
    softmax activation function,
  \item
    and categorical cross-entropy loss.
  \item
    Also here: this will only give linear boundaries between the classes
    (in the output space).
  \end{itemize}
\end{itemize}

\end{frame}

\begin{frame}

\begin{itemize}
\tightlist
\item
  We have seen that parameters (weights) can be found using gradient
  descent algorithms:

  \begin{itemize}
  \tightlist
  \item
    activation function ``must'' be differentiable
  \item
    step length (learning rate) must be set
  \end{itemize}
\item
  But, we have now only looked at linear models --- which give linar
  boundaries in the classification case --- now we need to move on to
  allow for non-linearities!
\end{itemize}

\end{frame}

\begin{frame}{Feedforward networks}

In feedforward networks we have only connections (weights) forward in
the network, and no feedback connctions that sends the output of the
model back into the network. The word \emph{multi-layer perceptron}
(MLP) and \emph{sequentially layered} networks are also used

The examples of MLR, logistic regression and multiclass regression are
examples of feedfeedward networks without any socalled \emph{hidden
layers} (between the input and output layers).

We may have no hidden layer, one (to be studied next), or many.

The number of hidden layers is called the \emph{depth} of the network,
and the number of nodes in a layer is called the \emph{width} of the
layer.

\end{frame}

\begin{frame}

The idea of using many layers of many nodes is inspired from
neuroscience, but today we don't have the goal to model the brain ---
but instead to approximate function to perform statistical
generalizations and maybe also insight into the problem at hand.

Now we will see how adding \emph{hidden layers} with \emph{non-linear
activation functions} between the input and output layer will make
nonlinear statistical models.

\end{frame}

\begin{frame}

\begin{block}{The single hidden layer feedforward network}

The word neuron and node can be used interchangeably. We stick with
greek letters \(\alpha\) and \(\beta\) for parameters, but call them
weights.

We use the following notation:

\begin{enumerate}
\def\labelenumi{\arabic{enumi}.}
\tightlist
\item
  Inputs (input layer nodes), \(j=1,\dots p\):
  \(x_1, x_2, \ldots, x_p\), or as a vector \({\boldsymbol x}\).
\item
  The nodes in the hidden layer, \(m=1,\ldots, M\):
  \(z_1, z_2, \ldots, z_m\), or as vector \({\boldsymbol z}\), and the
  hidden layer activation function \(\phi_h\). \[
  z_m({\boldsymbol x})=\phi_h(\alpha_{0m}+\sum_{j=1}^p \alpha_{jm}x_{j})
  \] where \(\alpha_{jm}\) is the weight from input \(j\) to hidden node
  \(m\), and \(\alpha_{0m}\) is the bias term for the \(m\)th hidden
  node. The hidden nodes can be thought of as \emph{latent variables}.
\end{enumerate}

\end{block}

\end{frame}

\begin{frame}

\begin{enumerate}
\def\labelenumi{\arabic{enumi}.}
\setcounter{enumi}{2}
\tightlist
\item
  The node(s) in the output layer, \(c=1,\ldots C\):
  \(y_1, y_2, \ldots, y_C\), or as vector \({\boldsymbol y}\), and
  output layer activation function \(\phi_o\). \[
  \hat{y}_c({\boldsymbol x})=\phi_o(\beta_{0c}+\sum_{m=1}^M \beta_{mc}z_{m}({\boldsymbol x}))
  \] where \(\beta_{mc}\) is from hidden neuron \(m\) to ouput node
  \(c\), and \(\beta_{0c}\) is the bias term for the \(c\)th output
  node.
\end{enumerate}

\end{frame}

\begin{frame}

To sum up: \(c=1,\ldots C\): \(y_1, y_2, \ldots, y_C\), where \(y_c\) is
given as

\[
\hat{y}_c({\boldsymbol x})=\phi_o(\beta_{0c}+\sum_{m=1}^M \beta_{mc}z_{m})=\phi_o(\beta_{0c}+\sum_{m=1}^M \beta_{mc}\phi_h(\alpha_{0m}+\sum_{j=1}^p \alpha_{jm}x_{j}))
\]

\textbf{Hands on:}

\begin{itemize}
\tightlist
\item
  Identify p, M, C in the network figure below, and relate that to the
  \(y_{c}({\boldsymbol x})\) equation.
\item
  How many parameters need to be estimated for this network?
\item
  What decides the value of \(p\) and \(C\)?
\item
  What is the connection between problem, \(\phi_o\) and \(C\)?
\end{itemize}

\end{frame}

\begin{frame}

\centering
\includegraphics[width=0.80000\textwidth]{drawNNp3h2o3.png}

\end{frame}

\begin{frame}

\begin{block}{Special case with linear activation function for the
hidden layer}

\[
\hat{y}_c({\boldsymbol x})=\phi_o(\beta_{0c}+\sum_{m=1}^M \beta_{mc}z_{m})=\phi_o(\beta_{0c}+\sum_{m=1}^M \beta_{mc}\phi_h(\alpha_{0m}+\sum_{j=1}^p \alpha_{jm}x_{j}))
\]

Here we assume that \(\phi_h(z)=z\), called linear or identity
activiation:

\[
\hat{y}_c({\boldsymbol x})=\phi_o(\beta_{0c}+\sum_{m=1}^M \beta_{mc}(\alpha_{0m}+\sum_{j=1}^p \alpha_{jm}x_{j}))
\]

\textbf{Q:} Does this look like something you have seen before?

\end{block}

\end{frame}

\begin{frame}

For regression (linear output activation also) we have looked at a
similar model in Module 6: principal component regression. Then the
weights \(\alpha_{jm}\) were not estimated, but were defined to be the
eigenvectors corresponding to the of the \(m\)th largest eigenvalue of
the estimated covariance matrix of the covariates \({\boldsymbol x}_i\).
Therefore we may think of the hidden nodes as \emph{latent variables}.
In PCR only the \(\beta\)s were estimated using the responses (and the
latent variables).

In Module 6 we also touched upon Partial least squares, where the latent
variables were found with the help of the response \(y\). Here both the
\(\alpha\)s and \(\beta\)s were estimated.

In Module 4 we only considered ``ordinary logistic regression'', but we
could have used the principal components in place of the original
covariates --- which then could have been depictured as a feedforward
network with one hidden layer, but where only the \(\beta\)s were
estimated.

\end{frame}

\begin{frame}

\begin{block}{Universal approximation property}

We may think of the goal of a feedforward network to approximate some
function \(f\), mapping our input vector \({\boldsymbol x}\) to an
output value \({\boldsymbol y}\) (as we saw with regression and
classification).

What type of mathematical function can a feedforward neural network with
one hidden layer and linear output activation represent?

\end{block}

\end{frame}

\begin{frame}

According to Goodfellow, Bengio, and Courville (2016), page 198: The
\emph{universal approximation theorem} says that a feedforward network
with

\begin{itemize}
\tightlist
\item
  a \emph{linear output layer}
\item
  at least one hidden layer with a ``squashing'' activation function and
  ``enough'' hidden units
\end{itemize}

can approximate any (Borel measurable) function from one
finite-dimensional space (our input layer) to another (our output layer)
with any desired non-zero amount of error.

The theorem has been shown both for the sigmoid (logistic) and ReLU
activation functions in the hidden layer.

\end{frame}

\begin{frame}

\textbf{The rectified linear unit ReLU} \(\phi_h(a)=\max(0,a)\)

\includegraphics{11Nnet_files/figure-beamer/unnamed-chunk-11-1.pdf}

\end{frame}

\begin{frame}

Earlier, the default activation function for the hiddan layer was the
sigmoid, but now the ReLU is default activiation function to be used in
the hidden layer(s) of a feedforward network --- in particular when more
than one hidden layer is used. The ReLU function is piecewise linear,
but in total non-linear.

\end{frame}

\begin{frame}

Even though a large feedforward network with one hidden layer may be
able to represent a desired function, we may not be able to estimate the
parameters of the function,

\begin{itemize}
\tightlist
\item
  we may choose a too many or too few nodes in the hidden layer
\item
  our optimization routine may fail
\item
  we may overfit/underfit the training data
\end{itemize}

Therefore, sometimes networks with more than one hidden layer is used
--- with fewer total number of nodes but more layers. A network with
many hidden layers is called a \emph{deep network}.

\end{frame}

\begin{frame}

In module 7 we looked at additive models of ``complex'' functions
(splines) of one covariate each.

Now we look at many rather simple non-linear function of linear
combinations of covariates, and non-linear functions of non-linear
functions of linear combinations of covariates.

\textbf{Q:} Is one better than the other when it comes to
interpretation, and to prediction?

\end{frame}

\begin{frame}[fragile]

\begin{block}{The \texttt{nnet} R package}

In order to not make this too complex (since we only have one week to
work with this module), we will use the \texttt{nnet} R package by Brian
Ripley (instead of the currently very popular \texttt{keras} package for
deep learning --- however, we will also present the \texttt{keras}
package for completeness).

\texttt{nnet} fits one hidden layer with sigmoid activiation function.
The implementation is not gradient descent, but instead BFGS using
\texttt{optim}.

\textbf{Description:} Fit single-hidden-layer neural network, possibly
with skip-layer connections.

Usage: (for formula class):
\texttt{nnet(formula,\ data,\ weights,\ ...,\ subset,\ na.action,\ contrasts\ =\ NULL)}

\texttt{nnet(x,\ y,\ ...)}

\end{block}

\end{frame}

\begin{frame}[fragile]

Some arguments to \texttt{nnet()}

\begin{itemize}
\tightlist
\item
  formula, or x and y
\item
  x= input variables, matrix or data frame
\item
  y= response (target) values, if factor=classification
\item
  size: number of units in the hidden layer. Can be zero if there are
  skip-layer units.
\item
  data: the dataset to be used
\item
  subset: index of entries in data that is the training sample
\end{itemize}

If the response in formula is a factor, an appropriate classification
network is constructed; this has one output and entropy fit if the
number of levels is two, and a number of outputs equal to the number of
classes and a softmax output stage for more levels.

\end{frame}

\begin{frame}

\begin{itemize}
\tightlist
\item
  linout: switch for linear output units. Default logistic output units.
\item
  entropy: switch for entropy (= maximum conditional likelihood)
  fitting. Default by least-squares.
\item
  softmax switch for softmax (log-linear model) and maximum conditional
  likelihood fitting. linout, entropy, softmax and censored are mutually
  exclusive.
\item
  censored: A variant on softmax, in which non-zero targets mean
  possible classes. Thus for softmax a row of (0, 1, 1) means one
  example each of classes 2 and 3, but for censored it means one example
  whose class is only known to be 2 or 3.
\item
  skip: switch to add skip-layer connections from input to output.
\item
  decay: parameter for weight decay. Default 0.
\item
  maxit: maximum number of iterations. Default 100.
\item
  MaxNWts: The maximum allowable number of weights. There is no
  intrinsic limit in the code, but increasing MaxNWts will probably
  allow fits that are very slow and time-consuming.
\end{itemize}

\end{frame}

\begin{frame}[fragile]{Examples}

\begin{block}{Boston house prices}

The objective here is to predict the median price of homes in a given
Boston suburb in the mid-1970s, and 10 input variables are given.

This data set is both available in the \texttt{MASS} and \texttt{keras}
R package.

\begin{block}{Preparing the data}

\begin{itemize}
\tightlist
\item
  Few data points: only 506, split between 404 training samples and 102
  test samples (this split already done in the \texttt{keras} library)
\item
  Each feature in the input data (for example, the crime rate) has a
  different scale, some values are proportions, which take values
  between 0 and 1; others take values between 1 and 12, others between 0
  and 100, and so on.
\item
  The targets are the median values of owner-occupied homes, in
  thousands of dollars.
\end{itemize}

\end{block}

\end{block}

\end{frame}

\begin{frame}[fragile]

\begin{Shaded}
\begin{Highlighting}[]
\KeywordTok{library}\NormalTok{(keras)}
\NormalTok{dataset <-}\StringTok{ }\KeywordTok{dataset_boston_housing}\NormalTok{()}
\KeywordTok{c}\NormalTok{(}\KeywordTok{c}\NormalTok{(train_data, train_targets), }\KeywordTok{c}\NormalTok{(test_data, test_targets)) }\OperatorTok\StringTok{ }\NormalTok{dataset}
\KeywordTok{str}\NormalTok{(train_targets)}
\end{Highlighting}
\end{Shaded}

\begin{verbatim}
##  num [1:404(1d)] 15.2 42.3 50 21.1 17.7 18.5 11.3 15.6 15.6 14.4 ...
\end{verbatim}

\begin{Shaded}
\begin{Highlighting}[]
\KeywordTok{head}\NormalTok{(train_data)}
\end{Highlighting}
\end{Shaded}

\begin{verbatim}
##         [,1] [,2]  [,3] [,4]  [,5]  [,6]  [,7]   [,8] [,9] [,10] [,11]
## [1,] 1.23247  0.0  8.14    0 0.538 6.142  91.7 3.9769    4   307  21.0
## [2,] 0.02177 82.5  2.03    0 0.415 7.610  15.7 6.2700    2   348  14.7
## [3,] 4.89822  0.0 18.10    0 0.631 4.970 100.0 1.3325   24   666  20.2
## [4,] 0.03961  0.0  5.19    0 0.515 6.037  34.5 5.9853    5   224  20.2
## [5,] 3.69311  0.0 18.10    0 0.713 6.376  88.4 2.5671   24   666  20.2
## [6,] 0.28392  0.0  7.38    0 0.493 5.708  74.3 4.7211    5   287  19.6
##       [,12] [,13]
## [1,] 396.90 18.72
## [2,] 395.38  3.11
## [3,] 375.52  3.26
## [4,] 396.90  8.01
## [5,] 391.43 14.65
## [6,] 391.13 11.74
\end{verbatim}

\begin{Shaded}
\begin{Highlighting}[]
\KeywordTok{library}\NormalTok{(MASS)  }\CommentTok{#no names for columns - take that from other source}
\KeywordTok{colnames}\NormalTok{(Boston)}
\end{Highlighting}
\end{Shaded}

\begin{verbatim}
##  [1] "crim"    "zn"      "indus"   "chas"    "nox"     "rm"      "age"    
##  [8] "dis"     "rad"     "tax"     "ptratio" "black"   "lstat"   "medv"
\end{verbatim}

\begin{Shaded}
\begin{Highlighting}[]
\KeywordTok{colnames}\NormalTok{(train_data) =}\StringTok{ }\KeywordTok{colnames}\NormalTok{(Boston)[}\OperatorTok{-}\DecValTok{13}\NormalTok{]}
\KeywordTok{colnames}\NormalTok{(test_data) =}\StringTok{ }\KeywordTok{colnames}\NormalTok{(Boston)[}\OperatorTok{-}\DecValTok{13}\NormalTok{]}
\end{Highlighting}
\end{Shaded}

\end{frame}

\begin{frame}[fragile]

A widespread best practice to deal with such data is to do feature-wise
normalization. This is to make the optimization easier with gradient
based methods.

\begin{Shaded}
\begin{Highlighting}[]
\NormalTok{org_train =}\StringTok{ }\NormalTok{train_data}
\NormalTok{mean <-}\StringTok{ }\KeywordTok{apply}\NormalTok{(train_data, }\DecValTok{2}\NormalTok{, mean)}
\NormalTok{std <-}\StringTok{ }\KeywordTok{apply}\NormalTok{(train_data, }\DecValTok{2}\NormalTok{, sd)}
\NormalTok{train_data <-}\StringTok{ }\KeywordTok{scale}\NormalTok{(train_data, }\DataTypeTok{center =}\NormalTok{ mean, }\DataTypeTok{scale =}\NormalTok{ std)}
\NormalTok{test_data <-}\StringTok{ }\KeywordTok{scale}\NormalTok{(test_data, }\DataTypeTok{center =}\NormalTok{ mean, }\DataTypeTok{scale =}\NormalTok{ std)}
\end{Highlighting}
\end{Shaded}

Note that the quantities used for normalizing the test data are computed
using the training data. You should never use in your workflow any
quantity computed on the test data, even for something as simple as data
normalization. This also means that we need to do this standardization
again (from scratch) if we need to do cross-validation.

\end{frame}

\begin{frame}[fragile]

Just checking out one hidden layer with 5 units to get going.

\begin{Shaded}
\begin{Highlighting}[]
\KeywordTok{library}\NormalTok{(nnet)}
\NormalTok{fit5 <-}\StringTok{ }\KeywordTok{nnet}\NormalTok{(train_targets }\OperatorTok{~}\StringTok{ }\NormalTok{., }\DataTypeTok{data =}\NormalTok{ train_data, }\DataTypeTok{size =} \DecValTok{5}\NormalTok{, }\DataTypeTok{linout =} \OtherTok{TRUE}\NormalTok{, }
    \DataTypeTok{maxit =} \DecValTok{1000}\NormalTok{)}
\end{Highlighting}
\end{Shaded}

\begin{verbatim}
## # weights:  76
## initial  value 245152.741465 
## iter  10 value 12420.525228
## iter  20 value 6902.467151
## iter  30 value 5723.819800
## iter  40 value 5177.095194
## iter  50 value 4689.188029
## iter  60 value 4451.205619
## iter  70 value 4357.164555
## iter  80 value 4254.337102
## iter  90 value 4118.662278
## iter 100 value 4026.432205
## iter 110 value 3918.047699
## iter 120 value 3738.715855
## iter 130 value 3406.126780
## iter 140 value 3168.917490
## iter 150 value 3084.137765
## iter 160 value 3069.231409
## iter 170 value 3061.111815
## iter 180 value 3048.847428
## iter 190 value 3021.014988
## iter 200 value 2992.050465
## iter 210 value 2984.054015
## iter 220 value 2978.747128
## iter 230 value 2971.909234
## iter 240 value 2951.028001
## iter 250 value 2936.939591
## iter 260 value 2934.430484
## iter 270 value 2932.502517
## iter 280 value 2931.194886
## iter 290 value 2930.713119
## iter 300 value 2930.496846
## iter 310 value 2930.072463
## iter 320 value 2927.357698
## iter 330 value 2925.042771
## iter 340 value 2924.214962
## iter 350 value 2924.019369
## iter 360 value 2922.190825
## iter 370 value 2921.733842
## iter 380 value 2921.709770
## iter 390 value 2921.673054
## iter 400 value 2921.586000
## iter 410 value 2921.087768
## iter 420 value 2920.013649
## iter 430 value 2918.119987
## iter 440 value 2917.580537
## iter 450 value 2917.529205
## iter 460 value 2917.478376
## iter 470 value 2917.428053
## iter 480 value 2917.255010
## iter 490 value 2917.232905
## iter 500 value 2917.193604
## iter 510 value 2916.542979
## final  value 2916.377014 
## converged
\end{verbatim}

\begin{Shaded}
\begin{Highlighting}[]
\KeywordTok{summary}\NormalTok{(fit5)}
\end{Highlighting}
\end{Shaded}

\begin{verbatim}
## a 13-5-1 network with 76 weights
## options were - linear output units 
##    b->h1   i1->h1   i2->h1   i3->h1   i4->h1   i5->h1   i6->h1   i7->h1 
##  -132.12   -78.30    -9.02    19.32     5.10    14.23    59.89     7.00 
##   i8->h1   i9->h1  i10->h1  i11->h1  i12->h1  i13->h1 
##    21.03    64.26   -60.43    -3.87     7.50   -51.93 
##    b->h2   i1->h2   i2->h2   i3->h2   i4->h2   i5->h2   i6->h2   i7->h2 
##    -3.85    -1.01     0.56    -0.03     0.12    -0.51     0.54    -0.07 
##   i8->h2   i9->h2  i10->h2  i11->h2  i12->h2  i13->h2 
##    -1.24     1.26     1.27    -0.48     0.36    -1.88 
##    b->h3   i1->h3   i2->h3   i3->h3   i4->h3   i5->h3   i6->h3   i7->h3 
##    25.84    -0.73    41.04    -4.75    -2.00    -8.34    -6.28    -3.93 
##   i8->h3   i9->h3  i10->h3  i11->h3  i12->h3  i13->h3 
##   -23.34    -4.65    -5.50    -1.82     2.61   -14.67 
##    b->h4   i1->h4   i2->h4   i3->h4   i4->h4   i5->h4   i6->h4   i7->h4 
## -1124.48  -773.24  -102.16  -329.55   748.05  -657.22  1618.86 -1292.65 
##   i8->h4   i9->h4  i10->h4  i11->h4  i12->h4  i13->h4 
## -2108.62  1346.05 -1746.72  -541.44  1736.48   -43.50 
##    b->h5   i1->h5   i2->h5   i3->h5   i4->h5   i5->h5   i6->h5   i7->h5 
## -1120.64   683.80  -142.16    61.33 -6226.81  -497.87  -446.33     4.21 
##   i8->h5   i9->h5  i10->h5  i11->h5  i12->h5  i13->h5 
##  -255.20  -278.21   826.66   257.90   438.04    -3.16 
##     b->o    h1->o    h2->o    h3->o    h4->o    h5->o 
##    14.28     7.49    24.48     7.41     3.51    -2.83
\end{verbatim}

\begin{Shaded}
\begin{Highlighting}[]
\NormalTok{pred =}\StringTok{ }\KeywordTok{predict}\NormalTok{(fit5, }\DataTypeTok{newdata =}\NormalTok{ test_data, }\DataTypeTok{type =} \StringTok{"raw"}\NormalTok{)}
\KeywordTok{sqrt}\NormalTok{(}\KeywordTok{mean}\NormalTok{((pred[, }\DecValTok{1}\NormalTok{] }\OperatorTok{-}\StringTok{ }\NormalTok{test_targets)}\OperatorTok{^}\DecValTok{2}\NormalTok{))}
\end{Highlighting}
\end{Shaded}

\begin{verbatim}
## [1] 4.977579
\end{verbatim}

\begin{Shaded}
\begin{Highlighting}[]
\KeywordTok{mean}\NormalTok{(}\KeywordTok{abs}\NormalTok{(pred[, }\DecValTok{1}\NormalTok{] }\OperatorTok{-}\StringTok{ }\NormalTok{test_targets))}
\end{Highlighting}
\end{Shaded}

\begin{verbatim}
## [1] 3.29981
\end{verbatim}

\end{frame}

\begin{frame}[fragile]

Now, use cross-validation to find the best number of hidden nodes ---
but that took some time, so only results shown below. Remember the
scaling need to be done within the loop.

\begin{Shaded}
\begin{Highlighting}[]
\NormalTok{grid =}\StringTok{ }\KeywordTok{c}\NormalTok{(}\DecValTok{5}\NormalTok{, }\DecValTok{10}\NormalTok{, }\DecValTok{15}\NormalTok{, }\DecValTok{20}\NormalTok{, }\DecValTok{25}\NormalTok{, }\DecValTok{30}\NormalTok{, }\DecValTok{50}\NormalTok{)}

\NormalTok{k <-}\StringTok{ }\DecValTok{4}
\KeywordTok{set.seed}\NormalTok{(}\DecValTok{123}\NormalTok{)}
\NormalTok{indices <-}\StringTok{ }\KeywordTok{sample}\NormalTok{(}\DecValTok{1}\OperatorTok{:}\KeywordTok{nrow}\NormalTok{(train_data))}
\NormalTok{folds <-}\StringTok{ }\KeywordTok{cut}\NormalTok{(indices, }\DataTypeTok{breaks =}\NormalTok{ k, }\DataTypeTok{labels =} \OtherTok{FALSE}\NormalTok{)}
\CommentTok{# have now assigned the traning data to 4 diffeent folds all of}
\CommentTok{# the same size (101)}

\NormalTok{resmat =}\StringTok{ }\KeywordTok{matrix}\NormalTok{(}\OtherTok{NA}\NormalTok{, }\DataTypeTok{ncol =}\NormalTok{ k, }\DataTypeTok{nrow =} \KeywordTok{length}\NormalTok{(grid))}
\ControlFlowTok{for}\NormalTok{ (j }\ControlFlowTok{in} \DecValTok{1}\OperatorTok{:}\NormalTok{k) \{}
\NormalTok{    thistrain =}\StringTok{ }\NormalTok{(}\DecValTok{1}\OperatorTok{:}\KeywordTok{dim}\NormalTok{(train_data)[}\DecValTok{1}\NormalTok{])[folds }\OperatorTok{!=}\StringTok{ }\NormalTok{j]}
\NormalTok{    thisvalid =}\StringTok{ }\NormalTok{(}\DecValTok{1}\OperatorTok{:}\KeywordTok{dim}\NormalTok{(train_data)[}\DecValTok{1}\NormalTok{])[folds }\OperatorTok{==}\StringTok{ }\NormalTok{j]}
\NormalTok{    mean <-}\StringTok{ }\KeywordTok{apply}\NormalTok{(org_train[thistrain, ], }\DecValTok{2}\NormalTok{, mean)}
\NormalTok{    std <-}\StringTok{ }\KeywordTok{apply}\NormalTok{(org_train[thistrain, ], }\DecValTok{2}\NormalTok{, sd)}
\NormalTok{    new <-}\StringTok{ }\KeywordTok{scale}\NormalTok{(org_train, }\DataTypeTok{center =}\NormalTok{ mean, }\DataTypeTok{scale =}\NormalTok{ std)}
    \ControlFlowTok{for}\NormalTok{ (i }\ControlFlowTok{in} \DecValTok{1}\OperatorTok{:}\KeywordTok{length}\NormalTok{(grid)) \{}
\NormalTok{        thissize =}\StringTok{ }\NormalTok{grid[i]}
        
\NormalTok{        fit =}\StringTok{ }\KeywordTok{nnet}\NormalTok{(train_targets }\OperatorTok{~}\StringTok{ }\NormalTok{., }\DataTypeTok{data =}\NormalTok{ new, }\DataTypeTok{size =}\NormalTok{ thissize, }
            \DataTypeTok{linout =} \OtherTok{TRUE}\NormalTok{, }\DataTypeTok{maxit =} \DecValTok{5000}\NormalTok{)}
\NormalTok{        pred =}\StringTok{ }\KeywordTok{predict}\NormalTok{(fit, }\DataTypeTok{newdata =}\NormalTok{ new[thisvalid, ], }\DataTypeTok{type =} \StringTok{"raw"}\NormalTok{)}
\NormalTok{        resmat[i, j] =}\StringTok{ }\KeywordTok{sum}\NormalTok{((pred[, }\DecValTok{1}\NormalTok{] }\OperatorTok{-}\StringTok{ }\NormalTok{train_targets[thisvalid])}\OperatorTok{^}\DecValTok{2}\NormalTok{)}
\NormalTok{    \}}
\NormalTok{\}}
\NormalTok{mse =}\StringTok{ }\KeywordTok{apply}\NormalTok{(resmat, }\DecValTok{1}\NormalTok{, sum)}\OperatorTok{/}\KeywordTok{nrow}\NormalTok{(train_data)}
\KeywordTok{plot}\NormalTok{(grid, mse, }\DataTypeTok{type =} \StringTok{"l"}\NormalTok{)}
\NormalTok{mse}
\CommentTok{# 7.727735e+00 2.550359e+00 9.582672e-01 3.458643e-01 4.290271e-01}
\CommentTok{# 1.337877e-03 1.786437e-07}
\end{Highlighting}
\end{Shaded}

\end{frame}

\begin{frame}[fragile]

The best model here was the model with 50 nodes, the largest model we
tried. Fitting that model on the full training set and testing on the
test set

\begin{Shaded}
\begin{Highlighting}[]
\KeywordTok{library}\NormalTok{(nnet)}
\NormalTok{fit50 <-}\StringTok{ }\KeywordTok{nnet}\NormalTok{(train_targets }\OperatorTok{~}\StringTok{ }\NormalTok{., }\DataTypeTok{data =}\NormalTok{ train_data, }\DataTypeTok{size =} \DecValTok{50}\NormalTok{, }\DataTypeTok{linout =} \OtherTok{TRUE}\NormalTok{, }
    \DataTypeTok{maxit =} \DecValTok{5000}\NormalTok{)}
\end{Highlighting}
\end{Shaded}

\begin{verbatim}
## # weights:  751
## initial  value 270057.408487 
## iter  10 value 7120.629568
## iter  20 value 1912.248122
## iter  30 value 1045.291511
## iter  40 value 654.075165
## iter  50 value 483.020417
## iter  60 value 381.456615
## iter  70 value 303.378804
## iter  80 value 206.528830
## iter  90 value 117.599100
## iter 100 value 75.531273
## iter 110 value 48.326269
## iter 120 value 31.066476
## iter 130 value 20.833693
## iter 140 value 14.263621
## iter 150 value 10.101553
## iter 160 value 7.434405
## iter 170 value 5.503605
## iter 180 value 4.199943
## iter 190 value 3.403191
## iter 200 value 2.654316
## iter 210 value 1.858079
## iter 220 value 1.304752
## iter 230 value 0.863482
## iter 240 value 0.594627
## iter 250 value 0.404327
## iter 260 value 0.292482
## iter 270 value 0.206237
## iter 280 value 0.142111
## iter 290 value 0.101128
## iter 300 value 0.075254
## iter 310 value 0.053798
## iter 320 value 0.037321
## iter 330 value 0.024749
## iter 340 value 0.013239
## iter 350 value 0.006591
## iter 360 value 0.003741
## iter 370 value 0.001995
## iter 380 value 0.000971
## iter 390 value 0.000412
## iter 400 value 0.000122
## final  value 0.000096 
## converged
\end{verbatim}

\begin{Shaded}
\begin{Highlighting}[]
\KeywordTok{head}\NormalTok{(}\KeywordTok{summary}\NormalTok{(fit50))}
\end{Highlighting}
\end{Shaded}

\begin{verbatim}
## $n
## [1] 13 50  1
## 
## $nunits
## [1] 65
## 
## $nconn
##  [1]   0   0   0   0   0   0   0   0   0   0   0   0   0   0   0  14  28
## [18]  42  56  70  84  98 112 126 140 154 168 182 196 210 224 238 252 266
## [35] 280 294 308 322 336 350 364 378 392 406 420 434 448 462 476 490 504
## [52] 518 532 546 560 574 588 602 616 630 644 658 672 686 700 751
## 
## $conn
##   [1]  0  1  2  3  4  5  6  7  8  9 10 11 12 13  0  1  2  3  4  5  6  7  8
##  [24]  9 10 11 12 13  0  1  2  3  4  5  6  7  8  9 10 11 12 13  0  1  2  3
##  [47]  4  5  6  7  8  9 10 11 12 13  0  1  2  3  4  5  6  7  8  9 10 11 12
##  [70] 13  0  1  2  3  4  5  6  7  8  9 10 11 12 13  0  1  2  3  4  5  6  7
##  [93]  8  9 10 11 12 13  0  1  2  3  4  5  6  7  8  9 10 11 12 13  0  1  2
## [116]  3  4  5  6  7  8  9 10 11 12 13  0  1  2  3  4  5  6  7  8  9 10 11
## [139] 12 13  0  1  2  3  4  5  6  7  8  9 10 11 12 13  0  1  2  3  4  5  6
## [162]  7  8  9 10 11 12 13  0  1  2  3  4  5  6  7  8  9 10 11 12 13  0  1
## [185]  2  3  4  5  6  7  8  9 10 11 12 13  0  1  2  3  4  5  6  7  8  9 10
## [208] 11 12 13  0  1  2  3  4  5  6  7  8  9 10 11 12 13  0  1  2  3  4  5
## [231]  6  7  8  9 10 11 12 13  0  1  2  3  4  5  6  7  8  9 10 11 12 13  0
## [254]  1  2  3  4  5  6  7  8  9 10 11 12 13  0  1  2  3  4  5  6  7  8  9
## [277] 10 11 12 13  0  1  2  3  4  5  6  7  8  9 10 11 12 13  0  1  2  3  4
## [300]  5  6  7  8  9 10 11 12 13  0  1  2  3  4  5  6  7  8  9 10 11 12 13
## [323]  0  1  2  3  4  5  6  7  8  9 10 11 12 13  0  1  2  3  4  5  6  7  8
## [346]  9 10 11 12 13  0  1  2  3  4  5  6  7  8  9 10 11 12 13  0  1  2  3
## [369]  4  5  6  7  8  9 10 11 12 13  0  1  2  3  4  5  6  7  8  9 10 11 12
## [392] 13  0  1  2  3  4  5  6  7  8  9 10 11 12 13  0  1  2  3  4  5  6  7
## [415]  8  9 10 11 12 13  0  1  2  3  4  5  6  7  8  9 10 11 12 13  0  1  2
## [438]  3  4  5  6  7  8  9 10 11 12 13  0  1  2  3  4  5  6  7  8  9 10 11
## [461] 12 13  0  1  2  3  4  5  6  7  8  9 10 11 12 13  0  1  2  3  4  5  6
## [484]  7  8  9 10 11 12 13  0  1  2  3  4  5  6  7  8  9 10 11 12 13  0  1
## [507]  2  3  4  5  6  7  8  9 10 11 12 13  0  1  2  3  4  5  6  7  8  9 10
## [530] 11 12 13  0  1  2  3  4  5  6  7  8  9 10 11 12 13  0  1  2  3  4  5
## [553]  6  7  8  9 10 11 12 13  0  1  2  3  4  5  6  7  8  9 10 11 12 13  0
## [576]  1  2  3  4  5  6  7  8  9 10 11 12 13  0  1  2  3  4  5  6  7  8  9
## [599] 10 11 12 13  0  1  2  3  4  5  6  7  8  9 10 11 12 13  0  1  2  3  4
## [622]  5  6  7  8  9 10 11 12 13  0  1  2  3  4  5  6  7  8  9 10 11 12 13
## [645]  0  1  2  3  4  5  6  7  8  9 10 11 12 13  0  1  2  3  4  5  6  7  8
## [668]  9 10 11 12 13  0  1  2  3  4  5  6  7  8  9 10 11 12 13  0  1  2  3
## [691]  4  5  6  7  8  9 10 11 12 13  0 14 15 16 17 18 19 20 21 22 23 24 25
## [714] 26 27 28 29 30 31 32 33 34 35 36 37 38 39 40 41 42 43 44 45 46 47 48
## [737] 49 50 51 52 53 54 55 56 57 58 59 60 61 62 63
## 
## $nsunits
## [1] 64
## 
## $decay
## [1] 0
\end{verbatim}

\begin{Shaded}
\begin{Highlighting}[]
\NormalTok{pred =}\StringTok{ }\KeywordTok{predict}\NormalTok{(fit50, }\DataTypeTok{newdata =}\NormalTok{ test_data, }\DataTypeTok{type =} \StringTok{"raw"}\NormalTok{)}
\KeywordTok{sqrt}\NormalTok{(}\KeywordTok{mean}\NormalTok{((pred[, }\DecValTok{1}\NormalTok{] }\OperatorTok{-}\StringTok{ }\NormalTok{test_targets)}\OperatorTok{^}\DecValTok{2}\NormalTok{))}
\end{Highlighting}
\end{Shaded}

\begin{verbatim}
## [1] 5.155742
\end{verbatim}

\begin{Shaded}
\begin{Highlighting}[]
\NormalTok{mae =}\StringTok{ }\KeywordTok{mean}\NormalTok{(}\KeywordTok{abs}\NormalTok{(pred[, }\DecValTok{1}\NormalTok{] }\OperatorTok{-}\StringTok{ }\NormalTok{test_targets))}
\NormalTok{mae}
\end{Highlighting}
\end{Shaded}

\begin{verbatim}
## [1] 3.894279
\end{verbatim}

See here:
\url{https://www.math.ntnu.no/emner/TMA4268/2018v/11NN/11-neural_networks_boston_housing.html}
for running two hidden layers of 64 nodes each with \texttt{keras}.
(This gave a mean absolute error on the test set of 2.682524 after early
stopping.)

\end{frame}

\begin{frame}[fragile]

\begin{block}{Using \texttt{nnet}}

\begin{itemize}
\tightlist
\item
  only one hidden layer available
\item
  only sigmoid hidden layer activation function
\item
  optimization done with an optimization method where the learning rate
  is automatically calculated
\item
  weight decay (\(L_2\) regularization as for the ridge regression) is
  available (see below)
\item
  optimization is supposed to be done until convergence
\item
  the possible choice of hyper parameter is the number of notes in the
  hidden layer
\end{itemize}

\textbf{We suggest you use \texttt{nnet} in Compulsory exercise 2,
Problem 3.}

\end{block}

\end{frame}

\begin{frame}{Neural network parts}

We now focus on the different elements of neural networks.

Illustration drawn in class, motivated from Figure 1.9/3.1 from Chollet
and Allaire (2018).

\end{frame}

\begin{frame}

\begin{block}{Model}

\begin{block}{Output layer activation}

These choices have been guided by solutions in statistics (multiple
linear regression, logistic regression, multiclass regression)

\begin{itemize}
\tightlist
\item
  Linear: for regression problems
\item
  Sigmoid: for two-class classification problems
\item
  Softmax: for more than two classes classification problems
\end{itemize}

Remark: it is important that the output activation is matched with an
appropriate loss function.

\end{block}

\end{block}

\end{frame}

\begin{frame}

\begin{block}{Hidden layer activation}

\textbf{Q:} Why can we not just use linar activation function in all
hidden layers?

\end{block}

\end{frame}

\begin{frame}

\textbf{A:} Then each layer would only be able to do linear
transformations of the input data and a deep stack of linear layers
would still implement a linear operation. The activation functions
sigmoid and relu add non-linearity to the model. And, the universial
approximation property is dependent on a squashing type activation
function.

\end{frame}

\begin{frame}

\includegraphics{11Nnet_files/figure-beamer/unnamed-chunk-17-1.pdf}

\end{frame}

\begin{frame}[fragile]

\begin{itemize}
\tightlist
\item
  Sigmoid: \(g(a)=1/(1+\exp(-a))\). The previous standard choice. These
  units were found to saturate at a high value when the input was very
  positive and at a low value when the input was very negative, and that
  they were only strongly sensitive when the input was close to 0. This
  saturation lead to problems in the gradient descent routines.
\end{itemize}

\textbf{However, for our Compulsory exercise 2, Problem 3, there should
be quite ok to use sigmoid hidden layer activation in the \texttt{nnet}
R package (default hidden layer activation is sigmoid, and that can not
be changed).}

\end{frame}

\begin{frame}

\begin{itemize}
\tightlist
\item
  ReLU: \(\phi_h(a)=\max(0,a)\). The standard choice for deep networks
  (many hidden layers and many nodes) today.
\end{itemize}

The function is piecewise linear, but in total non-linear. It has shown
to be easy to use with gradient descent --- even though the function is
not differentiable at 0. As we will touch upon later, we don't expect to
train a network until the gradient is 0. The derivative from the left at
0 is 0, and the derivative from the right is 1. See Goodfellow, Bengio,
and Courville (2016) page 192 for a discussion on this topic.

When initializing the parameter going into a ReLU node, the advice is to
set weight of the bias node to be small and positive, e.g.~0.1, which
makes it likely that node will be active (larger than 0) for most inputs
in the training set.

\end{frame}

\begin{frame}

Goodfellow, Bengio, and Courville (2016), page 226, reports this
(replacing sigmoid with ReLU) to be one of the major changes that have
improved the performance of the feedforward networks.

ReLU can also be motivated from biology.

\begin{itemize}
\tightlist
\item
  For some inputs a biological neuron can be completely inactive
\item
  For some inputs a biological neuron output can be proportional to the
  input
\item
  But, most of the time a biological neuron is inactive.
\end{itemize}

According to Goodfellow, Bengio, and Courville (2016), page 197, hidden
unit design is an \emph{active area of research.}

\end{frame}

\begin{frame}

Other possible choices for hidden layer activation functions are:

\begin{itemize}
\tightlist
\item
  Radial basis functions: as we looked at in Module 9.
\item
  Softplus: \(\phi_h(a)=\ln(1+\exp(a))\)
\item
  Hard tanh: \(\phi_h(a)=\max(-1,\min(1,a))\)
\end{itemize}

\end{frame}

\begin{frame}

\begin{block}{Neural motivation for squashing functions}

\centering
\includegraphics[width=0.80000\textwidth]{Action_potential.png}

\url{https://commons.wikimedia.org/wiki/File:Action_potential.svg}

\end{block}

\end{frame}

\begin{frame}

\begin{block}{Network architecture}

How many nodes in the network and how are the nodes connected to
eachother? This depends on the problem, and here experience is
important.

We will only consider feedforward networks, where all nodes in one layer
is connected to all the nodes in the next layer. The layers are then
\emph{fully connected} and \emph{dense}.

Choosing architecture to us is to chose the \emph{depth of the network
(how many hidden layers)} and the \emph{width of each layer}.

\end{block}

\end{frame}

\begin{frame}

However, the recent practice, see e.g. Chollet and Allaire (2018),
Section 4.5.6 and Goodfellow, Bengio, and Courville (2016), page 229, is
to

\begin{itemize}
\tightlist
\item
  choose a too large network (too many nodes and/or too many layers) so
  that if trained until convergence (optimum) then the this would result
  in overfitting, and
\item
  then use other means to avoid this (various variants of regularization
  and hyperparameter optimization).
\end{itemize}

This makes the choice of network architecture to be to \emph{choose a
large enough network}. See also Hyperparameter optimization below.

\end{frame}

\begin{frame}[fragile]

\begin{block}{Method}

The method part is to choose the loss function for the output layer.

The choice of loss function is closely related to the output layer
activation function. To sum up, the popular problem types, output
activation and loss functions are:

\begin{longtable}[]{@{}lll@{}}
\toprule
Problem & Output activation & Loss function\tabularnewline
\midrule
\endhead
Regression & \texttt{linear} & \texttt{mse}\tabularnewline
Classification (C=2) & \texttt{sigmoid} &
\texttt{binary\_crossentropy}\tabularnewline
Classification (C\textgreater{}2) & \texttt{softmax} &
\texttt{categorical\_crossentropy}\tabularnewline
\bottomrule
\end{longtable}

where the mathematical formulas are given for \(n\) training samples.

\end{block}

\end{frame}

\begin{frame}

Let all network parameters (weights) be denoted by
\({\boldsymbol \theta}\).

\begin{block}{Mean squared error}

\[ J({\boldsymbol \theta})=\frac{1}{n}\sum_{i=1}^n (y_i-{\hat{y}_1({\boldsymbol x}_i)})^2\]
\(\hat{y}_1({\boldsymbol x}_i)\) is the output from the linear output
node, and \(y_i\) is the response.

\end{block}

\end{frame}

\begin{frame}

\begin{block}{Binary cross-entropy}

\[ J({\boldsymbol \theta})=-\frac{1}{n}\sum_{i=1}^n (y_i\ln({\hat{y}_1({\boldsymbol x}_i)})+(1-y_i)\ln(1-{\hat{y}_1({\boldsymbol x}_i)})\]

where \(\hat{y}_1({\boldsymbol x}_i)\) is the output from the sigmoid
output node, and \(y_i\) is the 0/1 observed class.

\end{block}

\end{frame}

\begin{frame}

\begin{block}{Categorical cross-entropy}

\[ J({\boldsymbol \theta})=-\frac{1}{n}\sum_{i=1}^n\frac{1}{C} \sum_{c=1}^C (y_{ic}\ln({\hat{y}_c({\boldsymbol x}_i)})\]
where \(\hat{y}_c({\boldsymbol x}_i)\) is the output from the softmax
output node \(c\) and \(y_{ic}\) is the observed one-hot coding (0/1)
for class \(c\).

\end{block}

\end{frame}

\begin{frame}

\begin{block}{Overall}

Due to how estimation is done (see below), the loss functions chosen
``need'' to be:

\begin{itemize}
\tightlist
\item
  differentiable
\item
  possible to compute for each single training data point (or a
  mini-batch --- to be explained soon)
\end{itemize}

In the 1980-1990s the mean squared error was the prominent loss function
also for classification problems, but this has subsequently changed ---
motivated by the spread of maximum likelihood from statistics to machine
learning.

Observe that we have only given the formula for the loss function, and
not explicitly assumed anything about any probability distribution of
the responses (not even assumed that the responses are random
variables). However, we know which statistical model assumptions would
give the loss functions as related to the negative of the loglikelihood.

\end{block}

\end{frame}

\begin{frame}

\begin{block}{Optimizors}

Let the unknown parameters be denoted \({\boldsymbol \theta}\) (what we
have previously denotes as \(\alpha\)s and \(\beta\)s), and the loss
function to be minimized \(J({\boldsymbol \theta})\).

Illustration: Figure 1.9 from Chollet and Allaire (2018) (again).

\end{block}

\end{frame}

\begin{frame}[fragile]

\begin{block}{Gradient descent}

Let \(\nabla J({\boldsymbol \theta}^{(t)})\) be the gradient of the loss
function evaluated at the current estimate
\({\boldsymbol \theta}^{(t)}\), then a step \(t+1\) of the algorithm the
new values (estimates) of of the parameters are given as:

\[{\boldsymbol \theta}^{(t+1)}={\boldsymbol \theta}^{(t)} - \lambda \nabla_{\boldsymbol \theta} J({\boldsymbol \theta}^{(t)})\]

The \emph{learning rate} \(\lambda\) is often set to some small value.
In \texttt{keras} the default learning rate is \(0.01\).

Remember that the gradient is the vector of partical derivative of the
loss function with respect to each of the parameter in the network.

\textbf{Q}: Why are we moving in the direction of the negative of the
gradient? Why not the positive?

\end{block}

\end{frame}

\begin{frame}

\begin{block}{Mini-batch stochastic gradient descent (SGD)}

The loss function is computed as a mean over all training examples.

\[J({\boldsymbol \theta})=\frac{1}{n}\sum_{i=1}^n J({\boldsymbol x}_i, y_i)\]

This means that the gradient will also be a mean over the gradient
contribution from each training example.

\[\nabla_{\boldsymbol \theta} J({\boldsymbol \theta})=\frac{1}{n}\sum_{i=1}^n \nabla_{\boldsymbol \theta} J({\boldsymbol x}_i, y_i)\]

To build a network that generalizes well, it is important to have many
training examples, but that would make us spend a lot of time and
computer resources at calculating each gradient descent step.

\end{block}

\end{frame}

\begin{frame}

We observe that we may see the gradient as an average over many
individual gradients, and think of this as an estimator for an
expectation. This expectation can we (also) approximate by the average
gradient over just a \emph{mini-batch} (random sample) of the
observations.

The idea here is that the optimizer will converge much faster if they
can rapidly compute approximate estimates of the gradient, instead of
slowly computing the exact gradient (using all training data).

In addition with multicore systems, mini-batches may be processed in
parallell and the batch size is often a power of 2 (32 or 256).

It also turns out that small batches also serves as a regularization
effect maybe due to the variability they bring to the optimization
process.

\end{frame}

\begin{frame}

In the 4th video (on backpropagation) from 3Blue1Brown there is nice
example of one trajectory from gradient decent and one from SGD (13:50
minutes into the video):
\url{https://www.youtube.com/watch?v=tIeHLnjs5U8}

\end{frame}

\begin{frame}

This means that for (mini-batch) stochastic gradient descent we do as
follows:

\begin{enumerate}
\def\labelenumi{\arabic{enumi}.}
\tightlist
\item
  Divide all the training samples randomly into mini-batches.
\item
  For each mini-batch: Make predictions of the reponses in the
  mini-batch in a \emph{forward pass}.
\item
  Compute the loss for the training data in this batch.
\item
  Compute the gradient of the loss with regard to the model's parameters
  (\emph{backward pass}) based on the training data in the batch.
  \(\nabla_{\boldsymbol \theta}^* J({\boldsymbol \theta}^{(t)})\)
\item
  Update all weighs, but just using the average gradient from the
  mini-batch
  \({\boldsymbol \theta}^{(t+1)}={\boldsymbol \theta}^{(t)} - \lambda \nabla_{\boldsymbol \theta} ^* J({\boldsymbol \theta}^{(t)})\)
\item
  Repeat 2-5 until convergence. (Could have gone back to 1, but often
  not done.)
\end{enumerate}

\end{frame}

\begin{frame}

\begin{itemize}
\tightlist
\item
  The algorithm defined above is called mini-batch SGD. The Stochastic
  part comes from the fact that we are randomly sampling batches from
  the training data.
\item
  Stochastic gradient descent (SGD) for size equals to 1.
\item
  Mini-batch SGD is a compromise between SGD (one sample per iteration)
  and full gradient descent (full dataset per iteration)
\end{itemize}

\end{frame}

\begin{frame}

\begin{block}{Backpropagation algorithm}

Computing the analytical expression for the gradient \(\nabla J\) is not
difficult, but the numerical evaluation may be expensive. The
Backpropagation algorithm is an simple and inexpensive way to calculate
the gradient.

The chain rue is used to compute derivatives of functions of other
functions where the derivatives are known, this is done efficiently with
backpropagation.

Backpropagation starts with the value of the loss function and works
backward from the top layers to the bottom layers, applying the chain
rule to compute the contribution that each parameter have in the loss
value.

\end{block}

\end{frame}

\begin{frame}

More information:

\begin{itemize}
\tightlist
\item
  Mathematical details in Goodfellow, Bengio, and Courville (2016)
  Section 6.5 (pages 204-238),
\item
  3Blue1Brown: video overview:
  \url{https://www.youtube.com/watch?v=Ilg3gGewQ5U} and chain rule maths
  \url{https://www.youtube.com/watch?v=tIeHLnjs5U8}
\end{itemize}

\end{frame}

\begin{frame}[fragile]

\begin{block}{Variations of SDG --- with adaptive learning rates}

General concept:

\begin{itemize}
\tightlist
\item
  momentum term: previous gradients are allowed to contribute.
\end{itemize}

Named variants: In \texttt{keras} the ``default optimizer'' is RMSprop.

\begin{itemize}
\tightlist
\item
  AdaGrad: individually adapt the learning rates of all model parameters
  by scaling them inversely proportional to the square root of the sum
  of all their historical squared values. (Nice properties for convex
  problems, but with non-linear hidden activation function we do not
  have a convex problem.)
\item
  RMSprop: modification to AdaGrad in non-convex setting. Scales with
  exponentially weighted moving average instead of all historical
  squared gradient values.
\end{itemize}

\end{block}

\end{frame}

\begin{frame}

\begin{block}{Further reading on optimizers}

\begin{itemize}
\tightlist
\item
  \href{https://keras.io/optimizers/}{Keras documentation for
  Optimizers}
\item
  \href{http://ruder.io/optimizing-gradient-descent/index.html}{An
  overview of gradient descent optimization algorithms}
\item
  \href{http://www.cs.toronto.edu/~tijmen/csc321/slides/lecture_slides_lec6.pdf}{Overview
  of mini-batch gradient descent}
\item
  Andrew Ng explains about RMSprop in Coursera course:
  \href{https://www.coursera.org/learn/deep-neural-network/lecture/BhJlm/rmsprop}{Improving
  Deep Neural Networks: Hyperparameter tuning, Regularization and
  Optimization}
\end{itemize}

\end{block}

\end{frame}

\begin{frame}[fragile]

\begin{block}{Regularization}

Goodfellow, Bengio, and Courville (2016), Chapter 7, define
regularization as \emph{any modification we make to a learning algorithm
that is intended to reduce its generalization error but not its training
error}.

We looked at regularization in Module 6, where our aim was to trade
\emph{increased bias} for \emph{reduced variance}. Another way of
looking at this is we need not focus entirely on finding the model of
``correct size'', but instead find a large model that has been
regularized properly.

In Module 6 we looked in particular at adding a penalty to the loss
function. The penalties we looked at were of type absolute value of
parameter (\(L_1\), lasso, where we looked at this as model selection)
and square value of parameter (\(L_2\), ridge regression). This can also
done for neural networks.

In neural networks, \emph{weight decay} is the expression for adding a
\(L_2\)-penalty to the loss function, and is available in the
\texttt{nnet} R package.

\end{block}

\end{frame}

\begin{frame}

Other versions of regularization are \emph{dataset augmentation} and
\emph{label smoothing}:

\begin{itemize}
\item
  Dataset augmentation means adding fake data to the dataset, in order
  that the trained model will generalize better. For some learning task
  it is straightforward to create the new fake data --- for image data
  this can be done by rotating and scaling the images.
\item
  Label smoothing is motivated by the fact that the training data may
  contain errors in the reponses recorded, and replaced the one-hot
  coding for \(C\) classes with \(\epsilon/(C-1)\) and \(1-\epsilon\)
  for some small \(\epsilon\).
\end{itemize}

\end{frame}

\begin{frame}

\begin{block}{Early stopping}

(Based on Goodfellow, Bengio, and Courville (2016), Section 7.8)

The most commonly used for of regularization is \emph{early stopping}.

If we have chosen a sufficiently large model with the capacity to
overfit the training data, we would observe that the training error
decreases steadily during training, but the error on the validation set
at some point begins to increase.

If we stop the learning early and return the parameters giving the test
performance on the validation set, this model would hopefully be a
better model than if we trained the model until convergence --- and this
model will then also give a better test set error.

\end{block}

\end{frame}

\begin{frame}

It is possible to think of the number of \emph{training steps} as a
hyperparameter. This hyperparameter can easily be set, and the cost is
just that we need to monitor the performance on the validation set
during training. Alternatively, cross-validation can be used.

One strategy is to first find the optimal stopping time for the training
based on the valiation set (or with cross-validation with small data
sets), and then retrain the full training set (including the validation
part) and stop at the selected stopping time.

Why is early stopping a regularization technique? By early stopping the
optimization procedure has only seen a relatively small part of the
parameter space in the neighbourhood of the intitial parameter value.
See Goodfellow, Bengio, and Courville (2016), page 250 for a
relationship with \(L_2\) regularization.

\end{frame}

\begin{frame}

\begin{block}{Hyperparameter optimization}

How to avoid overfitting:

\begin{itemize}
\tightlist
\item
  reduce network size,
\item
  collect more observations,
\item
  regularization (including early stopping and drop-out)
\end{itemize}

The network architacture, the number of batches to run before
terminating the optimzation, the drop-out rate, are all examples of
hyperparameters that need to be chosen in a sensible way before fitting
the final model.

It is important that the hyperparameters are chosen on a validation set
or by cross-validation.

However, a ``popular'' term is \emph{validation-set overfitting} and
refers to using the validation set to decide many hyperparameters, so
many that you may effectively overfit the validation set.

\end{block}

\end{frame}

\begin{frame}

In statistics we use design of experiments to explore these
hyperparameters, and just using marginal grids (one hyperparameter at a
time) is common in machine learning.

Example on DOE:hyperparameter optimization with boosting (which of cause
also can be used for neural networks). Article: Design of experiments
and response surface methodology to tune machine learning
hyperparameters, with a random forest case-study (2018), Gustavo A.
Lujan-Moreno, Phillip R. Howard, Omar G. Rojas, Douglas Montgomery,
Expert Systems with Applications, Volume 109,
\url{https://doi.org/10.1016/j.eswa.2018.05.024}

\end{frame}

\begin{frame}

\begin{block}{Dropout}

(Based on Goodfellow, Bengio, and Courville (2016), Section 7.12, and
Chollet and Allaire (2018) 4.4.3)

Dropout was developed by Geoff Hinton and his students.

\begin{itemize}
\tightlist
\item
  During training: randomly \emph{dropout} (set to zero) outputs in a
  layer. Drop-out rates may be chosen between 0.2 and 0.5.
\item
  During test: not dropout, but scale done the layer output values by a
  factor equal to the drop-out rate (since now more units are active
  than we had during training)
\end{itemize}

Alternatively, the drop-out and scaling (now upscaling) can be done
during training.

\end{block}

\end{frame}

\begin{frame}

One way to look at dropout is on the lines of what we did in Module 8
when we used bootstrapping to produced many data sets and then fitted a
model to each of them and then took the average (bagging). But randomly
dropping out outputs in a layer, this can be looked as mimicking bagging
--- in an efficient way.

See Goodfellow, Bengio, and Courville (2016), Section 7.12 for more
insight into the mathematics behind drop-out.

\end{frame}

\begin{frame}

\url{https://www.reddit.com/r/MachineLearning/comments/4w6tsv/ama_we_are_the_google_brain_team_wed_love_to/}

The following is a direct quotation.

figplucker: \textbf{How was `Dropout' conceived? Was there an `aha'
moment?}

geoffhinton (2 years ago)

There were actually three aha moments. One was in about 2004 when
Radford Neal suggested to me that the brain might be big because it was
learning a large ensemble of models. I thought this would be a very
inefficient use of hardware since the same features would need to be
invented separately by different models. Then I realized that the
``models'' could just be the subset of active neurons. This would allow
combinatorially many models and might explain why randomness in spiking
was helpful.

Soon after that I went to my bank. The tellers kept changing and I asked
one of them why. He said he didn't know but they got moved around a lot.
I figured it must be because it would require cooperation between
employees to successfully defraud the bank. \textbf{This made me realize
that randomly removing a different subset of neurons on each example
would prevent conspiracies and thus reduce overfitting.}

I tried this out rather sloppily (I didn't have an adviser) in 2004 and
it didn't seem to work any better than keeping the squared weights small
so I forgot about it.

Then in 2011, Christos Papadimitriou gave a talk at Toronto in which he
said that the whole point of sexual reproduction was to break up complex
co-adaptations. He may not have said it quite like that, but that's what
I heard. It was clearly the same abstract idea as randomly removing
subsets of the neurons. So I went back and tried harder and in
collaboration with my grad students we showed that it worked really
well.

\end{frame}

\begin{frame}[fragile]

\begin{block}{Metrics}

Provides different forms to measure how well the predictions are
compared with the true values.

\begin{itemize}
\tightlist
\item
  \texttt{accuracy}: Percentage of correct classifications
\item
  \texttt{mae}: Mean absolute error.
\end{itemize}

These metrics can be monitored during training (on validation set) or in
the end on the test set, and can be the basis for the choice when to top
training.

\end{block}

\end{frame}

\begin{frame}{Deep learning}

\begin{block}{Timeline}

(based on Chollet and Allaire (2018))

Neural networks were investigated in ``toy form'' in the 1950s. The
first big step was taken in the 1980s when the backpropagation algorithm
were developed (rediscovered) to perform gradient descent in an
efficient way.

In 1989 (Bell Labs, Yann LeCun) used convolutional neural networks to
classifying handwritten digits, and \emph{LeNet} was used in the US
Postal Service for reading ZIP codes in the 1990s.

Not so much seen (?) activity in the neural network field in the 2000s.

\end{block}

\end{frame}

\begin{frame}

In 2011 neural networks with many layers (and trained with GPUs) were
performing well on image classification tasks. The
\href{http://www.image-net.org/}{\emph{ImageNet}} classification
challenge (classify high resolution colour images into 1k different
categories after training on 1.4M images) was won by solutions with deep
convolutional neural networks (convnets). In 2011 the accuracy was
74.3\%, in 2012 83.6\% and in 2015 96.4\%.

From 2012, convnets is the general solution for computer vision tasks.
Other application areas are natural language processing.

\end{frame}

\begin{frame}

\begin{block}{Deep?}

Deep learning does not mean a deeper understanding, but refers to
sucessive layers of representations - where the number of layers gives
the \emph{depth} of the model. Often tens to hundreds of layers are
used.

Deep neural networks are not seen as models of the brain, and are not
related to neurobiology.

A deep network can be seen as many stages of
\emph{information-destillation} (Chollet and Allaire (2018), page 9),
where each stage performes a simple data transformation. These
transformations are not curated by the data analyst, but is estimated in
the network.

In statistics we first select a set of inputs, then look at how these
inputs should be transformed (projections in simple form or
high-dimensional and nonlinear forms), before we apply some statistical
methods. This transformation step can be called \emph{feature
engineering} and has been automated in deep learning.

\end{block}

\end{frame}

\begin{frame}

\emph{Deep Learning is an algorithm which has no theoretical limitations
of what it can learn; the more data you give and the more computational
time you provide, the better it is.}

Geoffrey Hinton (Google)

\end{frame}

\begin{frame}

In addition, this built-in feature engineering of the deep network is
not performed in a greedy fashion, but \emph{jointly} with
estimating/learning the full model.

The success of deep learning is dependent upon the breakthroughts in
hardware development, expecially with faster CPUs and massively
parallell graphical processing units (GPUs). Tensor processing units
(TPUs) is the next step.

Achievements of deep learning includes high quality (near-human to super
human) image classification, speech recognition, handwriting
transcription, machine translation, digital assistants, autonomous
driving, advertise targeting, web searches, playing Go and chess.

\end{frame}

\begin{frame}

\begin{block}{The R keras package}

Earlier good programming skills in C++ was essential to work in deep
learning. In addition also skills on programming for GPUs were needed
(e.g NVIDIA CUDA programming interface). With the launch of the Keras
library now users may only need basic skills in Python or R.

\href{https://keras.io/}{Keras} can be seen as a way to use LEGO bricks
in deep learning. To quote the web-page:

\emph{Keras is a high-level neural networks API, written in Python and
capable of running on top of TensorFlow, CNTK, or Theano. It was
developed with a focus on enabling fast experimentation. Being able to
go from idea to result with the least possible delay is key to doing
good research.}

\end{block}

\end{frame}

\begin{frame}

\href{https://www.tensorflow.org/}{Tensorflow} is a
\emph{symbolic-tensor manipulation} framework that also can perform
autodifferentiation.

A tensor is defined by its number of axis (below), shape (dimension) and
data type (double, integer, character)

\begin{itemize}
\tightlist
\item
  0D tensor is a scalar
\item
  1D tensor is a vector
\item
  2D tensor is a matrix - and the two \emph{axis} of the tensor is the
  rows and columns
\item
  3D tensor generalization of a matrix, and may be used for time series
  data
\item
  4D tensors may be used for images (samples, height, width, channels),
  and
\item
  5D tensors may be used for video (as 4D, plus frames).
\end{itemize}

Tensor operations (reshaping, dot product) are performed in TensorFlow.

The use of Tensorflow in R Keras is referred to as ``using Tensorflow as
a backend engine''. Tensorflow is the default backend.

\end{frame}

\begin{frame}

More information on the R solution: \url{https://keras.rstudio.com/}

Cheat-sheet for R Keras:
\url{https://github.com/rstudio/cheatsheets/raw/master/keras.pdf}

\end{frame}

\begin{frame}

\begin{block}{Simple data analysis workflow}

\begin{enumerate}
\def\labelenumi{\arabic{enumi}.}
\tightlist
\item
  Select training and test data
\item
  Define the model(s): layers, nodes in layers, activation function
\item
  Configure the learning process by choosing a loss function, an
  optimizer, and some metrics to monitor. If needed decide on an
  evaluation protocol (validation set, cross-validation, iterated
  cross-validation)
\item
  Perform any needed preprocessing of data to fit the choices in 2 and 3
\item
  Fit the model to the data
\item
  Make preditions for test data or use the model
\end{enumerate}

\end{block}

\end{frame}

\begin{frame}

Chollet and Allaire (2018) Section 4.5 has the following recommentations
for step 3+5:

\begin{itemize}
\tightlist
\item
  Develop a first model that performs better than a basic baseline
  (maybe the basic baseline could be standard statistics solutions)
\item
  Develop a model that overfits the data
\item
  Regularize and tune hyperparameters
\end{itemize}

Here is a tutorial:
\url{https://keras.rstudio.com/articles/tutorial_overfit_underfit.html}

\end{frame}

\begin{frame}[fragile]

\begin{block}{MNIST dataset}

This is a larger image version of the handwritten digits data set (a
different version, ZIP-codes is found under Recommended exercises).

This data analysis is based on
\url{https://www.math.ntnu.no/emner/TMA4268/2018v/11NN/8-neural_networks_mnist.html}
and the \texttt{R\ keras} cheat sheet. An advanced version using
convolutional neural nets is found here:
\url{https://www.math.ntnu.no/emner/TMA4268/2018v/11NN/12-neural_networks_convolution_mnist.html}

\end{block}

\end{frame}

\begin{frame}

Objective: classify the digit contained in an image (128 \(\times\) 128
greyscale). Problem type: Multiclass classification based on image data.

\includegraphics{11Nnet_files/figure-beamer/unnamed-chunk-18-1.pdf}

\end{frame}

\begin{frame}[fragile]

Labels for the training data:

\begin{verbatim}
## train_labels
##    0    1    2    3    4    5    6    7    8    9 
## 5923 6742 5958 6131 5842 5421 5918 6265 5851 5949
\end{verbatim}

\end{frame}

\begin{frame}[fragile]

\begin{block}{1. Training and test data}

60 000 images for training and 10 000 images for testing.

\begin{Shaded}
\begin{Highlighting}[]
\CommentTok{# Training data}
\NormalTok{train_images <-}\StringTok{ }\NormalTok{mnist}\OperatorTok{$}\NormalTok{train}\OperatorTok{$}\NormalTok{x}
\NormalTok{train_labels <-}\StringTok{ }\NormalTok{mnist}\OperatorTok{$}\NormalTok{train}\OperatorTok{$}\NormalTok{y}

\CommentTok{# Test data}
\NormalTok{test_images <-}\StringTok{ }\NormalTok{mnist}\OperatorTok{$}\NormalTok{test}\OperatorTok{$}\NormalTok{x}
\NormalTok{test_labels <-}\StringTok{ }\NormalTok{mnist}\OperatorTok{$}\NormalTok{test}\OperatorTok{$}\NormalTok{y}
\NormalTok{org_testlabels <-}\StringTok{ }\NormalTok{test_labels}
\end{Highlighting}
\end{Shaded}

The \texttt{train\_images} is a tensor (generalization of a matrix) with
3 axis, \texttt{(samples,\ height,\ width)}.

\end{block}

\end{frame}

\begin{frame}[fragile]

\begin{block}{2. Defining the model}

In this case we are using a \texttt{layer\_dense} (fully connected)
which expects an input tensor of rank equal to two
\texttt{(sample,\ features)} where each \texttt{sample} should contain
\texttt{28*28=784} pixels. Adding a bias term (intercept) is default for
\texttt{layer\_dense}.

\begin{Shaded}
\begin{Highlighting}[]
\NormalTok{network <-}\StringTok{ }\KeywordTok{keras_model_sequential}\NormalTok{() }\OperatorTok\StringTok{ }\KeywordTok{layer_dense}\NormalTok{(}\DataTypeTok{units =} \DecValTok{512}\NormalTok{, }\DataTypeTok{activation =} \StringTok{"relu"}\NormalTok{, }
    \DataTypeTok{input_shape =} \KeywordTok{c}\NormalTok{(}\DecValTok{28} \OperatorTok{*}\StringTok{ }\DecValTok{28}\NormalTok{)) }\OperatorTok\StringTok{ }\KeywordTok{layer_dense}\NormalTok{(}\DataTypeTok{units =} \DecValTok{10}\NormalTok{, }\DataTypeTok{activation =} \StringTok{"softmax"}\NormalTok{)}
\KeywordTok{summary}\NormalTok{(network)}
\end{Highlighting}
\end{Shaded}

\begin{verbatim}
## Model: "sequential"
## ___________________________________________________________________________
## Layer (type)                     Output Shape                  Param #     
## ===========================================================================
## dense (Dense)                    (None, 512)                   401920      
## ___________________________________________________________________________
## dense_1 (Dense)                  (None, 10)                    5130        
## ===========================================================================
## Total params: 407,050
## Trainable params: 407,050
## Non-trainable params: 0
## ___________________________________________________________________________
\end{verbatim}

As activation function we use \texttt{relu} for the hidden layer, and
\texttt{softmax} for the output layer - since we have 10 classes (where
one is correct each time). We could have included more layers in our
model - and then maybe used early stopping if our model was chosen too
big.

\end{block}

\end{frame}

\begin{frame}[fragile]

\begin{block}{3. Configure the learning process}

We next need to choose the loss function we will use, which is
cross-entropy, and then the version of the optimizer. Here htis is
RMSprop. Finally, which measure - metrics - do we want to monitor in our
training phase? Here we choose accuracy (=percentage correctly
classified).

\begin{Shaded}
\begin{Highlighting}[]
\NormalTok{network }\OperatorTok\StringTok{ }\KeywordTok{compile}\NormalTok{(}\DataTypeTok{optimizer =} \StringTok{"rmsprop"}\NormalTok{, }\DataTypeTok{loss =} \StringTok{"categorical_crossentropy"}\NormalTok{, }
    \DataTypeTok{metrics =} \KeywordTok{c}\NormalTok{(}\StringTok{"accuracy"}\NormalTok{))}
\end{Highlighting}
\end{Shaded}

\end{block}

\end{frame}

\begin{frame}[fragile]

\begin{block}{4. Preprocessing to match with model inputs and outputs}

The training data is scored in an array of dimension (60000, 28, 28) of
type integer with values in the {[}0, 255{]} interval. The data must be
reshaped into the shape the network expects (28*28). In addition the
grey scale values are scales to be in the {[}0, 1{]} interval.

Also, the response must be transformed from 0-10 to a vector of 0s and
1s (dummy variable coding) aka one-hot-coding.

\begin{Shaded}
\begin{Highlighting}[]
\NormalTok{train_images <-}\StringTok{ }\KeywordTok{array_reshape}\NormalTok{(train_images, }\KeywordTok{c}\NormalTok{(}\DecValTok{60000}\NormalTok{, }\DecValTok{28} \OperatorTok{*}\StringTok{ }\DecValTok{28}\NormalTok{))}
\NormalTok{train_images <-}\StringTok{ }\NormalTok{train_images}\OperatorTok{/}\DecValTok{255}
\NormalTok{train_labels <-}\StringTok{ }\KeywordTok{to_categorical}\NormalTok{(train_labels)}

\NormalTok{test_images <-}\StringTok{ }\KeywordTok{array_reshape}\NormalTok{(test_images, }\KeywordTok{c}\NormalTok{(}\DecValTok{10000}\NormalTok{, }\DecValTok{28} \OperatorTok{*}\StringTok{ }\DecValTok{28}\NormalTok{))}
\NormalTok{test_images <-}\StringTok{ }\NormalTok{test_images}\OperatorTok{/}\DecValTok{255}
\NormalTok{test_labels <-}\StringTok{ }\KeywordTok{to_categorical}\NormalTok{(test_labels)}
\end{Highlighting}
\end{Shaded}

\end{block}

\end{frame}

\begin{frame}[fragile]

\begin{block}{5. Fit the model}

\begin{Shaded}
\begin{Highlighting}[]
\NormalTok{fitted<-network }\OperatorTok\StringTok{ }\KeywordTok{fit}\NormalTok{(train_images, train_labels,}
 \DataTypeTok{epochs =} \DecValTok{30}\NormalTok{, }\DataTypeTok{batch_size =} \DecValTok{128}\NormalTok{)}
\KeywordTok{library}\NormalTok{(ggplot2)}
\KeywordTok{plot}\NormalTok{(fitted)}\OperatorTok{+}\KeywordTok{ggtitle}\NormalTok{(}\StringTok{"Fitted model"}\NormalTok{)}
\end{Highlighting}
\end{Shaded}

\centering
\includegraphics[width=0.50000\textwidth]{mnistex.png}

\end{block}

\end{frame}

\begin{frame}[fragile]

\begin{block}{6. Evaluation and prediction}

\begin{Shaded}
\begin{Highlighting}[]
\NormalTok{network }\OperatorTok\StringTok{ }\KeywordTok{evaluate}\NormalTok{(test_images, test_labels)}
\NormalTok{testres =}\StringTok{ }\NormalTok{network }\OperatorTok\StringTok{ }\KeywordTok{predict_classes}\NormalTok{(test_images)}
\CommentTok{# $loss [1] 0.1194063 $acc [1] 0.9827}
\KeywordTok{confusionMatrix}\NormalTok{(}\KeywordTok{factor}\NormalTok{(testres), }\KeywordTok{factor}\NormalTok{(org_testlabels))}
\end{Highlighting}
\end{Shaded}

\begin{Shaded}
\begin{Highlighting}[]
\NormalTok{Confusion Matrix and Statistics}

\NormalTok{          Reference}
\NormalTok{Prediction    }\DecValTok{0}    \DecValTok{1}    \DecValTok{2}    \DecValTok{3}    \DecValTok{4}    \DecValTok{5}    \DecValTok{6}    \DecValTok{7}    \DecValTok{8}    \DecValTok{9}
         \DecValTok{0}  \DecValTok{971}    \DecValTok{0}    \DecValTok{3}    \DecValTok{0}    \DecValTok{1}    \DecValTok{2}    \DecValTok{5}    \DecValTok{1}    \DecValTok{1}    \DecValTok{1}
         \DecValTok{1}    \DecValTok{1} \DecValTok{1128}    \DecValTok{2}    \DecValTok{0}    \DecValTok{0}    \DecValTok{0}    \DecValTok{2}    \DecValTok{2}    \DecValTok{2}    \DecValTok{3}
         \DecValTok{2}    \DecValTok{1}    \DecValTok{1} \DecValTok{1009}    \DecValTok{3}    \DecValTok{3}    \DecValTok{0}    \DecValTok{2}    \DecValTok{7}    \DecValTok{2}    \DecValTok{0}
         \DecValTok{3}    \DecValTok{0}    \DecValTok{1}    \DecValTok{2}  \DecValTok{997}    \DecValTok{0}   \DecValTok{11}    \DecValTok{1}    \DecValTok{3}    \DecValTok{4}    \DecValTok{3}
         \DecValTok{4}    \DecValTok{1}    \DecValTok{0}    \DecValTok{2}    \DecValTok{0}  \DecValTok{969}    \DecValTok{1}    \DecValTok{4}    \DecValTok{2}    \DecValTok{3}    \DecValTok{7}
         \DecValTok{5}    \DecValTok{0}    \DecValTok{1}    \DecValTok{0}    \DecValTok{1}    \DecValTok{0}  \DecValTok{871}    \DecValTok{3}    \DecValTok{0}    \DecValTok{1}    \DecValTok{4}
         \DecValTok{6}    \DecValTok{3}    \DecValTok{2}    \DecValTok{2}    \DecValTok{0}    \DecValTok{3}    \DecValTok{4}  \DecValTok{940}    \DecValTok{0}    \DecValTok{1}    \DecValTok{0}
         \DecValTok{7}    \DecValTok{1}    \DecValTok{1}    \DecValTok{4}    \DecValTok{2}    \DecValTok{1}    \DecValTok{0}    \DecValTok{0} \DecValTok{1002}    \DecValTok{2}    \DecValTok{3}
         \DecValTok{8}    \DecValTok{2}    \DecValTok{1}    \DecValTok{7}    \DecValTok{0}    \DecValTok{0}    \DecValTok{2}    \DecValTok{1}    \DecValTok{4}  \DecValTok{953}    \DecValTok{1}
         \DecValTok{9}    \DecValTok{0}    \DecValTok{0}    \DecValTok{1}    \DecValTok{7}    \DecValTok{5}    \DecValTok{1}    \DecValTok{0}    \DecValTok{7}    \DecValTok{5}  \DecValTok{987}

\NormalTok{Overall Statistics}

\NormalTok{               Accuracy }\OperatorTok{:}\StringTok{ }\FloatTok{0.9827}
                 \DecValTok{95}\NormalTok{% CI }\OperatorTok{:}\StringTok{ }\NormalTok{(}\FloatTok{0.9799}\NormalTok{, }\FloatTok{0.9852}\NormalTok{)}
\NormalTok{    No Information Rate }\OperatorTok{:}\StringTok{ }\FloatTok{0.1135}
\NormalTok{    P}\OperatorTok{-}\NormalTok{Value [Acc }\OperatorTok{>}\StringTok{ }\NormalTok{NIR] }\OperatorTok{:}\StringTok{ }\ErrorTok{<}\StringTok{ }\FloatTok{2.2e-16}

\NormalTok{                  Kappa }\OperatorTok{:}\StringTok{ }\FloatTok{0.9808}
\NormalTok{ Mcnemar}\StringTok{'s Test P-Value : NA}
\end{Highlighting}
\end{Shaded}

\end{block}

\end{frame}

\begin{frame}

\begin{block}{Kaggle}

\href{https://www.kaggle.com/}{Kaggle} has hosted machine-learning
competitions since 2010, and by looking at solutions to competitions it
is possible to get an overview of what works. In 2016-2017 gradient
boosting methods won the competitions with structured data (``shallow''
learning problems), while deeplearning won perceptual problems (as image
classification), Chollet and Allaire (2018) (page 18). Kaggle has helped
(and is helping) the rise in deep learning.

\end{block}

\end{frame}

\begin{frame}

\begin{block}{Other analyses}

\begin{block}{Boston housing price}

taken from Chollet and Allaire (2018):
\url{https://www.math.ntnu.no/emner/TMA4268/2018v/11NN/11-neural_networks_boston_housing.html}

\end{block}

\begin{block}{Movie data base}

\url{https://www.math.ntnu.no/emner/TMA4268/2018v/11NN/9-neural_networks_imdb.html}

\end{block}

\begin{block}{Reuters data}

\url{https://www.math.ntnu.no/emner/TMA4268/2018v/11NN/10-neural_networks_reuters.html}

\end{block}

\end{block}

\end{frame}

\begin{frame}[fragile]{Summing up}

\begin{itemize}
\tightlist
\item
  Feedforward network architecture: mathematical formula - layers of
  multivariate transformed (\texttt{relu}, \texttt{linear},
  \texttt{sigmoid}) inner products - sequentially connected.
\item
  What is the number of parameters that need to be estimated? Intercept
  term (for each layer) is possible and is referred to as ``bias term''.
\item
  Loss function to minimize (on output layer): regression (mean
  squared), classification binary (binary crossentropy), classification
  multiple classes (categorical crossentropy) --- and remember to
  connect to the correct choice of output activiation function: mean
  squared loss goes with linear activiation, binary crossentropy with
  sigmoid, categorical crossentropy with softmax.
\item
  How to minimize the loss function: gradient based (chain rule)
  back-propagation - many variants.
\item
  Technicalities: \texttt{nnet} in R
\item
  Optional: \texttt{keras} in R. Use of tensors. Piping sequential
  layers, piping to estimation and then to evaluation (metrics).
\end{itemize}

\end{frame}

\begin{frame}[fragile]

\begin{block}{Not covered}

(The first two topics are covered in Chollet and Allaire (2018))

Recurrent networks: extending the feedforward network to also have
feedback connections. This is a popular type of network to analyse time
series data and natural language applications.

Convolutional networks: some layers in a sequential network contain
operations specially suitable for grid-like topology (images).
Convolution is used in place of general layer (where we do matrix
multiplication) in at least one layer. A popular operation is
\emph{pooling}.

Explanable AI (XAI): how to use methods on the network or the
predictions of the network to figure out the underlying reasoning of the
network. Popular methods are called LIME (local linear regression),
Shaply (concept from game theory). The \texttt{DALEX} R package contains
different socalled \emph{explainers}
\url{https://arxiv.org/abs/1806.08915}.

\end{block}

\end{frame}

\begin{frame}{References and further reading}

\begin{itemize}
\tightlist
\item
  \url{https://youtu.be/aircAruvnKk} from 3BLUE1BROWN - 4 videos - using
  the MNIST-data set as the running example
\item
  Look at how the hidden layer behave:
  \url{https://playground.tensorflow.org}
\item
  Friedman, Hastie, and Tibshirani (2001),Chapter 11: Neural Networks
\item
  Efron and Hastie (2016), Chapter 18: Neural Networks and Deep Learning
\item
  Chollet and Allaire (2018)
\item
  Goodfellow, Bengio, and Courville (2016) (to be used in IT3030)
\item
  Explaining backpropagation
  \url{http://neuralnetworksanddeeplearning.com/chap2.html}
\item
  Slides from MA8701 (Thiago Martins)
  \url{https://www.math.ntnu.no/emner/MA8701/2019v/DeepLearning/}
\end{itemize}

\end{frame}

\begin{frame}{Acknowledgements}

\hypertarget{refs}{}
\hypertarget{ref-kerasR}{}
Chollet, François, and J. J. Allaire. 2018. \emph{Deep Learning with R}.
Manning Press. \url{https://www.manning.com/books/deep-learning-with-r}.

\hypertarget{ref-casi}{}
Efron, Bradley, and Trevor Hastie. 2016. \emph{Computer Age Statistical
Inference - Algorithms, Evidence, and Data Science}. Cambridge
University Press.

\hypertarget{ref-ESL}{}
Friedman, Jerome, Trevor Hastie, and Robert Tibshirani. 2001. \emph{The
Elements of Statistical Learning}. Vol. 1. Springer series in statistics
New York.

\hypertarget{ref-goodfellow}{}
Goodfellow, Ian, Yoshua Bengio, and Aaron Courville. 2016. \emph{Deep
Learning}. MIT Press.

\end{frame}

\end{document}
